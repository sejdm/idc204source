% Options for packages loaded elsewhere
\PassOptionsToPackage{unicode}{hyperref}
\PassOptionsToPackage{hyphens}{url}
%
\documentclass[
  ignorenonframetext,
]{beamer}
\usepackage{pgfpages}
\usepackage{longtable,booktabs}
\usepackage{array}
\setbeamertemplate{caption}[numbered]
\setbeamertemplate{caption label separator}{: }
\setbeamercolor{caption name}{fg=normal text.fg}
\beamertemplatenavigationsymbolsempty
% Prevent slide breaks in the middle of a paragraph
\widowpenalties 1 10000
\raggedbottom
\setbeamertemplate{part page}{
  \centering
  \begin{beamercolorbox}[sep=16pt,center]{part title}
    \usebeamerfont{part title}\insertpart\par
  \end{beamercolorbox}
}
\setbeamertemplate{section page}{
  \centering
  \begin{beamercolorbox}[sep=12pt,center]{part title}
    \usebeamerfont{section title}\insertsection\par
  \end{beamercolorbox}
}
\setbeamertemplate{subsection page}{
  \centering
  \begin{beamercolorbox}[sep=8pt,center]{part title}
    \usebeamerfont{subsection title}\insertsubsection\par
  \end{beamercolorbox}
}
\AtBeginPart{
  \frame{\partpage}
}
\AtBeginSection{
  \ifbibliography
  \else
    \frame{\sectionpage}
  \fi
}
\AtBeginSubsection{
  \frame{\subsectionpage}
}
\usepackage{lmodern}
\usepackage{amssymb,amsmath}
\usepackage{ifxetex,ifluatex}
\ifnum 0\ifxetex 1\fi\ifluatex 1\fi=0 % if pdftex
  \usepackage[T1]{fontenc}
  \usepackage[utf8]{inputenc}
  \usepackage{textcomp} % provide euro and other symbols
\else % if luatex or xetex
  \usepackage{unicode-math}
  \defaultfontfeatures{Scale=MatchLowercase}
  \defaultfontfeatures[\rmfamily]{Ligatures=TeX,Scale=1}
\fi
% Use upquote if available, for straight quotes in verbatim environments
\IfFileExists{upquote.sty}{\usepackage{upquote}}{}
\IfFileExists{microtype.sty}{% use microtype if available
  \usepackage[]{microtype}
  \UseMicrotypeSet[protrusion]{basicmath} % disable protrusion for tt fonts
}{}
\makeatletter
\@ifundefined{KOMAClassName}{% if non-KOMA class
  \IfFileExists{parskip.sty}{%
    \usepackage{parskip}
  }{% else
    \setlength{\parindent}{0pt}
    \setlength{\parskip}{6pt plus 2pt minus 1pt}}
}{% if KOMA class
  \KOMAoptions{parskip=half}}
\makeatother
\usepackage{xcolor}
\IfFileExists{xurl.sty}{\usepackage{xurl}}{} % add URL line breaks if available
\IfFileExists{bookmark.sty}{\usepackage{bookmark}}{\usepackage{hyperref}}
\hypersetup{
  hidelinks,
  pdfcreator={LaTeX via pandoc}}
\urlstyle{same} % disable monospaced font for URLs
\newif\ifbibliography
\setlength{\emergencystretch}{3em} % prevent overfull lines
\providecommand{\tightlist}{%
  \setlength{\itemsep}{0pt}\setlength{\parskip}{0pt}}
\setcounter{secnumdepth}{-\maxdimen} % remove section numbering

\date{}

\newcommand{\say}[2]{\only<#1>{
    \begin{center}
    {\footnotesize \color{gray}\it #2}
  \end{center}}
}

\begin{document}




\begin{frame}[fragile]
\protect\hypertarget{the-steps}{}

\say{1}{Do not worry too much about the algorithm.}
\say{2}{We will soon see what we mean by ``memory''}
\say{3}{This step is really redundant}
\say{4}{Anything before has already been done}
\say{5}{Find the beginning}
\say{6}{Restart everything}
\say{7}{Let us collect all the states in a set}

  \textbf{Example:} Shift the the string to the right of {\tt \#}.\\~\\\pause 
  

\begin{enumerate}
\item[Step 0:] If the character under the head is 0 or 1, erase it and
store it in the memory; otherwise, accept and halt.\\\pause
\item[Step 1:] Search for the \texttt{\#} toward the right\\\pause
\item[Step 2:] Now search for the first blank character and print the
character stored in memory\\\pause
\item[Step 3:] Search for the \texttt{\#} toward the left\\\pause
\item[Step 4:] Now search for the first blank character, shift to the
right and go to Step 0.\pause
\end{enumerate}

\[S := \{\text{Step 0, Step 1, Step 2, Step 3, Step 4}\}\].

\end{frame}

\begin{frame}{The ``memory''}
\protect\hypertarget{the-memory}{}

\begin{itemize}
\item Not the tape!\pause
\item Fixed size\pause
\item component of the state\pause
\end{itemize}

\[M :=\{ 0, 1, \text{Nothing} \}\].

\end{frame}

\begin{frame}
\protect\hypertarget{the-states}{}

\[S := \{\text{Step 0, Step 1, Step 2, Step 3, Step 4}\}\]
\[M :=\{ 0, 1, \text{Nothing} \}\]

\uncover<2->{States,
$Q \subset S \times M$}

\uncover<3->{Examples,}
\[\only<3->{(\text{Step 2}, 0)}\]
\[\only<4->{(\text{Step 1}, \text{Nothing})}\]

\end{frame}

\begin{frame}
Explicit blank symbol: \(\sqcup\) 
\end{frame}


\begin{frame}


\textbf{Step 0:} \emph{``\alert<1>{If} \alert<2>{the character under focus is a 0 or 1}, \alert<3>{store it in the memory}, \alert<4>{shift right}, \alert<5>{erase the character}, and \alert<6>{go to step 1}; \alert<1>{otherwise}, \alert<7>{accept}''}
  
\[\delta(\alert<8>{(\alert<6>{\uncover<6->{\text{Step 0}}},\alert<3>{\uncover<3->{\text{Nothing}}})},\alert<2,5>{\uncover<2->{x}}) = \begin{cases}
(\alert<9,10>{(\alert<6>{\uncover<6->{\text{Step 1}}}, \alert<3>{\uncover<3->{x}})},\alert<5>{\uncover<5->{\sqcup}} ,\alert<4>{\uncover<4->{\mathrm{R}}}) & \alert<1>{\uncover<1->{\text{if }}} \alert<2>{\uncover<2->{x \text{= \alert<9>{0},\alert<10>{1}}}}\\
(\alert<7>{\uncover<7->{Accept}},\alert<7>{\uncover<7->{x}}, \alert<7>{\uncover<7->{\mathrm{R}}}) & \alert<1>{\uncover<1->{\text{otherwise}}}\\
\end{cases}\]



$\uncover<8->{\alert<8>{q_0:=(\text{Step 0}, \text{Nothing})}}$\\
$\uncover<9->{\alert<9>{q_1:=(\text{Step 1}, 0)}}$\\
$\uncover<10->{\alert<10>{q_1':=(\text{Step 1},1)}}$


\uncover<8->{\[\delta(\alert<8>{\uncover<8->{q_0}}, x) = \begin{cases}
(\alert<9>{\uncover<9->{q_1}},\sqcup , \mathrm{R}) & \text{if } x \text{= \alert<9>{0}}\\
(\alert<10>{\uncover<10->{q_1'}},\sqcup , \mathrm{R}) & \text{if } x \text{= \alert<10>{1}}\\
(Accept,x, \mathrm{R}) & \text{otherwise}\\
\end{cases}
\]}

\end{frame}

\begin{frame}[fragile]
\protect\hypertarget{step-1}{}


\textbf{Step 1:} \emph{``if the character under focus is a \texttt{\#}, then move right and go to Step 2; otherwise, repeat this step.''}\pause

\[\delta((\text{Step 1}, n), x) = \begin{cases}
((\text{Step 2}, n) ,x, \mathrm{R}) & \text{if } x \text{=} \#\\
((\text{Step 1}, n),x, \mathrm{R}) & \text{otherwise}
\end{cases}
\]\pause


$q_2=(\text{Step 2},0)$\\
$q_2'=(\text{Step 2},1)$\pause

\[\delta(q_1, x) = \begin{cases}
(q_2 ,x, \mathrm{R}) & \text{if } x \text{=} \#\\
(q_1,x, \mathrm{R}) & \text{otherwise}\\
\end{cases}
\]

\[\delta(q_1', x) = \begin{cases}
(q_2' ,x, \mathrm{R}) & \text{if } x \text{=} \#\\
(q_1',x, \mathrm{R}) & \text{otherwise}\\
\end{cases}
\]

\end{frame}

\begin{frame}
\protect\hypertarget{step-2}{}


\textbf{Step 2:} \emph{``If the character under focus is blank, erase the memory, move right, and go to Step 3; otherwise, move right.''}\pause

\[\delta((\text{Step 2},n), x) = \begin{cases}
((\text{Step 3},\text{Nothing}),n, \mathrm{R}) & \text{if } x \text{=} \sqcup\\
((\text{Step 2},n),x, \mathrm{R}) & \text{otherwise}\\
\end{cases}
\]\pause


$q_3:=(\text{Step 3},\text{Nothing})$\pause

\[\delta(q_2, x) = \begin{cases}
(q_3,0, \mathrm{R}) & \text{if } x \text{=} \sqcup\\
(q_2,x, \mathrm{R}) & \text{otherwise}\\
\end{cases}
\]

\[\delta(q_2', x) = \begin{cases}
(q_3,1, \mathrm{R}) & \text{if } x \text{=} \sqcup \\
(q_2',x, \mathrm{R}) & \text{otherwise}\\
\end{cases}
\]

\end{frame}

\begin{frame}[fragile]
\protect\hypertarget{step-3}{}

\textbf{Step 3:}
\emph{``If the character under focus is \texttt{\#}, then move left and go to Step 4; otherwise, move left and repeat this step.''}\pause

\[\delta((\text{Step 3},\text{Nothing}), x) = \begin{cases}
((\text{Step 4},\text{Nothing}),x, \mathrm{L}) & \text{if } x \text{=} \#\\
((\text{Step 3},\text{Nothing}),x, \mathrm{L}) & \text{otherwise}\\
\end{cases}
\]\pause


$q_3=(\text{Step 3},\text{Nothing})$\\
$q_4=(\text{Step 4},\text{Nothing})$\pause

\[\delta(q_3, x) = \begin{cases}
(q_4 ,x, \mathrm{L}) & \text{if } x \text{=} \#\\
(q_3,x, \mathrm{L}) & \text{otherwise}\\
\end{cases}
\]

\end{frame}

\begin{frame}{Step 4}
\protect\hypertarget{step-4}{}
\emph{``If the character under focus is a blank, then move right and go to Step 0; otherwise, move left and repeat this step.''}\pause

\[\delta((\text{Step 4},\text{Nothing}), x) = \begin{cases}
((\text{Step 0}, \text{Nothing}),x, \mathrm{R}) & \text{if } x \text{=} \sqcup \\
((\text{Step 4},\text{Nothing}),x, \mathrm{L}) & \text{otherwise}\\
\end{cases}
\]\pause


\[\delta(q_4, x) = \begin{cases}
(q_0 ,x, \mathrm{R}) & \text{if } x \text{=} \sqcup \\
(q_4,x, \mathrm{L}) & \text{otherwise}\\
\end{cases}
\]

\end{frame}

\begin{frame}

  
\only<1>{\[\delta(q_0, x) = \begin{cases}
(q_1,\sqcup , \mathrm{R}) & \text{if } x \text{= 0}\\
(q_1',\sqcup , \mathrm{R}) & \text{if } x \text{= 1}\\
(Accept,x, \mathrm{R}) & \text{otherwise}\\
\end{cases}
\]}

\only<2>{\[\delta(q_1, x) = \begin{cases}
(q_2 ,x, \mathrm{R}) & \text{if } x \text{=} \#\\
(q_1,x, \mathrm{R}) & \text{otherwise}\\
\end{cases}
\]}

\only<3>{\[\delta(q_1', x) = \begin{cases}
(q_2' ,x, \mathrm{R}) & \text{if } x \text{=} \#\\
(q_1',x, \mathrm{R}) & \text{otherwise}\\
\end{cases}
\]}

\only<4>{\[\delta(q_2, x) = \begin{cases}
(q_3,0, \mathrm{R}) & \text{if } x \text{=} \sqcup\\
(q_2,x, \mathrm{R}) & \text{otherwise}\\
\end{cases}
\]}

\only<5>{\[\delta(q_2', x) = \begin{cases}
(q_3,1, \mathrm{R}) & \text{if } x \text{=} \sqcup \\
(q_2',x, \mathrm{R}) & \text{otherwise}\\
\end{cases}
\]}

\only<6>{
\[\delta(q_3, x) = \begin{cases}
(q_4 ,x, \mathrm{L}) & \text{if } x \text{=} \#\\
(q_3,x, \mathrm{L}) & \text{otherwise}\\
\end{cases}
\]}


\only<7>{\[\delta(q_4, x) = \begin{cases}
(q_0 ,x, \mathrm{R}) & \text{if } x \text{=} \sqcup \\
(q_4,x, \mathrm{L}) & \text{otherwise}\\
\end{cases}
\]}

%\end{frame}
%\begin{frame}{The transition table}
%\protect\hypertarget{the-transition-table}{}




\begin{longtable}[]{@{}r|cccc@{}}
& 0 & 1 & \(\sqcup\) & \#\tabularnewline
\hline
\uncover<1->{\(q_0\) & (\(q_1\), \(\sqcup\), R) & (\(q_1'\), \(\sqcup\), R) & (Accept, \(\sqcup\), R) & (Accept, \#, R)\tabularnewline}
\uncover<2->{\(q_1\) & (\(q_1\), 0, R) & (\(q_1\), 1, R) & (\(q_1\), \(\sqcup\), R) & (\(q_2\), \#, R)\tabularnewline}
\uncover<3->{\(q_1'\) & (\(q_1'\), 0, R) & (\(q_1'\), 1, R) & (\(q_1'\), \(\sqcup\), R) & (\(q_2'\), \#, R)\tabularnewline}
\uncover<4->{\(q_2\) & (\(q_2\), 0, R) & (\(q_2\), 1, R) & (\(q_3\), 0, L) & (\(q_2\), \#, R)\tabularnewline}
\uncover<5->{\(q_2'\) & (\(q_2'\), 0, R) & (\(q_2'\), 1, R) & (\(q_3\), 1, L) & (\(q_2'\), \#, R)\tabularnewline}
\uncover<6->{\(q_3\) & (\(q_3\), 0, L) & (\(q_3\), 1, L) & (\(q_3\), \(\sqcup\), L) & (\(q_4\), \#, L)\tabularnewline}
\uncover<7->{\(q_4\) & (\(q_4\), 0, L) & (\(q_4\), 1, L) & (\(q_0\), \(\sqcup\), R) & (\(q_4\), \#, L)}
\end{longtable}


\end{frame}

\begin{frame}

\begin{tabular}{m{5cm}m{0.4cm}m{5cm}}
  \uncover<1->{\textit{Step i: \dots until the character \dots}} & \uncover<2->{$\longrightarrow$ & \textit{if the character under the head is \dots, go to Step (i+1) ; otherwise, \ldots and go to Step i}}
\end{tabular}
\end{frame}

\begin{frame}{Subroutine}

Example,\\~\\
Step i: If string \ldots is equal to \ldots then \ldots else \ldots\\~\\\pause

Step i $\to$ start state of ``equality checking'' Turing machine\\~\\\pause

Accept state of ``equality checking'' Turing machine $\to$ ``then'' state\\~\\\pause

Reject state of ``equality checking'' Turing machine $\to$ ``else'' state


\end{frame}


\begin{frame}{Multitape Turing machine}
\protect\hypertarget{multitape-turing-machine}{}

\((Q, \Sigma, \Gamma, \delta, q_{accept}, q_{reject})\)

\(\delta : Q \times \Gamma^k \to Q \times \Gamma^k \times \{L, R, S\}^k\)

\end{frame}

\begin{frame}[fragile]{Multitape Turing machine}
\protect\hypertarget{multitape-turing-machine-1}{}

\begin{block}{Simulating a tape's head}

~

Actual head:

\begin{verbatim}
  0  |0|  1   1   1
\end{verbatim}

Simulated head, replaces \texttt{0} with \texttt{o}:

\begin{verbatim}
  0   A   1   1   1
\end{verbatim}

\end{block}

\end{frame}

\begin{frame}[fragile]{Multitape Turing machine}
\protect\hypertarget{multitape-turing-machine-2}{}

\begin{block}{Simulating a tape's head}

~

Actual head:

\begin{verbatim}
  0   0  |1|  1   1
\end{verbatim}

Simulated head, replaces \texttt{1} with \texttt{I}:

\begin{verbatim}
  0   0   B   1   1
\end{verbatim}

\end{block}

\end{frame}

\begin{frame}[fragile]{Multitape Turing machine}
\protect\hypertarget{multitape-turing-machine-3}{}

Need more characters in the simulated Turing machine:

\(\Sigma':=\{\) \texttt{0,1,A,B,.} \(\}\)

\begin{longtable}[]{@{}ll@{}}
\toprule
Without head & With head\tabularnewline
\midrule
\endhead
\texttt{0} & \texttt{A}\tabularnewline
\texttt{1} & \texttt{B}\tabularnewline
\emph{blankspace} & \texttt{.}\tabularnewline
\bottomrule
\end{longtable}

\end{frame}

\begin{frame}[fragile]{Multitape Turing machine}
\protect\hypertarget{multitape-turing-machine-4}{}

Two tapes:

\begin{verbatim}
  0  |0|  1   1   1
\end{verbatim}

\begin{verbatim}
  1   0   0  |0|  0
\end{verbatim}

Simulated on one tape:

\begin{verbatim}
  0   A   1   1   1   #   1   0   0   A   0
\end{verbatim}

\end{frame}

\begin{frame}[fragile]{Multitape Turing machine}
\protect\hypertarget{multitape-turing-machine-5}{}

Two tapes:

\begin{verbatim}
  0   0  |1|  1   1
\end{verbatim}

\begin{verbatim}
  1   0  |0|  0   0
\end{verbatim}

Simulated on one tape:

\begin{verbatim}
  0   0   B   1   1   #   1   0   A   0   0
\end{verbatim}

\end{frame}

\begin{frame}[fragile]{Multitape Turing machine}
\protect\hypertarget{multitape-turing-machine-6}{}

Two tapes:

\begin{verbatim}
  0   0   1  |1|  1
\end{verbatim}

\begin{verbatim}
  1   0  |0|  0   0
\end{verbatim}

Simulated on one tape:

\begin{verbatim}
  0   0   1   B   1   #   1   0   A   0   0
\end{verbatim}

\end{frame}

\begin{frame}[fragile]{Multitape Turing machine}
\protect\hypertarget{multitape-turing-machine-7}{}

Two tapes:

\begin{verbatim}
  0   0   1   1  |1|
\end{verbatim}

\begin{verbatim}
  1   0  |0|  0   0
\end{verbatim}

Simulated on one tape:

\begin{verbatim}
  0   0   1   1   B   #   1   0   A   0   0
\end{verbatim}

\end{frame}

\begin{frame}[fragile]{Multitape Turing machine}
\protect\hypertarget{multitape-turing-machine-8}{}

Two tapes:

\begin{verbatim}
  0   0   1   1   1  | |
\end{verbatim}

\begin{verbatim}
  1   0  |0|  0   0
\end{verbatim}

Simulated on one tape:

\begin{verbatim}
  0   0   1   1   1   .   #   1   0   A   0   0
\end{verbatim}

\end{frame}

\begin{frame}[fragile]{Multitape Turing machine}
\protect\hypertarget{multitape-turing-machine-9}{}

To simulate two tapes, the alphabet needed in the single tape
simulation:

\(\Sigma':=\{\) \texttt{0,1,A,B,.,\ \#} \(\}\)

\begin{longtable}[]{@{}ll@{}}
\toprule
Without head & With head\tabularnewline
\midrule
\endhead
\texttt{0} & \texttt{A}\tabularnewline
\texttt{1} & \texttt{B}\tabularnewline
\emph{blankspace} & \texttt{.}\tabularnewline
\bottomrule
\end{longtable}

\texttt{\#}: Tape separator

\end{frame}

\begin{frame}
\begin{enumerate}
\item[Step 1:] Add the input on the ``first simulated'' tape, and keep the other ones blank\pause
\item[Step 2:] Determine the symbols under the virtual heads\pause
\item[Step 3:] Update the tapes, shifting tapes if more space is needed\pause
\end{enumerate}
\end{frame}

\begin{frame}
\begin{enumerate}
\item[S:] Steps\pause
\item[N:=] $\{0,1,\ldots,k\}$ to note the tape number in focus\pause
\item[$\tilde{Q}$:=] $Q\cup \{\text{Nothing}\}$, to note the current simulated state\pause
\item[$\Gamma^k$:] to note the characters under the $k$ simulated heads\pause
\item[$Y$:=] $Q \times \Gamma^k \times \{L, R, S\}^k$, to note the output of $\delta$\pause
\end{enumerate}

$$Q'\subset S \times N \times \tilde{Q}\times \Gamma^k \times Y$$
\end{frame}

\begin{frame}
$((\text{Step 0}, \alert{2}, q_i, (0,1, \text{Nothing}, \ldots), \text{Nothing}),\  \alert{{\tt \#}})$\\ $\to ((\text{Step 0}, \alert{3}, q_i, (0, 1, \text{Nothing}, \ldots), \text{Nothing}), \alert{{\tt \#}}, \text{R}))$\\~\\\pause

$((\text{Step 0}, {\color{blue}3}, q_i, (0,1, \text{Nothing}, \ldots), \text{Nothing}),\  \alert{\text{A}})$\\ $\to ((\text{Step 0}, 3, q_i, (0, 1, \alert{0}, \ldots), \text{Nothing}), \alert{\text{A}}, \text{R}))$\\~\\\pause

$((\text{Step 0}, {\color{blue}3}, q_i, (0,1, \text{Nothing}, \ldots), \text{Nothing}),\  \alert{\text{B}})$\\ $\to ((\text{Step 0}, 3, q_i, (0, 1, \alert{1}, \ldots), \text{Nothing}), \alert{\text{B}}, \text{R}))$

\end{frame}



\begin{frame}
$((\text{Step 1}, k, q_i, (0,1, 0, \ldots), \text{Nothing}),\  x)$\\ $\to ((\text{Step 2}, k, q_i, (0, 1, 0, \ldots), \alert{\delta(q_i, (0,1,0,\ldots))}, x, \text{R}))$\\~\\\pause


Step 3 is a subroutine.
\end{frame}

\begin{frame}
$$(q_0, 0) \to (q_3, 1, \text{R})$$
$$(q_0, 1) \to (q_3, 0, \text{R})$$
$$(q_2, 0) \to (q_0, 1, \text{R})$$
$$\cdots$$

Input: 0101\\~\\\pause
Representation on the tapes:\\~\\
Tape 1: {\tt q 0 , 0 , q 3 , 1 , R | q 0 , 1 , q 3 , 0 , R}\\~\\
Tape 2: {\tt q 2 }\\~\\
Tape 3: {\tt A 1 0 1 }

\end{frame}

\begin{frame}
$$\{(T, s)\ |\ T\text{ is a Turing machine}, s \in \Sigma^*\}$$

$$\{(T, s)\ |\ T\text{ halts on } s\}$$
\end{frame}

\begin{frame}
$$q_0010101011$$
$$0101q_i01011$$
\end{frame}
\end{document}
