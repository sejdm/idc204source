% Options for packages loaded elsewhere
\PassOptionsToPackage{unicode}{hyperref}
\PassOptionsToPackage{hyphens}{url}
%
\documentclass[
  ignorenonframetext,
]{beamer}
\usepackage{pgfpages}
\setbeamertemplate{caption}[numbered]
\setbeamertemplate{caption label separator}{: }
\setbeamercolor{caption name}{fg=normal text.fg}
\beamertemplatenavigationsymbolsempty
% Prevent slide breaks in the middle of a paragraph
\widowpenalties 1 10000
\raggedbottom
\setbeamertemplate{part page}{
  \centering
  \begin{beamercolorbox}[sep=16pt,center]{part title}
    \usebeamerfont{part title}\insertpart\par
  \end{beamercolorbox}
}
\setbeamertemplate{section page}{
  \centering
  \begin{beamercolorbox}[sep=12pt,center]{part title}
    \usebeamerfont{section title}\insertsection\par
  \end{beamercolorbox}
}
\setbeamertemplate{subsection page}{
  \centering
  \begin{beamercolorbox}[sep=8pt,center]{part title}
    \usebeamerfont{subsection title}\insertsubsection\par
  \end{beamercolorbox}
}
\AtBeginPart{
  \frame{\partpage}
}
\AtBeginSection{
  \ifbibliography
  \else
    \frame{\sectionpage}
  \fi
}
\AtBeginSubsection{
  \frame{\subsectionpage}
}
\usepackage{lmodern}
\usepackage{amssymb,amsmath}
\usepackage{ifxetex,ifluatex}
\ifnum 0\ifxetex 1\fi\ifluatex 1\fi=0 % if pdftex
  \usepackage[T1]{fontenc}
  \usepackage[utf8]{inputenc}
  \usepackage{textcomp} % provide euro and other symbols
\else % if luatex or xetex
  \usepackage{unicode-math}
  \defaultfontfeatures{Scale=MatchLowercase}
  \defaultfontfeatures[\rmfamily]{Ligatures=TeX,Scale=1}
\fi
% Use upquote if available, for straight quotes in verbatim environments
\IfFileExists{upquote.sty}{\usepackage{upquote}}{}
\IfFileExists{microtype.sty}{% use microtype if available
  \usepackage[]{microtype}
  \UseMicrotypeSet[protrusion]{basicmath} % disable protrusion for tt fonts
}{}
\makeatletter
\@ifundefined{KOMAClassName}{% if non-KOMA class
  \IfFileExists{parskip.sty}{%
    \usepackage{parskip}
  }{% else
    \setlength{\parindent}{0pt}
    \setlength{\parskip}{6pt plus 2pt minus 1pt}}
}{% if KOMA class
  \KOMAoptions{parskip=half}}
\makeatother
\usepackage{xcolor}
\IfFileExists{xurl.sty}{\usepackage{xurl}}{} % add URL line breaks if available
\IfFileExists{bookmark.sty}{\usepackage{bookmark}}{\usepackage{hyperref}}
\hypersetup{
  hidelinks,
  pdfcreator={LaTeX via pandoc}}
\urlstyle{same} % disable monospaced font for URLs
\newif\ifbibliography
\setlength{\emergencystretch}{3em} % prevent overfull lines
\providecommand{\tightlist}{%
  \setlength{\itemsep}{0pt}\setlength{\parskip}{0pt}}
\setcounter{secnumdepth}{-\maxdimen} % remove section numbering

\date{}

\begin{document}

\begin{frame}[fragile]
\protect\hypertarget{the-steps}{}

\begin{enumerate}
\item[Step 0:] If the character under the head is 0 or 1, erase it and
store it in the memory; otherwise, accept and halt.\\\pause
\item[Step 1:] Search for the \texttt{\#} toward the right\\\pause
\item[Step 2:] Now search for the first blank character and print the
character stored in memory\\\pause
\item[Step 3:] Search for the \texttt{\#} toward the left\\\pause
\item[Step 4:] Now search for the first blank character, shift to the
right and go to Step 0.\pause
\end{enumerate}

\[S := \{\text{Step 0, Step 1, Step 2, Step 3, Step 4}\}\].

\end{frame}

\begin{frame}{The ``memory''}
\protect\hypertarget{the-memory}{}

\begin{itemize}
\item Not the tape!\pause
\item Fixed size\pause
\item component of the state\pause
\end{itemize}

\[M :=\{ 0, 1, \text{Nothing} \}\].

\end{frame}

\begin{frame}
\protect\hypertarget{the-states}{}


States,
$Q \subset S \times M$

explicit blank symbol: \(\sqcup\) 
\end{frame}


\begin{frame}


\textbf{Step 0:} \emph{``\alert<1>{If} \alert<2>{the character under focus is a 0 or 1}, \alert<3>{store it in the memory}, \alert<4>{shift right}, \alert<5>{erase the character}, and \alert<6>{go to step 1}; \alert<1>{otherwise}, \alert<7>{accept}''}
  
\[\delta(\alert<8>{(\alert<6>{\uncover<6->{\text{Step 0}}},\alert<3>{\uncover<3->{\text{Nothing}}})},\alert<2,5>{\uncover<2->{x}}) = \begin{cases}
(\alert<9,10>{(\alert<6>{\uncover<6->{\text{Step 1}}}, \alert<3>{\uncover<3->{x}})},\alert<5>{\uncover<5->{\sqcup}} ,\alert<4>{\uncover<4->{\mathrm{R}}}) & \alert<1>{\uncover<1->{\text{if }}} \alert<2>{\uncover<2->{x \text{= \alert<9>{0},\alert<10>{1}}}}\\
(\alert<7>{\uncover<7->{Accept}},\alert<7>{\uncover<7->{x}}, \alert<7>{\uncover<7->{\mathrm{R}}}) & \alert<1>{\uncover<1->{\text{otherwise}}}\\
\end{cases}\]



$\uncover<8->{\alert<8>{q_0:=(\text{Step 0}, \text{Nothing})}}$\\
$\uncover<9->{\alert<9>{q_1:=(\text{Step 1}, 0)}}$\\
$\uncover<10->{\alert<10>{q_1':=(\text{Step 1},1)}}$


\uncover<8->{\[\delta(\alert<8>{\uncover<8->{q_0}}, x) = \begin{cases}
(\alert<9>{\uncover<9->{q_1}},\sqcup , \mathrm{R}) & \text{if } x \text{= \alert<9>{\uncover<9->{0}}}\\
(\alert<10>{\uncover<10->{q_1'}},\sqcup , \mathrm{R}) & \text{if } x \text{= \alert<10>{\uncover<10->{1}}}\\
(Accept,x, \mathrm{R}) & \text{otherwise}\\
\end{cases}
\]}

\end{frame}

\begin{frame}[fragile]{Step 1}
\protect\hypertarget{step-1}{}

\[\delta((\text{Step 1}, n), x) = \begin{cases}
((\text{Step 2}, n) ,x, \mathrm{R}) & \text{if } x \text{=} \#\\
((\text{Step 1}, n),x, \mathrm{R}) & \text{otherwise}
\end{cases}
\]

\emph{``if the character under focus is a \texttt{\#}, then move right and go to Step 2; otherwise, repeat this step.''}

Here, \(n=0\) or 1, so if we denote \[q_2=(\text{Step 2},0)\]
\[q_2'=(\text{Step 2},1)\] then we can rewrite the above as

\[\delta(q_1, x) = \begin{cases}
(q_2 ,x, \mathrm{R}) & \text{if } x \text{=} \#\\
(q_1,x, \mathrm{R}) & \text{otherwise}\\
\end{cases}
\]

and

\[\delta(q_1', x) = \begin{cases}
(q_2' ,x, \mathrm{R}) & \text{if } x \text{=} \#\\
(q_1',x, \mathrm{R}) & \text{otherwise}\\
\end{cases}
\]

\end{frame}

\begin{frame}{Step 2}
\protect\hypertarget{step-2}{}

\[\delta((\text{Step 2},n), x) = \begin{cases}
((\text{Step 3},\text{Nothing}),n, \mathrm{R}) & \text{if } x \text{=} \sqcup\\
((\text{Step 2},n),x, \mathrm{R}) & \text{otherwise}\\
\end{cases}
\]

\emph{``If the character under focus is blank, erase the memory, move
right, and go to Step 3; otherwise, move right.''}

Denoting \[q_3:=(\text{Step 3},\text{Nothing})\] we get

\[\delta(q_2, x) = \begin{cases}
(q_3,0, \mathrm{R}) & \text{if } x \text{=} \sqcup\\
(q_2,x, \mathrm{R}) & \text{otherwise}\\
\end{cases}
\]

and

\[\delta(q_2', x) = \begin{cases}
(q_3,1, \mathrm{R}) & \text{if } x \text{=} \sqcup \\
(q_2',x, \mathrm{R}) & \text{otherwise}\\
\end{cases}
\]

\end{frame}

\begin{frame}[fragile]{Step 3}
\protect\hypertarget{step-3}{}

\[\delta((\text{Step 3},\text{Nothing}), x) = \begin{cases}
((\text{Step 4},\text{Nothing}),x, \mathrm{L}) & \text{if } x \text{=} \#\\
((\text{Step 3},\text{Nothing}),x, \mathrm{L}) & \text{otherwise}\\
\end{cases}
\]

\emph{``If the character under focus is \texttt{\#}, then move left and
go to Step 4; otherwise, move left and repeat this step.''}

Denote, \[q_3=(\text{Step 3},\text{Nothing})\]
\[q_4=(\text{Step 4},\text{Nothing})\] to get

\[\delta(q_3, x) = \begin{cases}
(q_4 ,x, \mathrm{L}) & \text{if } x \text{=} \#\\
(q_3,x, \mathrm{L}) & \text{otherwise}\\
\end{cases}
\]

\end{frame}

\begin{frame}{Step 4}
\protect\hypertarget{step-4}{}

\[\delta((\text{Step 4},\text{Nothing}), x) = \begin{cases}
((\text{Step 0}, \text{Nothing}),x, \mathrm{R}) & \text{if } x \text{=} \sqcup \\
((\text{Step 4},\text{Nothing}),x, \mathrm{L}) & \text{otherwise}\\
\end{cases}
\]

which, using the \(q_i\)'s already defined, is

\[\delta(q_4, x) = \begin{cases}
(q_0 ,x, \mathrm{R}) & \text{if } x \text{=} \sqcup \\
(q_4,x, \mathrm{L}) & \text{otherwise}\\
\end{cases}
\]

\end{frame}

\begin{frame}{The transition table}
\protect\hypertarget{the-transition-table}{}

Now it is very straightforward to translate all the partial definitions
of \(\delta\) above that use \(q_i\)'s rather than the explicit tuples,
into a transition table.

label: shift \textbar{} shift \textbar{} 0 \textbar{} 1 \textbar{}
\textbar{} \# \textbar{}
\textbar--------\textbar-------------\textbar-------------\textbar-------------\textbar-------------\textbar{}
\textbar{} \(q_0\) \textbar{} \(q_1\), , R \textbar{} \(q_1'\), , R
\textbar{} acc, ,R \textbar{} acc,\#,R \textbar{} \textbar{} \(q_1\)
\textbar{} \(q_1\),0, R \textbar{} \(q_1\),1, R \textbar{} \(q_1\), , R
\textbar{} \(q_2\),\#, R \textbar{} \textbar{} \(q_1'\) \textbar{}
\(q_1'\),0, R \textbar{} \(q_1'\),1, R \textbar{} \(q_1'\), , R
\textbar{} \(q_2'\),\#, R \textbar{} \textbar{} \(q_2\) \textbar{}
\(q_2\),0,R \textbar{} \(q_2\),1,R \textbar{} \(q_3\),0,L \textbar{}
\(q_2\),\#,R \textbar{} \textbar{} \(q_2'\) \textbar{} \(q_2'\),0,R
\textbar{} \(q_2'\),1,R \textbar{} \(q_3\),1,L \textbar{} \(q_2'\),\#,R
\textbar{} \textbar{} \(q_3\) \textbar{} \(q_3\),0,L \textbar{}
\(q_3\),1,L \textbar{} \(q_3\), ,L \textbar{} \(q_4\),\#,L \textbar{}
\textbar{} \(q_4\) \textbar{} \(q_4\),0,L \textbar{} \(q_4\),1,L
\textbar{} \(q_0\), ,R \textbar{} \(q_4\),\#,L \textbar{}

\end{frame}

\end{document}
