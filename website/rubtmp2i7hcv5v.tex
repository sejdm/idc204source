\documentclass[a4paper]{article}
\usepackage{amssymb}
\usepackage{amsthm}
\usepackage{amsmath}
\setcounter{secnumdepth}{0}
\newtheorem{theorem}{Theorem}
\title{Exercise sheet 4\\ {\small Theory of Computation, IDC204} }
\date{}
\begin{document}
\maketitle
\begin{enumerate}
\item For each language that you proved was regular in the previous exercise sets, find a regular expression to describe the language.
\item For each language that you proved was regular in the previous exercise sets, find a context free grammar to define it. Can any regular language be defined by context free grammar?
\item Find a context-free grammar to generate the following languages over $\Sigma:=\{0,1\}$
  \begin{enumerate}
    \item $\{0^n1^n \ |\ n=0,1,\ldots\}$
        \item The complement of the previous language, i.e. $\{0^m1^n \ |\  m>n,m,n=0,1,\ldots\}$
  \end{enumerate}

\item What language does the context free grammar $G:=(\{X\},\{(,)\},R,X)$, where $R=\{X\to (X) | XX | ϵ\}$ generate?

\item Find a context free grammar that generates the language consisting of precisely those strings that are polynomials in $x$ with integer coefficients.
\end{enumerate}
\end{document}