\documentclass[twocolumn,20pt,fleqn]{extarticle}
\usepackage{color}
\usepackage{geometry}
\usepackage[normalem]{ulem}
\usepackage{cancel}
\usepackage{dingbat}
\usepackage{marginnote}
\usepackage{amsmath}
\usepackage{amsthm}
\usepackage{amssymb}
\usepackage{amsfonts}
\usepackage{mathtools}
\usepackage{tcolorbox}
\usepackage{multicol}
\usepackage{tikz}
\usepackage{tikz-cd}
\usepackage{wrapfig}
\usepackage{caption}
%\usepackage{draftwatermark}

\usetikzlibrary{calc}

\setlength{\parindent}{0pt}
\setlength{\parskip}{0.5em}
\makeatletter
\renewcommand{\@seccntformat}[1]{}
\makeatother

\geometry{paperwidth=34cm, paperheight=19cm, margin=0.5cm}

\newcommand{\clr}[2]{{\color{#1} #2}}
\newcommand{\alert}[1]{{\color{red} #1}}
\newcommand{\ovrbrc}[2]{\overbrace{#2}^{#1}}
\newcommand{\undrbrc}[2]{\underbrace{#2}_{#1}}
\newcommand{\lnspc}{\vspace{1cm}}
\newcommand{\sep}{\vspace{0.5cm}}
\newcommand{\e}{\textrm{e}}
%\newcommand{\dydx}[2]{\frac{\mathrm{d}}}

\theoremstyle{plain}
\newtheorem*{theorem}{Theorem}
\newtheorem*{proposition}{Proposition}
\newtheorem*{lemma}{Lemma}
\newtheorem*{corollary}{Corollary}

\theoremstyle{definition}
\newtheorem*{exercise}{Exercise}
\newtheorem*{definition}{Definition}
\newtheorem*{example}{Example}
\newtheorem*{exmpls}{Examples}


\theoremstyle{remark}
\newtheorem*{remark}{Remark}
\newtheorem*{note}{Note}

\newenvironment*{examples}{\begin{exmpls} ~ \begin{enumerate}}{\end{enumerate}\end{exmpls}}

\begin{document}



\clearpage


\subsection{Reparametrization}
$\phi: (\alpha,\beta) \to (\alpha',\beta')$ is bijective,


\clearpage


\subsection{Reparametrization}
$\phi: (\alpha,\beta) \to (\alpha',\beta')$ is bijective, then it is invertible,


\clearpage


\subsection{Reparametrization}
$\phi: (\alpha,\beta) \to (\alpha',\beta')$ is bijective, then it is invertible, its inverse is,
denoted: $\phi^{-1} : (\alpha',\beta') \to (\alpha,\beta)$.




\clearpage


\subsection{Reparametrization}
$\phi: (\alpha,\beta) \to (\alpha',\beta')$ is bijective, then it is invertible, its inverse is,
denoted: $\phi^{-1} : (\alpha',\beta') \to (\alpha,\beta)$.

If $\phi$ is smooth, is $\phi^{-1}$ smooth?


\clearpage


\subsection{Reparametrization}
$\phi: (\alpha,\beta) \to (\alpha',\beta')$ is bijective, then it is invertible, its inverse is,
denoted: $\phi^{-1} : (\alpha',\beta') \to (\alpha,\beta)$.

If $\phi$ is smooth, is $\phi^{-1}$ smooth? Not necessarily!






\clearpage


\subsection{Reparametrization}
$\phi: (\alpha,\beta) \to (\alpha',\beta')$ is bijective, then it is invertible, its inverse is,
denoted: $\phi^{-1} : (\alpha',\beta') \to (\alpha,\beta)$.

If $\phi$ is smooth, is $\phi^{-1}$ smooth? Not necessarily!





\newpage

\begin{tikzpicture}
\draw[thin,->] (-5,0) -- (5,0) node[anchor=west] {$x$};
\foreach \x in {-4,-3,-2,-1,0,1,2,3,4}
\draw (\x cm,2pt) -- (\x cm,-2pt);
\end{tikzpicture}


\clearpage


\subsection{Reparametrization}
$\phi: (\alpha,\beta) \to (\alpha',\beta')$ is bijective, then it is invertible, its inverse is,
denoted: $\phi^{-1} : (\alpha',\beta') \to (\alpha,\beta)$.

If $\phi$ is smooth, is $\phi^{-1}$ smooth? Not necessarily!





\newpage

\begin{tikzpicture}
\draw[thin,->] (-5,0) -- (5,0) node[anchor=west] {$x$};
\foreach \x in {-4,-3,-2,-1,0,1,2,3,4}
\draw (\x cm,2pt) -- (\x cm,-2pt);
\foreach \x in {-4,-3,-2,-1,1,2,3,4}
\node[anchor=north] at (\x cm,0) {\footnotesize \x};
\draw[thin,->] (0,-5) -- (0,5) node[anchor=south] {$y$};
\foreach \y in {-4,-3,-2,-1,0,1,2,3,4}
\draw (2pt, \y cm) -- (-2pt, \y cm);
\end{tikzpicture}


\clearpage


\subsection{Reparametrization}
$\phi: (\alpha,\beta) \to (\alpha',\beta')$ is bijective, then it is invertible, its inverse is,
denoted: $\phi^{-1} : (\alpha',\beta') \to (\alpha,\beta)$.

If $\phi$ is smooth, is $\phi^{-1}$ smooth? Not necessarily!





\newpage

\begin{tikzpicture}
\draw[thin,->] (-5,0) -- (5,0) node[anchor=west] {$x$};
\foreach \x in {-4,-3,-2,-1,0,1,2,3,4}
\draw (\x cm,2pt) -- (\x cm,-2pt);
\foreach \x in {-4,-3,-2,-1,1,2,3,4}
\node[anchor=north] at (\x cm,0) {\footnotesize \x};
\draw[thin,->] (0,-5) -- (0,5) node[anchor=south] {$y$};
\foreach \y in {-4,-3,-2,-1,0,1,2,3,4}
\draw (2pt, \y cm) -- (-2pt, \y cm);
\foreach \y in {-4,-3,-2,-1,1,2,3,4}
\node[anchor=west] at (0, \y cm) {\footnotesize \y};
\end{tikzpicture}


\clearpage


\subsection{Reparametrization}
$\phi: (\alpha,\beta) \to (\alpha',\beta')$ is bijective, then it is invertible, its inverse is,
denoted: $\phi^{-1} : (\alpha',\beta') \to (\alpha,\beta)$.

If $\phi$ is smooth, is $\phi^{-1}$ smooth? Not necessarily!





\newpage

\begin{tikzpicture}
\draw[thin,->] (-5,0) -- (5,0) node[anchor=west] {$x$};
\foreach \x in {-4,-3,-2,-1,0,1,2,3,4}
\draw (\x cm,2pt) -- (\x cm,-2pt);
\foreach \x in {-4,-3,-2,-1,1,2,3,4}
\node[anchor=north] at (\x cm,0) {\footnotesize \x};
\draw[thin,->] (0,-5) -- (0,5) node[anchor=south] {$y$};
\foreach \y in {-4,-3,-2,-1,0,1,2,3,4}
\draw (2pt, \y cm) -- (-2pt, \y cm);
\foreach \y in {-4,-3,-2,-1,1,2,3,4}
\node[anchor=west] at (0, \y cm) {\footnotesize \y};
\node[anchor=north west] at (0,0) {\footnotesize O};
;
\end{tikzpicture}
 





\clearpage


\subsection{Reparametrization}
$\phi: (\alpha,\beta) \to (\alpha',\beta')$ is bijective, then it is invertible, its inverse is,
denoted: $\phi^{-1} : (\alpha',\beta') \to (\alpha,\beta)$.

If $\phi$ is smooth, is $\phi^{-1}$ smooth? Not necessarily!





\newpage

\begin{tikzpicture}
\draw[thin,->] (-5,0) -- (5,0) node[anchor=west] {$x$};
\foreach \x in {-4,-3,-2,-1,0,1,2,3,4}
\draw (\x cm,2pt) -- (\x cm,-2pt);
\foreach \x in {-4,-3,-2,-1,1,2,3,4}
\node[anchor=north] at (\x cm,0) {\footnotesize \x};
\draw[thin,->] (0,-5) -- (0,5) node[anchor=south] {$y$};
\foreach \y in {-4,-3,-2,-1,0,1,2,3,4}
\draw (2pt, \y cm) -- (-2pt, \y cm);
\foreach \y in {-4,-3,-2,-1,1,2,3,4}
\node[anchor=west] at (0, \y cm) {\footnotesize \y};
\node[anchor=north west] at (0,0) {\footnotesize O};

\draw[color=blue,thick] (2,0) arc
    [
        start angle=0,
        end angle=290,
        x radius=2cm,
        y radius =2cm
    ] ;

\draw[thin,->] (-5,-7) -- (5,-7) node[anchor=west] {};
;
\end{tikzpicture}
 






\clearpage


\subsection{Reparametrization}
$\phi: (\alpha,\beta) \to (\alpha',\beta')$ is bijective, then it is invertible, its inverse is,
denoted: $\phi^{-1} : (\alpha',\beta') \to (\alpha,\beta)$.

If $\phi$ is smooth, is $\phi^{-1}$ smooth? Not necessarily!





\newpage

\begin{tikzpicture}
\draw[thin,->] (-5,0) -- (5,0) node[anchor=west] {$x$};
\foreach \x in {-4,-3,-2,-1,0,1,2,3,4}
\draw (\x cm,2pt) -- (\x cm,-2pt);
\foreach \x in {-4,-3,-2,-1,1,2,3,4}
\node[anchor=north] at (\x cm,0) {\footnotesize \x};
\draw[thin,->] (0,-5) -- (0,5) node[anchor=south] {$y$};
\foreach \y in {-4,-3,-2,-1,0,1,2,3,4}
\draw (2pt, \y cm) -- (-2pt, \y cm);
\foreach \y in {-4,-3,-2,-1,1,2,3,4}
\node[anchor=west] at (0, \y cm) {\footnotesize \y};
\node[anchor=north west] at (0,0) {\footnotesize O};

\draw[color=blue,thick] (2,0) arc
    [
        start angle=0,
        end angle=290,
        x radius=2cm,
        y radius =2cm
    ] ;
 \node at (2,0)[circle,fill,inner sep=2pt,color=blue]{}; 
\draw[thin,->] (-5,-7) -- (5,-7) node[anchor=west] {};
\node at (-4,-7)[circle,fill,inner sep=2pt,color=blue]{};
;
\end{tikzpicture}
 






\clearpage


\subsection{Reparametrization}
$\phi: (\alpha,\beta) \to (\alpha',\beta')$ is bijective, then it is invertible, its inverse is,
denoted: $\phi^{-1} : (\alpha',\beta') \to (\alpha,\beta)$.

If $\phi$ is smooth, is $\phi^{-1}$ smooth? Not necessarily!





\newpage

\begin{tikzpicture}
\draw[thin,->] (-5,0) -- (5,0) node[anchor=west] {$x$};
\foreach \x in {-4,-3,-2,-1,0,1,2,3,4}
\draw (\x cm,2pt) -- (\x cm,-2pt);
\foreach \x in {-4,-3,-2,-1,1,2,3,4}
\node[anchor=north] at (\x cm,0) {\footnotesize \x};
\draw[thin,->] (0,-5) -- (0,5) node[anchor=south] {$y$};
\foreach \y in {-4,-3,-2,-1,0,1,2,3,4}
\draw (2pt, \y cm) -- (-2pt, \y cm);
\foreach \y in {-4,-3,-2,-1,1,2,3,4}
\node[anchor=west] at (0, \y cm) {\footnotesize \y};
\node[anchor=north west] at (0,0) {\footnotesize O};

\draw[color=blue,thick] (2,0) arc
    [
        start angle=0,
        end angle=290,
        x radius=2cm,
        y radius =2cm
    ] ;
 \node at (2,0)[circle,fill,inner sep=2pt,color=blue]{};  \node at (0,2)[circle,fill,inner sep=2pt,color=green]{}; 
\draw[thin,->] (-5,-7) -- (5,-7) node[anchor=west] {};
\node at (-4,-7)[circle,fill,inner sep=2pt,color=blue]{};
\node at (-2,-7)[circle,fill,inner sep=2pt,color=green]{}; 
;
\end{tikzpicture}
 






\clearpage


\subsection{Reparametrization}
$\phi: (\alpha,\beta) \to (\alpha',\beta')$ is bijective, then it is invertible, its inverse is,
denoted: $\phi^{-1} : (\alpha',\beta') \to (\alpha,\beta)$.

If $\phi$ is smooth, is $\phi^{-1}$ smooth? Not necessarily!





\newpage

\begin{tikzpicture}
\draw[thin,->] (-5,0) -- (5,0) node[anchor=west] {$x$};
\foreach \x in {-4,-3,-2,-1,0,1,2,3,4}
\draw (\x cm,2pt) -- (\x cm,-2pt);
\foreach \x in {-4,-3,-2,-1,1,2,3,4}
\node[anchor=north] at (\x cm,0) {\footnotesize \x};
\draw[thin,->] (0,-5) -- (0,5) node[anchor=south] {$y$};
\foreach \y in {-4,-3,-2,-1,0,1,2,3,4}
\draw (2pt, \y cm) -- (-2pt, \y cm);
\foreach \y in {-4,-3,-2,-1,1,2,3,4}
\node[anchor=west] at (0, \y cm) {\footnotesize \y};
\node[anchor=north west] at (0,0) {\footnotesize O};

\draw[color=blue,thick] (2,0) arc
    [
        start angle=0,
        end angle=290,
        x radius=2cm,
        y radius =2cm
    ] ;
 \node at (2,0)[circle,fill,inner sep=2pt,color=blue]{};  \node at (0,2)[circle,fill,inner sep=2pt,color=green]{};  \node at (-2,0)[circle,fill,inner sep=2pt,color=yellow]{}; 
\draw[thin,->] (-5,-7) -- (5,-7) node[anchor=west] {};
\node at (-4,-7)[circle,fill,inner sep=2pt,color=blue]{};
\node at (-2,-7)[circle,fill,inner sep=2pt,color=green]{}; 
\node at (0,-7)[circle,fill,inner sep=2pt,color=yellow]{};
;
\end{tikzpicture}
 






\clearpage


\subsection{Reparametrization}
$\phi: (\alpha,\beta) \to (\alpha',\beta')$ is bijective, then it is invertible, its inverse is,
denoted: $\phi^{-1} : (\alpha',\beta') \to (\alpha,\beta)$.

If $\phi$ is smooth, is $\phi^{-1}$ smooth? Not necessarily!





\newpage

\begin{tikzpicture}
\draw[thin,->] (-5,0) -- (5,0) node[anchor=west] {$x$};
\foreach \x in {-4,-3,-2,-1,0,1,2,3,4}
\draw (\x cm,2pt) -- (\x cm,-2pt);
\foreach \x in {-4,-3,-2,-1,1,2,3,4}
\node[anchor=north] at (\x cm,0) {\footnotesize \x};
\draw[thin,->] (0,-5) -- (0,5) node[anchor=south] {$y$};
\foreach \y in {-4,-3,-2,-1,0,1,2,3,4}
\draw (2pt, \y cm) -- (-2pt, \y cm);
\foreach \y in {-4,-3,-2,-1,1,2,3,4}
\node[anchor=west] at (0, \y cm) {\footnotesize \y};
\node[anchor=north west] at (0,0) {\footnotesize O};

\draw[color=blue,thick] (2,0) arc
    [
        start angle=0,
        end angle=290,
        x radius=2cm,
        y radius =2cm
    ] ;
 \node at (2,0)[circle,fill,inner sep=2pt,color=blue]{};  \node at (0,2)[circle,fill,inner sep=2pt,color=green]{};  \node at (-2,0)[circle,fill,inner sep=2pt,color=yellow]{};  \node at (0,-2)[circle,fill,inner sep=2pt,color=red]{}; 
\draw[thin,->] (-5,-7) -- (5,-7) node[anchor=west] {};
\node at (-4,-7)[circle,fill,inner sep=2pt,color=blue]{};
\node at (-2,-7)[circle,fill,inner sep=2pt,color=green]{}; 
\node at (0,-7)[circle,fill,inner sep=2pt,color=yellow]{};
\node at (2,-7)[circle,fill,inner sep=2pt,color=red]{};

;
\end{tikzpicture}
 






\clearpage


\subsection{Reparametrization}
$\phi: (\alpha,\beta) \to (\alpha',\beta')$ is bijective, then it is invertible, its inverse is,
denoted: $\phi^{-1} : (\alpha',\beta') \to (\alpha,\beta)$.

If $\phi$ is smooth, is $\phi^{-1}$ smooth? Not necessarily!





\newpage

\begin{tikzpicture}
\draw[thin,->] (-5,0) -- (5,0) node[anchor=west] {$x$};
\foreach \x in {-4,-3,-2,-1,0,1,2,3,4}
\draw (\x cm,2pt) -- (\x cm,-2pt);
\foreach \x in {-4,-3,-2,-1,1,2,3,4}
\node[anchor=north] at (\x cm,0) {\footnotesize \x};
\draw[thin,->] (0,-5) -- (0,5) node[anchor=south] {$y$};
\foreach \y in {-4,-3,-2,-1,0,1,2,3,4}
\draw (2pt, \y cm) -- (-2pt, \y cm);
\foreach \y in {-4,-3,-2,-1,1,2,3,4}
\node[anchor=west] at (0, \y cm) {\footnotesize \y};
\node[anchor=north west] at (0,0) {\footnotesize O};

\draw[color=blue,thick] (2,0) arc
    [
        start angle=0,
        end angle=290,
        x radius=2cm,
        y radius =2cm
    ] ;
 \node at (2,0)[circle,fill,inner sep=2pt,color=blue]{};  \node at (0,2)[circle,fill,inner sep=2pt,color=green]{};  \node at (-2,0)[circle,fill,inner sep=2pt,color=yellow]{};  \node at (0,-2)[circle,fill,inner sep=2pt,color=red]{}; 
\draw[thin,->] (-5,-7) -- (5,-7) node[anchor=west] {};
\node at (-4,-7)[circle,fill,inner sep=2pt,color=blue]{};
\node at (-2,-7)[circle,fill,inner sep=2pt,color=green]{}; 
\node at (0,-7)[circle,fill,inner sep=2pt,color=yellow]{};
\node at (2,-7)[circle,fill,inner sep=2pt,color=red]{};

\draw[thin,->] (-5,-10) -- (5,-10) node[anchor=west] {};
;
\end{tikzpicture}
 






\clearpage


\subsection{Reparametrization}
$\phi: (\alpha,\beta) \to (\alpha',\beta')$ is bijective, then it is invertible, its inverse is,
denoted: $\phi^{-1} : (\alpha',\beta') \to (\alpha,\beta)$.

If $\phi$ is smooth, is $\phi^{-1}$ smooth? Not necessarily!





\newpage

\begin{tikzpicture}
\draw[thin,->] (-5,0) -- (5,0) node[anchor=west] {$x$};
\foreach \x in {-4,-3,-2,-1,0,1,2,3,4}
\draw (\x cm,2pt) -- (\x cm,-2pt);
\foreach \x in {-4,-3,-2,-1,1,2,3,4}
\node[anchor=north] at (\x cm,0) {\footnotesize \x};
\draw[thin,->] (0,-5) -- (0,5) node[anchor=south] {$y$};
\foreach \y in {-4,-3,-2,-1,0,1,2,3,4}
\draw (2pt, \y cm) -- (-2pt, \y cm);
\foreach \y in {-4,-3,-2,-1,1,2,3,4}
\node[anchor=west] at (0, \y cm) {\footnotesize \y};
\node[anchor=north west] at (0,0) {\footnotesize O};

\draw[color=blue,thick] (2,0) arc
    [
        start angle=0,
        end angle=290,
        x radius=2cm,
        y radius =2cm
    ] ;
 \node at (2,0)[circle,fill,inner sep=2pt,color=blue]{};  \node at (0,2)[circle,fill,inner sep=2pt,color=green]{};  \node at (-2,0)[circle,fill,inner sep=2pt,color=yellow]{};  \node at (0,-2)[circle,fill,inner sep=2pt,color=red]{};  \node at (2,0)[circle,fill,inner sep=2pt,color=blue]{}; 
\draw[thin,->] (-5,-7) -- (5,-7) node[anchor=west] {};
\node at (-4,-7)[circle,fill,inner sep=2pt,color=blue]{};
\node at (-2,-7)[circle,fill,inner sep=2pt,color=green]{}; 
\node at (0,-7)[circle,fill,inner sep=2pt,color=yellow]{};
\node at (2,-7)[circle,fill,inner sep=2pt,color=red]{};

\draw[thin,->] (-5,-10) -- (5,-10) node[anchor=west] {};
\node at (-5,-10)[circle,fill,inner sep=2pt,color=blue]{};
;
\end{tikzpicture}
 






\clearpage


\subsection{Reparametrization}
$\phi: (\alpha,\beta) \to (\alpha',\beta')$ is bijective, then it is invertible, its inverse is,
denoted: $\phi^{-1} : (\alpha',\beta') \to (\alpha,\beta)$.

If $\phi$ is smooth, is $\phi^{-1}$ smooth? Not necessarily!





\newpage

\begin{tikzpicture}
\draw[thin,->] (-5,0) -- (5,0) node[anchor=west] {$x$};
\foreach \x in {-4,-3,-2,-1,0,1,2,3,4}
\draw (\x cm,2pt) -- (\x cm,-2pt);
\foreach \x in {-4,-3,-2,-1,1,2,3,4}
\node[anchor=north] at (\x cm,0) {\footnotesize \x};
\draw[thin,->] (0,-5) -- (0,5) node[anchor=south] {$y$};
\foreach \y in {-4,-3,-2,-1,0,1,2,3,4}
\draw (2pt, \y cm) -- (-2pt, \y cm);
\foreach \y in {-4,-3,-2,-1,1,2,3,4}
\node[anchor=west] at (0, \y cm) {\footnotesize \y};
\node[anchor=north west] at (0,0) {\footnotesize O};

\draw[color=blue,thick] (2,0) arc
    [
        start angle=0,
        end angle=290,
        x radius=2cm,
        y radius =2cm
    ] ;
 \node at (2,0)[circle,fill,inner sep=2pt,color=blue]{};  \node at (0,2)[circle,fill,inner sep=2pt,color=green]{};  \node at (-2,0)[circle,fill,inner sep=2pt,color=yellow]{};  \node at (0,-2)[circle,fill,inner sep=2pt,color=red]{};  \node at (2,0)[circle,fill,inner sep=2pt,color=blue]{};  \node at (0,2)[circle,fill,inner sep=2pt,color=green]{}; 
\draw[thin,->] (-5,-7) -- (5,-7) node[anchor=west] {};
\node at (-4,-7)[circle,fill,inner sep=2pt,color=blue]{};
\node at (-2,-7)[circle,fill,inner sep=2pt,color=green]{}; 
\node at (0,-7)[circle,fill,inner sep=2pt,color=yellow]{};
\node at (2,-7)[circle,fill,inner sep=2pt,color=red]{};

\draw[thin,->] (-5,-10) -- (5,-10) node[anchor=west] {};
\node at (-5,-10)[circle,fill,inner sep=2pt,color=blue]{};
\node at (-2,-10)[circle,fill,inner sep=2pt,color=green]{}; 
;
\end{tikzpicture}
 






\clearpage


\subsection{Reparametrization}
$\phi: (\alpha,\beta) \to (\alpha',\beta')$ is bijective, then it is invertible, its inverse is,
denoted: $\phi^{-1} : (\alpha',\beta') \to (\alpha,\beta)$.

If $\phi$ is smooth, is $\phi^{-1}$ smooth? Not necessarily!





\newpage

\begin{tikzpicture}
\draw[thin,->] (-5,0) -- (5,0) node[anchor=west] {$x$};
\foreach \x in {-4,-3,-2,-1,0,1,2,3,4}
\draw (\x cm,2pt) -- (\x cm,-2pt);
\foreach \x in {-4,-3,-2,-1,1,2,3,4}
\node[anchor=north] at (\x cm,0) {\footnotesize \x};
\draw[thin,->] (0,-5) -- (0,5) node[anchor=south] {$y$};
\foreach \y in {-4,-3,-2,-1,0,1,2,3,4}
\draw (2pt, \y cm) -- (-2pt, \y cm);
\foreach \y in {-4,-3,-2,-1,1,2,3,4}
\node[anchor=west] at (0, \y cm) {\footnotesize \y};
\node[anchor=north west] at (0,0) {\footnotesize O};

\draw[color=blue,thick] (2,0) arc
    [
        start angle=0,
        end angle=290,
        x radius=2cm,
        y radius =2cm
    ] ;
 \node at (2,0)[circle,fill,inner sep=2pt,color=blue]{};  \node at (0,2)[circle,fill,inner sep=2pt,color=green]{};  \node at (-2,0)[circle,fill,inner sep=2pt,color=yellow]{};  \node at (0,-2)[circle,fill,inner sep=2pt,color=red]{};  \node at (2,0)[circle,fill,inner sep=2pt,color=blue]{};  \node at (0,2)[circle,fill,inner sep=2pt,color=green]{};  \node at (-2,0)[circle,fill,inner sep=2pt,color=yellow]{}; 
\draw[thin,->] (-5,-7) -- (5,-7) node[anchor=west] {};
\node at (-4,-7)[circle,fill,inner sep=2pt,color=blue]{};
\node at (-2,-7)[circle,fill,inner sep=2pt,color=green]{}; 
\node at (0,-7)[circle,fill,inner sep=2pt,color=yellow]{};
\node at (2,-7)[circle,fill,inner sep=2pt,color=red]{};

\draw[thin,->] (-5,-10) -- (5,-10) node[anchor=west] {};
\node at (-5,-10)[circle,fill,inner sep=2pt,color=blue]{};
\node at (-2,-10)[circle,fill,inner sep=2pt,color=green]{}; 
\node at (0,-10)[circle,fill,inner sep=2pt,color=yellow]{};
;
\end{tikzpicture}
 






\clearpage


\subsection{Reparametrization}
$\phi: (\alpha,\beta) \to (\alpha',\beta')$ is bijective, then it is invertible, its inverse is,
denoted: $\phi^{-1} : (\alpha',\beta') \to (\alpha,\beta)$.

If $\phi$ is smooth, is $\phi^{-1}$ smooth? Not necessarily!





\newpage

\begin{tikzpicture}
\draw[thin,->] (-5,0) -- (5,0) node[anchor=west] {$x$};
\foreach \x in {-4,-3,-2,-1,0,1,2,3,4}
\draw (\x cm,2pt) -- (\x cm,-2pt);
\foreach \x in {-4,-3,-2,-1,1,2,3,4}
\node[anchor=north] at (\x cm,0) {\footnotesize \x};
\draw[thin,->] (0,-5) -- (0,5) node[anchor=south] {$y$};
\foreach \y in {-4,-3,-2,-1,0,1,2,3,4}
\draw (2pt, \y cm) -- (-2pt, \y cm);
\foreach \y in {-4,-3,-2,-1,1,2,3,4}
\node[anchor=west] at (0, \y cm) {\footnotesize \y};
\node[anchor=north west] at (0,0) {\footnotesize O};

\draw[color=blue,thick] (2,0) arc
    [
        start angle=0,
        end angle=290,
        x radius=2cm,
        y radius =2cm
    ] ;
 \node at (2,0)[circle,fill,inner sep=2pt,color=blue]{};  \node at (0,2)[circle,fill,inner sep=2pt,color=green]{};  \node at (-2,0)[circle,fill,inner sep=2pt,color=yellow]{};  \node at (0,-2)[circle,fill,inner sep=2pt,color=red]{};  \node at (2,0)[circle,fill,inner sep=2pt,color=blue]{};  \node at (0,2)[circle,fill,inner sep=2pt,color=green]{};  \node at (-2,0)[circle,fill,inner sep=2pt,color=yellow]{};  \node at (0,-2)[circle,fill,inner sep=2pt,color=red]{}; 

\draw[thin,->] (-5,-7) -- (5,-7) node[anchor=west] {};
\node at (-4,-7)[circle,fill,inner sep=2pt,color=blue]{};
\node at (-2,-7)[circle,fill,inner sep=2pt,color=green]{}; 
\node at (0,-7)[circle,fill,inner sep=2pt,color=yellow]{};
\node at (2,-7)[circle,fill,inner sep=2pt,color=red]{};

\draw[thin,->] (-5,-10) -- (5,-10) node[anchor=west] {};
\node at (-5,-10)[circle,fill,inner sep=2pt,color=blue]{};
\node at (-2,-10)[circle,fill,inner sep=2pt,color=green]{}; 
\node at (0,-10)[circle,fill,inner sep=2pt,color=yellow]{};
\node at (4,-10)[circle,fill,inner sep=2pt,color=red]{};;
\end{tikzpicture}
 






\clearpage


\subsection{Reparametrization}
$\phi: (\alpha,\beta) \to (\alpha',\beta')$ is bijective, then it is invertible, its inverse is,
denoted: $\phi^{-1} : (\alpha',\beta') \to (\alpha,\beta)$.

If $\phi$ is smooth, is $\phi^{-1}$ smooth? Not necessarily!





\newpage

\begin{tikzpicture}
\draw[thin,->] (-5,0) -- (5,0) node[anchor=west] {$x$};
\foreach \x in {-4,-3,-2,-1,0,1,2,3,4}
\draw (\x cm,2pt) -- (\x cm,-2pt);
\foreach \x in {-4,-3,-2,-1,1,2,3,4}
\node[anchor=north] at (\x cm,0) {\footnotesize \x};
\draw[thin,->] (0,-5) -- (0,5) node[anchor=south] {$y$};
\foreach \y in {-4,-3,-2,-1,0,1,2,3,4}
\draw (2pt, \y cm) -- (-2pt, \y cm);
\foreach \y in {-4,-3,-2,-1,1,2,3,4}
\node[anchor=west] at (0, \y cm) {\footnotesize \y};
\node[anchor=north west] at (0,0) {\footnotesize O};

\draw[color=blue,thick] (2,0) arc
    [
        start angle=0,
        end angle=290,
        x radius=2cm,
        y radius =2cm
    ] ;
 \node at (2,0)[circle,fill,inner sep=2pt,color=blue]{};  \node at (0,2)[circle,fill,inner sep=2pt,color=green]{};  \node at (-2,0)[circle,fill,inner sep=2pt,color=yellow]{};  \node at (0,-2)[circle,fill,inner sep=2pt,color=red]{};  \node at (2,0)[circle,fill,inner sep=2pt,color=blue]{};  \node at (0,2)[circle,fill,inner sep=2pt,color=green]{};  \node at (-2,0)[circle,fill,inner sep=2pt,color=yellow]{};  \node at (0,-2)[circle,fill,inner sep=2pt,color=red]{}; 
\draw[thick,->, shorten >=2.5pt] (-5,-10) -- (-4,-7);

\draw[thin,->] (-5,-7) -- (5,-7) node[anchor=west] {};
\node at (-4,-7)[circle,fill,inner sep=2pt,color=blue]{};
\node at (-2,-7)[circle,fill,inner sep=2pt,color=green]{}; 
\node at (0,-7)[circle,fill,inner sep=2pt,color=yellow]{};
\node at (2,-7)[circle,fill,inner sep=2pt,color=red]{};

\draw[thin,->] (-5,-10) -- (5,-10) node[anchor=west] {};
\node at (-5,-10)[circle,fill,inner sep=2pt,color=blue]{};
\node at (-2,-10)[circle,fill,inner sep=2pt,color=green]{}; 
\node at (0,-10)[circle,fill,inner sep=2pt,color=yellow]{};
\node at (4,-10)[circle,fill,inner sep=2pt,color=red]{};;
\end{tikzpicture}
 






\clearpage


\subsection{Reparametrization}
$\phi: (\alpha,\beta) \to (\alpha',\beta')$ is bijective, then it is invertible, its inverse is,
denoted: $\phi^{-1} : (\alpha',\beta') \to (\alpha,\beta)$.

If $\phi$ is smooth, is $\phi^{-1}$ smooth? Not necessarily!





\newpage

\begin{tikzpicture}
\draw[thin,->] (-5,0) -- (5,0) node[anchor=west] {$x$};
\foreach \x in {-4,-3,-2,-1,0,1,2,3,4}
\draw (\x cm,2pt) -- (\x cm,-2pt);
\foreach \x in {-4,-3,-2,-1,1,2,3,4}
\node[anchor=north] at (\x cm,0) {\footnotesize \x};
\draw[thin,->] (0,-5) -- (0,5) node[anchor=south] {$y$};
\foreach \y in {-4,-3,-2,-1,0,1,2,3,4}
\draw (2pt, \y cm) -- (-2pt, \y cm);
\foreach \y in {-4,-3,-2,-1,1,2,3,4}
\node[anchor=west] at (0, \y cm) {\footnotesize \y};
\node[anchor=north west] at (0,0) {\footnotesize O};

\draw[color=blue,thick] (2,0) arc
    [
        start angle=0,
        end angle=290,
        x radius=2cm,
        y radius =2cm
    ] ;
 \node at (2,0)[circle,fill,inner sep=2pt,color=blue]{};  \node at (0,2)[circle,fill,inner sep=2pt,color=green]{};  \node at (-2,0)[circle,fill,inner sep=2pt,color=yellow]{};  \node at (0,-2)[circle,fill,inner sep=2pt,color=red]{};  \node at (2,0)[circle,fill,inner sep=2pt,color=blue]{};  \node at (0,2)[circle,fill,inner sep=2pt,color=green]{};  \node at (-2,0)[circle,fill,inner sep=2pt,color=yellow]{};  \node at (0,-2)[circle,fill,inner sep=2pt,color=red]{}; 
\draw[thick,->, shorten >=2.5pt] (-5,-10) -- (-4,-7);
\draw[thick,->, shorten >=2.5pt] (-2,-10) -- (-2,-7);

\draw[thin,->] (-5,-7) -- (5,-7) node[anchor=west] {};
\node at (-4,-7)[circle,fill,inner sep=2pt,color=blue]{};
\node at (-2,-7)[circle,fill,inner sep=2pt,color=green]{}; 
\node at (0,-7)[circle,fill,inner sep=2pt,color=yellow]{};
\node at (2,-7)[circle,fill,inner sep=2pt,color=red]{};

\draw[thin,->] (-5,-10) -- (5,-10) node[anchor=west] {};
\node at (-5,-10)[circle,fill,inner sep=2pt,color=blue]{};
\node at (-2,-10)[circle,fill,inner sep=2pt,color=green]{}; 
\node at (0,-10)[circle,fill,inner sep=2pt,color=yellow]{};
\node at (4,-10)[circle,fill,inner sep=2pt,color=red]{};;
\end{tikzpicture}
 






\clearpage


\subsection{Reparametrization}
$\phi: (\alpha,\beta) \to (\alpha',\beta')$ is bijective, then it is invertible, its inverse is,
denoted: $\phi^{-1} : (\alpha',\beta') \to (\alpha,\beta)$.

If $\phi$ is smooth, is $\phi^{-1}$ smooth? Not necessarily!





\newpage

\begin{tikzpicture}
\draw[thin,->] (-5,0) -- (5,0) node[anchor=west] {$x$};
\foreach \x in {-4,-3,-2,-1,0,1,2,3,4}
\draw (\x cm,2pt) -- (\x cm,-2pt);
\foreach \x in {-4,-3,-2,-1,1,2,3,4}
\node[anchor=north] at (\x cm,0) {\footnotesize \x};
\draw[thin,->] (0,-5) -- (0,5) node[anchor=south] {$y$};
\foreach \y in {-4,-3,-2,-1,0,1,2,3,4}
\draw (2pt, \y cm) -- (-2pt, \y cm);
\foreach \y in {-4,-3,-2,-1,1,2,3,4}
\node[anchor=west] at (0, \y cm) {\footnotesize \y};
\node[anchor=north west] at (0,0) {\footnotesize O};

\draw[color=blue,thick] (2,0) arc
    [
        start angle=0,
        end angle=290,
        x radius=2cm,
        y radius =2cm
    ] ;
 \node at (2,0)[circle,fill,inner sep=2pt,color=blue]{};  \node at (0,2)[circle,fill,inner sep=2pt,color=green]{};  \node at (-2,0)[circle,fill,inner sep=2pt,color=yellow]{};  \node at (0,-2)[circle,fill,inner sep=2pt,color=red]{};  \node at (2,0)[circle,fill,inner sep=2pt,color=blue]{};  \node at (0,2)[circle,fill,inner sep=2pt,color=green]{};  \node at (-2,0)[circle,fill,inner sep=2pt,color=yellow]{};  \node at (0,-2)[circle,fill,inner sep=2pt,color=red]{}; 
\draw[thick,->, shorten >=2.5pt] (-5,-10) -- (-4,-7);
\draw[thick,->, shorten >=2.5pt] (-2,-10) -- (-2,-7);
\draw[thick,->, shorten >=2.5pt] (0,-10) -- (0,-7);

\draw[thin,->] (-5,-7) -- (5,-7) node[anchor=west] {};
\node at (-4,-7)[circle,fill,inner sep=2pt,color=blue]{};
\node at (-2,-7)[circle,fill,inner sep=2pt,color=green]{}; 
\node at (0,-7)[circle,fill,inner sep=2pt,color=yellow]{};
\node at (2,-7)[circle,fill,inner sep=2pt,color=red]{};

\draw[thin,->] (-5,-10) -- (5,-10) node[anchor=west] {};
\node at (-5,-10)[circle,fill,inner sep=2pt,color=blue]{};
\node at (-2,-10)[circle,fill,inner sep=2pt,color=green]{}; 
\node at (0,-10)[circle,fill,inner sep=2pt,color=yellow]{};
\node at (4,-10)[circle,fill,inner sep=2pt,color=red]{};;
\end{tikzpicture}
 






\clearpage


\subsection{Reparametrization}
$\phi: (\alpha,\beta) \to (\alpha',\beta')$ is bijective, then it is invertible, its inverse is,
denoted: $\phi^{-1} : (\alpha',\beta') \to (\alpha,\beta)$.

If $\phi$ is smooth, is $\phi^{-1}$ smooth? Not necessarily!





\newpage

\begin{tikzpicture}
\draw[thin,->] (-5,0) -- (5,0) node[anchor=west] {$x$};
\foreach \x in {-4,-3,-2,-1,0,1,2,3,4}
\draw (\x cm,2pt) -- (\x cm,-2pt);
\foreach \x in {-4,-3,-2,-1,1,2,3,4}
\node[anchor=north] at (\x cm,0) {\footnotesize \x};
\draw[thin,->] (0,-5) -- (0,5) node[anchor=south] {$y$};
\foreach \y in {-4,-3,-2,-1,0,1,2,3,4}
\draw (2pt, \y cm) -- (-2pt, \y cm);
\foreach \y in {-4,-3,-2,-1,1,2,3,4}
\node[anchor=west] at (0, \y cm) {\footnotesize \y};
\node[anchor=north west] at (0,0) {\footnotesize O};

\draw[color=blue,thick] (2,0) arc
    [
        start angle=0,
        end angle=290,
        x radius=2cm,
        y radius =2cm
    ] ;
 \node at (2,0)[circle,fill,inner sep=2pt,color=blue]{};  \node at (0,2)[circle,fill,inner sep=2pt,color=green]{};  \node at (-2,0)[circle,fill,inner sep=2pt,color=yellow]{};  \node at (0,-2)[circle,fill,inner sep=2pt,color=red]{};  \node at (2,0)[circle,fill,inner sep=2pt,color=blue]{};  \node at (0,2)[circle,fill,inner sep=2pt,color=green]{};  \node at (-2,0)[circle,fill,inner sep=2pt,color=yellow]{};  \node at (0,-2)[circle,fill,inner sep=2pt,color=red]{}; 
\draw[thick,->, shorten >=2.5pt] (-5,-10) -- (-4,-7);
\draw[thick,->, shorten >=2.5pt] (-2,-10) -- (-2,-7);
\draw[thick,->, shorten >=2.5pt] (0,-10) -- (0,-7);
\draw[thick,->, shorten >=2.5pt] (4,-10) -- (2,-7);
\draw[thin,->] (-5,-7) -- (5,-7) node[anchor=west] {};
\node at (-4,-7)[circle,fill,inner sep=2pt,color=blue]{};
\node at (-2,-7)[circle,fill,inner sep=2pt,color=green]{}; 
\node at (0,-7)[circle,fill,inner sep=2pt,color=yellow]{};
\node at (2,-7)[circle,fill,inner sep=2pt,color=red]{};

\draw[thin,->] (-5,-10) -- (5,-10) node[anchor=west] {};
\node at (-5,-10)[circle,fill,inner sep=2pt,color=blue]{};
\node at (-2,-10)[circle,fill,inner sep=2pt,color=green]{}; 
\node at (0,-10)[circle,fill,inner sep=2pt,color=yellow]{};
\node at (4,-10)[circle,fill,inner sep=2pt,color=red]{};;
\end{tikzpicture}
 







\clearpage


\subsection{Reparametrization}
$\phi: (\alpha,\beta) \to (\alpha',\beta')$ is bijective, then it is invertible, its inverse is,
denoted: $\phi^{-1} : (\alpha',\beta') \to (\alpha,\beta)$.

If $\phi$ is smooth, is $\phi^{-1}$ smooth? Not necessarily!



\begin{definition}~\\
  $\gamma : (\alpha, \beta) \to \mathbb{R}^2$.\\\end{definition}


\newpage

\begin{tikzpicture}
\draw[thin,->] (-5,0) -- (5,0) node[anchor=west] {$x$};
\foreach \x in {-4,-3,-2,-1,0,1,2,3,4}
\draw (\x cm,2pt) -- (\x cm,-2pt);
\foreach \x in {-4,-3,-2,-1,1,2,3,4}
\node[anchor=north] at (\x cm,0) {\footnotesize \x};
\draw[thin,->] (0,-5) -- (0,5) node[anchor=south] {$y$};
\foreach \y in {-4,-3,-2,-1,0,1,2,3,4}
\draw (2pt, \y cm) -- (-2pt, \y cm);
\foreach \y in {-4,-3,-2,-1,1,2,3,4}
\node[anchor=west] at (0, \y cm) {\footnotesize \y};
\node[anchor=north west] at (0,0) {\footnotesize O};

\draw[color=blue,thick] (2,0) arc
    [
        start angle=0,
        end angle=290,
        x radius=2cm,
        y radius =2cm
    ] ;
 \node at (2,0)[circle,fill,inner sep=2pt,color=blue]{};  \node at (0,2)[circle,fill,inner sep=2pt,color=green]{};  \node at (-2,0)[circle,fill,inner sep=2pt,color=yellow]{};  \node at (0,-2)[circle,fill,inner sep=2pt,color=red]{};  \node at (2,0)[circle,fill,inner sep=2pt,color=blue]{};  \node at (0,2)[circle,fill,inner sep=2pt,color=green]{};  \node at (-2,0)[circle,fill,inner sep=2pt,color=yellow]{};  \node at (0,-2)[circle,fill,inner sep=2pt,color=red]{}; 
\draw[thick,->, shorten >=2.5pt] (-5,-10) -- (-4,-7);
\draw[thick,->, shorten >=2.5pt] (-2,-10) -- (-2,-7);
\draw[thick,->, shorten >=2.5pt] (0,-10) -- (0,-7);
\draw[thick,->, shorten >=2.5pt] (4,-10) -- (2,-7);
\draw[thin,->] (-5,-7) -- (5,-7) node[anchor=west] {};
\node at (-4,-7)[circle,fill,inner sep=2pt,color=blue]{};
\node at (-2,-7)[circle,fill,inner sep=2pt,color=green]{}; 
\node at (0,-7)[circle,fill,inner sep=2pt,color=yellow]{};
\node at (2,-7)[circle,fill,inner sep=2pt,color=red]{};

\draw[thin,->] (-5,-10) -- (5,-10) node[anchor=west] {};
\node at (-5,-10)[circle,fill,inner sep=2pt,color=blue]{};
\node at (-2,-10)[circle,fill,inner sep=2pt,color=green]{}; 
\node at (0,-10)[circle,fill,inner sep=2pt,color=yellow]{};
\node at (4,-10)[circle,fill,inner sep=2pt,color=red]{};;
\end{tikzpicture}
 







\clearpage


\subsection{Reparametrization}
$\phi: (\alpha,\beta) \to (\alpha',\beta')$ is bijective, then it is invertible, its inverse is,
denoted: $\phi^{-1} : (\alpha',\beta') \to (\alpha,\beta)$.

If $\phi$ is smooth, is $\phi^{-1}$ smooth? Not necessarily!



\begin{definition}~\\
  $\gamma : (\alpha, \beta) \to \mathbb{R}^2$.\\
  $\tilde{\gamma} : (\tilde{\alpha}, \tilde{\beta}) \to \mathbb{R}^2$.\\\end{definition}


\newpage

\begin{tikzpicture}
\draw[thin,->] (-5,0) -- (5,0) node[anchor=west] {$x$};
\foreach \x in {-4,-3,-2,-1,0,1,2,3,4}
\draw (\x cm,2pt) -- (\x cm,-2pt);
\foreach \x in {-4,-3,-2,-1,1,2,3,4}
\node[anchor=north] at (\x cm,0) {\footnotesize \x};
\draw[thin,->] (0,-5) -- (0,5) node[anchor=south] {$y$};
\foreach \y in {-4,-3,-2,-1,0,1,2,3,4}
\draw (2pt, \y cm) -- (-2pt, \y cm);
\foreach \y in {-4,-3,-2,-1,1,2,3,4}
\node[anchor=west] at (0, \y cm) {\footnotesize \y};
\node[anchor=north west] at (0,0) {\footnotesize O};

\draw[color=blue,thick] (2,0) arc
    [
        start angle=0,
        end angle=290,
        x radius=2cm,
        y radius =2cm
    ] ;
 \node at (2,0)[circle,fill,inner sep=2pt,color=blue]{};  \node at (0,2)[circle,fill,inner sep=2pt,color=green]{};  \node at (-2,0)[circle,fill,inner sep=2pt,color=yellow]{};  \node at (0,-2)[circle,fill,inner sep=2pt,color=red]{};  \node at (2,0)[circle,fill,inner sep=2pt,color=blue]{};  \node at (0,2)[circle,fill,inner sep=2pt,color=green]{};  \node at (-2,0)[circle,fill,inner sep=2pt,color=yellow]{};  \node at (0,-2)[circle,fill,inner sep=2pt,color=red]{}; 
\draw[thick,->, shorten >=2.5pt] (-5,-10) -- (-4,-7);
\draw[thick,->, shorten >=2.5pt] (-2,-10) -- (-2,-7);
\draw[thick,->, shorten >=2.5pt] (0,-10) -- (0,-7);
\draw[thick,->, shorten >=2.5pt] (4,-10) -- (2,-7);
\draw[thin,->] (-5,-7) -- (5,-7) node[anchor=west] {};
\node at (-4,-7)[circle,fill,inner sep=2pt,color=blue]{};
\node at (-2,-7)[circle,fill,inner sep=2pt,color=green]{}; 
\node at (0,-7)[circle,fill,inner sep=2pt,color=yellow]{};
\node at (2,-7)[circle,fill,inner sep=2pt,color=red]{};

\draw[thin,->] (-5,-10) -- (5,-10) node[anchor=west] {};
\node at (-5,-10)[circle,fill,inner sep=2pt,color=blue]{};
\node at (-2,-10)[circle,fill,inner sep=2pt,color=green]{}; 
\node at (0,-10)[circle,fill,inner sep=2pt,color=yellow]{};
\node at (4,-10)[circle,fill,inner sep=2pt,color=red]{};;
\end{tikzpicture}
 







\clearpage


\subsection{Reparametrization}
$\phi: (\alpha,\beta) \to (\alpha',\beta')$ is bijective, then it is invertible, its inverse is,
denoted: $\phi^{-1} : (\alpha',\beta') \to (\alpha,\beta)$.

If $\phi$ is smooth, is $\phi^{-1}$ smooth? Not necessarily!



\begin{definition}~\\
  $\gamma : (\alpha, \beta) \to \mathbb{R}^2$.\\
  $\tilde{\gamma} : (\tilde{\alpha}, \tilde{\beta}) \to \mathbb{R}^2$.\\
  If $\phi : (\tilde{\alpha}, \tilde{\beta})\to (\alpha, \beta)$ is bijective\end{definition}


\newpage

\begin{tikzpicture}
\draw[thin,->] (-5,0) -- (5,0) node[anchor=west] {$x$};
\foreach \x in {-4,-3,-2,-1,0,1,2,3,4}
\draw (\x cm,2pt) -- (\x cm,-2pt);
\foreach \x in {-4,-3,-2,-1,1,2,3,4}
\node[anchor=north] at (\x cm,0) {\footnotesize \x};
\draw[thin,->] (0,-5) -- (0,5) node[anchor=south] {$y$};
\foreach \y in {-4,-3,-2,-1,0,1,2,3,4}
\draw (2pt, \y cm) -- (-2pt, \y cm);
\foreach \y in {-4,-3,-2,-1,1,2,3,4}
\node[anchor=west] at (0, \y cm) {\footnotesize \y};
\node[anchor=north west] at (0,0) {\footnotesize O};

\draw[color=blue,thick] (2,0) arc
    [
        start angle=0,
        end angle=290,
        x radius=2cm,
        y radius =2cm
    ] ;
 \node at (2,0)[circle,fill,inner sep=2pt,color=blue]{};  \node at (0,2)[circle,fill,inner sep=2pt,color=green]{};  \node at (-2,0)[circle,fill,inner sep=2pt,color=yellow]{};  \node at (0,-2)[circle,fill,inner sep=2pt,color=red]{};  \node at (2,0)[circle,fill,inner sep=2pt,color=blue]{};  \node at (0,2)[circle,fill,inner sep=2pt,color=green]{};  \node at (-2,0)[circle,fill,inner sep=2pt,color=yellow]{};  \node at (0,-2)[circle,fill,inner sep=2pt,color=red]{}; 
\draw[thick,->, shorten >=2.5pt] (-5,-10) -- (-4,-7);
\draw[thick,->, shorten >=2.5pt] (-2,-10) -- (-2,-7);
\draw[thick,->, shorten >=2.5pt] (0,-10) -- (0,-7);
\draw[thick,->, shorten >=2.5pt] (4,-10) -- (2,-7);
\draw[thin,->] (-5,-7) -- (5,-7) node[anchor=west] {};
\node at (-4,-7)[circle,fill,inner sep=2pt,color=blue]{};
\node at (-2,-7)[circle,fill,inner sep=2pt,color=green]{}; 
\node at (0,-7)[circle,fill,inner sep=2pt,color=yellow]{};
\node at (2,-7)[circle,fill,inner sep=2pt,color=red]{};

\draw[thin,->] (-5,-10) -- (5,-10) node[anchor=west] {};
\node at (-5,-10)[circle,fill,inner sep=2pt,color=blue]{};
\node at (-2,-10)[circle,fill,inner sep=2pt,color=green]{}; 
\node at (0,-10)[circle,fill,inner sep=2pt,color=yellow]{};
\node at (4,-10)[circle,fill,inner sep=2pt,color=red]{};;
\end{tikzpicture}
 







\clearpage


\subsection{Reparametrization}
$\phi: (\alpha,\beta) \to (\alpha',\beta')$ is bijective, then it is invertible, its inverse is,
denoted: $\phi^{-1} : (\alpha',\beta') \to (\alpha,\beta)$.

If $\phi$ is smooth, is $\phi^{-1}$ smooth? Not necessarily!



\begin{definition}~\\
  $\gamma : (\alpha, \beta) \to \mathbb{R}^2$.\\
  $\tilde{\gamma} : (\tilde{\alpha}, \tilde{\beta}) \to \mathbb{R}^2$.\\
  If $\phi : (\tilde{\alpha}, \tilde{\beta})\to (\alpha, \beta)$ is bijective, smooth,\end{definition}


\newpage

\begin{tikzpicture}
\draw[thin,->] (-5,0) -- (5,0) node[anchor=west] {$x$};
\foreach \x in {-4,-3,-2,-1,0,1,2,3,4}
\draw (\x cm,2pt) -- (\x cm,-2pt);
\foreach \x in {-4,-3,-2,-1,1,2,3,4}
\node[anchor=north] at (\x cm,0) {\footnotesize \x};
\draw[thin,->] (0,-5) -- (0,5) node[anchor=south] {$y$};
\foreach \y in {-4,-3,-2,-1,0,1,2,3,4}
\draw (2pt, \y cm) -- (-2pt, \y cm);
\foreach \y in {-4,-3,-2,-1,1,2,3,4}
\node[anchor=west] at (0, \y cm) {\footnotesize \y};
\node[anchor=north west] at (0,0) {\footnotesize O};

\draw[color=blue,thick] (2,0) arc
    [
        start angle=0,
        end angle=290,
        x radius=2cm,
        y radius =2cm
    ] ;
 \node at (2,0)[circle,fill,inner sep=2pt,color=blue]{};  \node at (0,2)[circle,fill,inner sep=2pt,color=green]{};  \node at (-2,0)[circle,fill,inner sep=2pt,color=yellow]{};  \node at (0,-2)[circle,fill,inner sep=2pt,color=red]{};  \node at (2,0)[circle,fill,inner sep=2pt,color=blue]{};  \node at (0,2)[circle,fill,inner sep=2pt,color=green]{};  \node at (-2,0)[circle,fill,inner sep=2pt,color=yellow]{};  \node at (0,-2)[circle,fill,inner sep=2pt,color=red]{}; 
\draw[thick,->, shorten >=2.5pt] (-5,-10) -- (-4,-7);
\draw[thick,->, shorten >=2.5pt] (-2,-10) -- (-2,-7);
\draw[thick,->, shorten >=2.5pt] (0,-10) -- (0,-7);
\draw[thick,->, shorten >=2.5pt] (4,-10) -- (2,-7);
\draw[thin,->] (-5,-7) -- (5,-7) node[anchor=west] {};
\node at (-4,-7)[circle,fill,inner sep=2pt,color=blue]{};
\node at (-2,-7)[circle,fill,inner sep=2pt,color=green]{}; 
\node at (0,-7)[circle,fill,inner sep=2pt,color=yellow]{};
\node at (2,-7)[circle,fill,inner sep=2pt,color=red]{};

\draw[thin,->] (-5,-10) -- (5,-10) node[anchor=west] {};
\node at (-5,-10)[circle,fill,inner sep=2pt,color=blue]{};
\node at (-2,-10)[circle,fill,inner sep=2pt,color=green]{}; 
\node at (0,-10)[circle,fill,inner sep=2pt,color=yellow]{};
\node at (4,-10)[circle,fill,inner sep=2pt,color=red]{};;
\end{tikzpicture}
 







\clearpage


\subsection{Reparametrization}
$\phi: (\alpha,\beta) \to (\alpha',\beta')$ is bijective, then it is invertible, its inverse is,
denoted: $\phi^{-1} : (\alpha',\beta') \to (\alpha,\beta)$.

If $\phi$ is smooth, is $\phi^{-1}$ smooth? Not necessarily!



\begin{definition}~\\
  $\gamma : (\alpha, \beta) \to \mathbb{R}^2$.\\
  $\tilde{\gamma} : (\tilde{\alpha}, \tilde{\beta}) \to \mathbb{R}^2$.\\
  If $\phi : (\tilde{\alpha}, \tilde{\beta})\to (\alpha, \beta)$ is bijective, smooth, \\and its inverse, $\phi^{-1}$ is smooth,\end{definition}


\newpage

\begin{tikzpicture}
\draw[thin,->] (-5,0) -- (5,0) node[anchor=west] {$x$};
\foreach \x in {-4,-3,-2,-1,0,1,2,3,4}
\draw (\x cm,2pt) -- (\x cm,-2pt);
\foreach \x in {-4,-3,-2,-1,1,2,3,4}
\node[anchor=north] at (\x cm,0) {\footnotesize \x};
\draw[thin,->] (0,-5) -- (0,5) node[anchor=south] {$y$};
\foreach \y in {-4,-3,-2,-1,0,1,2,3,4}
\draw (2pt, \y cm) -- (-2pt, \y cm);
\foreach \y in {-4,-3,-2,-1,1,2,3,4}
\node[anchor=west] at (0, \y cm) {\footnotesize \y};
\node[anchor=north west] at (0,0) {\footnotesize O};

\draw[color=blue,thick] (2,0) arc
    [
        start angle=0,
        end angle=290,
        x radius=2cm,
        y radius =2cm
    ] ;
 \node at (2,0)[circle,fill,inner sep=2pt,color=blue]{};  \node at (0,2)[circle,fill,inner sep=2pt,color=green]{};  \node at (-2,0)[circle,fill,inner sep=2pt,color=yellow]{};  \node at (0,-2)[circle,fill,inner sep=2pt,color=red]{};  \node at (2,0)[circle,fill,inner sep=2pt,color=blue]{};  \node at (0,2)[circle,fill,inner sep=2pt,color=green]{};  \node at (-2,0)[circle,fill,inner sep=2pt,color=yellow]{};  \node at (0,-2)[circle,fill,inner sep=2pt,color=red]{}; 
\draw[thick,->, shorten >=2.5pt] (-5,-10) -- (-4,-7);
\draw[thick,->, shorten >=2.5pt] (-2,-10) -- (-2,-7);
\draw[thick,->, shorten >=2.5pt] (0,-10) -- (0,-7);
\draw[thick,->, shorten >=2.5pt] (4,-10) -- (2,-7);
\draw[thin,->] (-5,-7) -- (5,-7) node[anchor=west] {};
\node at (-4,-7)[circle,fill,inner sep=2pt,color=blue]{};
\node at (-2,-7)[circle,fill,inner sep=2pt,color=green]{}; 
\node at (0,-7)[circle,fill,inner sep=2pt,color=yellow]{};
\node at (2,-7)[circle,fill,inner sep=2pt,color=red]{};

\draw[thin,->] (-5,-10) -- (5,-10) node[anchor=west] {};
\node at (-5,-10)[circle,fill,inner sep=2pt,color=blue]{};
\node at (-2,-10)[circle,fill,inner sep=2pt,color=green]{}; 
\node at (0,-10)[circle,fill,inner sep=2pt,color=yellow]{};
\node at (4,-10)[circle,fill,inner sep=2pt,color=red]{};;
\end{tikzpicture}
 







\clearpage


\subsection{Reparametrization}
$\phi: (\alpha,\beta) \to (\alpha',\beta')$ is bijective, then it is invertible, its inverse is,
denoted: $\phi^{-1} : (\alpha',\beta') \to (\alpha,\beta)$.

If $\phi$ is smooth, is $\phi^{-1}$ smooth? Not necessarily!



\begin{definition}~\\
  $\gamma : (\alpha, \beta) \to \mathbb{R}^2$.\\
  $\tilde{\gamma} : (\tilde{\alpha}, \tilde{\beta}) \to \mathbb{R}^2$.\\
  If $\phi : (\tilde{\alpha}, \tilde{\beta})\to (\alpha, \beta)$ is bijective, smooth, \\and its inverse, $\phi^{-1}$ is smooth,\\ and
 $\tilde{\gamma}(t) = \gamma(\phi(t))$,\end{definition}


\newpage

\begin{tikzpicture}
\draw[thin,->] (-5,0) -- (5,0) node[anchor=west] {$x$};
\foreach \x in {-4,-3,-2,-1,0,1,2,3,4}
\draw (\x cm,2pt) -- (\x cm,-2pt);
\foreach \x in {-4,-3,-2,-1,1,2,3,4}
\node[anchor=north] at (\x cm,0) {\footnotesize \x};
\draw[thin,->] (0,-5) -- (0,5) node[anchor=south] {$y$};
\foreach \y in {-4,-3,-2,-1,0,1,2,3,4}
\draw (2pt, \y cm) -- (-2pt, \y cm);
\foreach \y in {-4,-3,-2,-1,1,2,3,4}
\node[anchor=west] at (0, \y cm) {\footnotesize \y};
\node[anchor=north west] at (0,0) {\footnotesize O};

\draw[color=blue,thick] (2,0) arc
    [
        start angle=0,
        end angle=290,
        x radius=2cm,
        y radius =2cm
    ] ;
 \node at (2,0)[circle,fill,inner sep=2pt,color=blue]{};  \node at (0,2)[circle,fill,inner sep=2pt,color=green]{};  \node at (-2,0)[circle,fill,inner sep=2pt,color=yellow]{};  \node at (0,-2)[circle,fill,inner sep=2pt,color=red]{};  \node at (2,0)[circle,fill,inner sep=2pt,color=blue]{};  \node at (0,2)[circle,fill,inner sep=2pt,color=green]{};  \node at (-2,0)[circle,fill,inner sep=2pt,color=yellow]{};  \node at (0,-2)[circle,fill,inner sep=2pt,color=red]{}; 
\draw[thick,->, shorten >=2.5pt] (-5,-10) -- (-4,-7);
\draw[thick,->, shorten >=2.5pt] (-2,-10) -- (-2,-7);
\draw[thick,->, shorten >=2.5pt] (0,-10) -- (0,-7);
\draw[thick,->, shorten >=2.5pt] (4,-10) -- (2,-7);
\draw[thin,->] (-5,-7) -- (5,-7) node[anchor=west] {};
\node at (-4,-7)[circle,fill,inner sep=2pt,color=blue]{};
\node at (-2,-7)[circle,fill,inner sep=2pt,color=green]{}; 
\node at (0,-7)[circle,fill,inner sep=2pt,color=yellow]{};
\node at (2,-7)[circle,fill,inner sep=2pt,color=red]{};

\draw[thin,->] (-5,-10) -- (5,-10) node[anchor=west] {};
\node at (-5,-10)[circle,fill,inner sep=2pt,color=blue]{};
\node at (-2,-10)[circle,fill,inner sep=2pt,color=green]{}; 
\node at (0,-10)[circle,fill,inner sep=2pt,color=yellow]{};
\node at (4,-10)[circle,fill,inner sep=2pt,color=red]{};;
\end{tikzpicture}
 







\clearpage


\subsection{Reparametrization}
$\phi: (\alpha,\beta) \to (\alpha',\beta')$ is bijective, then it is invertible, its inverse is,
denoted: $\phi^{-1} : (\alpha',\beta') \to (\alpha,\beta)$.

If $\phi$ is smooth, is $\phi^{-1}$ smooth? Not necessarily!



\begin{definition}~\\
  $\gamma : (\alpha, \beta) \to \mathbb{R}^2$.\\
  $\tilde{\gamma} : (\tilde{\alpha}, \tilde{\beta}) \to \mathbb{R}^2$.\\
  If $\phi : (\tilde{\alpha}, \tilde{\beta})\to (\alpha, \beta)$ is bijective, smooth, \\and its inverse, $\phi^{-1}$ is smooth,\\ and
 $\tilde{\gamma}(t) = \gamma(\phi(t))$,\\ then $\phi$ is called a reparametrization of $\gamma$.
\end{definition}



\newpage

\begin{tikzpicture}
\draw[thin,->] (-5,0) -- (5,0) node[anchor=west] {$x$};
\foreach \x in {-4,-3,-2,-1,0,1,2,3,4}
\draw (\x cm,2pt) -- (\x cm,-2pt);
\foreach \x in {-4,-3,-2,-1,1,2,3,4}
\node[anchor=north] at (\x cm,0) {\footnotesize \x};
\draw[thin,->] (0,-5) -- (0,5) node[anchor=south] {$y$};
\foreach \y in {-4,-3,-2,-1,0,1,2,3,4}
\draw (2pt, \y cm) -- (-2pt, \y cm);
\foreach \y in {-4,-3,-2,-1,1,2,3,4}
\node[anchor=west] at (0, \y cm) {\footnotesize \y};
\node[anchor=north west] at (0,0) {\footnotesize O};

\draw[color=blue,thick] (2,0) arc
    [
        start angle=0,
        end angle=290,
        x radius=2cm,
        y radius =2cm
    ] ;
 \node at (2,0)[circle,fill,inner sep=2pt,color=blue]{};  \node at (0,2)[circle,fill,inner sep=2pt,color=green]{};  \node at (-2,0)[circle,fill,inner sep=2pt,color=yellow]{};  \node at (0,-2)[circle,fill,inner sep=2pt,color=red]{};  \node at (2,0)[circle,fill,inner sep=2pt,color=blue]{};  \node at (0,2)[circle,fill,inner sep=2pt,color=green]{};  \node at (-2,0)[circle,fill,inner sep=2pt,color=yellow]{};  \node at (0,-2)[circle,fill,inner sep=2pt,color=red]{}; 
\draw[thick,->, shorten >=2.5pt] (-5,-10) -- (-4,-7);
\draw[thick,->, shorten >=2.5pt] (-2,-10) -- (-2,-7);
\draw[thick,->, shorten >=2.5pt] (0,-10) -- (0,-7);
\draw[thick,->, shorten >=2.5pt] (4,-10) -- (2,-7);
\draw[thin,->] (-5,-7) -- (5,-7) node[anchor=west] {};
\node at (-4,-7)[circle,fill,inner sep=2pt,color=blue]{};
\node at (-2,-7)[circle,fill,inner sep=2pt,color=green]{}; 
\node at (0,-7)[circle,fill,inner sep=2pt,color=yellow]{};
\node at (2,-7)[circle,fill,inner sep=2pt,color=red]{};

\draw[thin,->] (-5,-10) -- (5,-10) node[anchor=west] {};
\node at (-5,-10)[circle,fill,inner sep=2pt,color=blue]{};
\node at (-2,-10)[circle,fill,inner sep=2pt,color=green]{}; 
\node at (0,-10)[circle,fill,inner sep=2pt,color=yellow]{};
\node at (4,-10)[circle,fill,inner sep=2pt,color=red]{};;
\end{tikzpicture}
 







\clearpage


\subsection{Reparametrization}
$\phi: (\alpha,\beta) \to (\alpha',\beta')$ is bijective, then it is invertible, its inverse is,
denoted: $\phi^{-1} : (\alpha',\beta') \to (\alpha,\beta)$.

If $\phi$ is smooth, is $\phi^{-1}$ smooth? Not necessarily!



\begin{definition}~\\
  $\gamma : (\alpha, \beta) \to \mathbb{R}^2$.\\
  $\tilde{\gamma} : (\tilde{\alpha}, \tilde{\beta}) \to \mathbb{R}^2$.\\
  If $\phi : (\tilde{\alpha}, \tilde{\beta})\to (\alpha, \beta)$ is bijective, smooth, \\and its inverse, $\phi^{-1}$ is smooth,\\ and
 $\tilde{\gamma}(t) = \gamma(\phi(t))$,\\ then $\phi$ is called a reparametrization of $\gamma$.
\end{definition}

Explictly:


\newpage

\begin{tikzpicture}
\draw[thin,->] (-5,0) -- (5,0) node[anchor=west] {$x$};
\foreach \x in {-4,-3,-2,-1,0,1,2,3,4}
\draw (\x cm,2pt) -- (\x cm,-2pt);
\foreach \x in {-4,-3,-2,-1,1,2,3,4}
\node[anchor=north] at (\x cm,0) {\footnotesize \x};
\draw[thin,->] (0,-5) -- (0,5) node[anchor=south] {$y$};
\foreach \y in {-4,-3,-2,-1,0,1,2,3,4}
\draw (2pt, \y cm) -- (-2pt, \y cm);
\foreach \y in {-4,-3,-2,-1,1,2,3,4}
\node[anchor=west] at (0, \y cm) {\footnotesize \y};
\node[anchor=north west] at (0,0) {\footnotesize O};

\draw[color=blue,thick] (2,0) arc
    [
        start angle=0,
        end angle=290,
        x radius=2cm,
        y radius =2cm
    ] ;
 \node at (2,0)[circle,fill,inner sep=2pt,color=blue]{};  \node at (0,2)[circle,fill,inner sep=2pt,color=green]{};  \node at (-2,0)[circle,fill,inner sep=2pt,color=yellow]{};  \node at (0,-2)[circle,fill,inner sep=2pt,color=red]{};  \node at (2,0)[circle,fill,inner sep=2pt,color=blue]{};  \node at (0,2)[circle,fill,inner sep=2pt,color=green]{};  \node at (-2,0)[circle,fill,inner sep=2pt,color=yellow]{};  \node at (0,-2)[circle,fill,inner sep=2pt,color=red]{}; 
\draw[thick,->, shorten >=2.5pt] (-5,-10) -- (-4,-7);
\draw[thick,->, shorten >=2.5pt] (-2,-10) -- (-2,-7);
\draw[thick,->, shorten >=2.5pt] (0,-10) -- (0,-7);
\draw[thick,->, shorten >=2.5pt] (4,-10) -- (2,-7);
\draw[thin,->] (-5,-7) -- (5,-7) node[anchor=west] {};
\node at (-4,-7)[circle,fill,inner sep=2pt,color=blue]{};
\node at (-2,-7)[circle,fill,inner sep=2pt,color=green]{}; 
\node at (0,-7)[circle,fill,inner sep=2pt,color=yellow]{};
\node at (2,-7)[circle,fill,inner sep=2pt,color=red]{};

\draw[thin,->] (-5,-10) -- (5,-10) node[anchor=west] {};
\node at (-5,-10)[circle,fill,inner sep=2pt,color=blue]{};
\node at (-2,-10)[circle,fill,inner sep=2pt,color=green]{}; 
\node at (0,-10)[circle,fill,inner sep=2pt,color=yellow]{};
\node at (4,-10)[circle,fill,inner sep=2pt,color=red]{};;
\end{tikzpicture}
 







\clearpage


\subsection{Reparametrization}
$\phi: (\alpha,\beta) \to (\alpha',\beta')$ is bijective, then it is invertible, its inverse is,
denoted: $\phi^{-1} : (\alpha',\beta') \to (\alpha,\beta)$.

If $\phi$ is smooth, is $\phi^{-1}$ smooth? Not necessarily!



\begin{definition}~\\
  $\gamma : (\alpha, \beta) \to \mathbb{R}^2$.\\
  $\tilde{\gamma} : (\tilde{\alpha}, \tilde{\beta}) \to \mathbb{R}^2$.\\
  If $\phi : (\tilde{\alpha}, \tilde{\beta})\to (\alpha, \beta)$ is bijective, smooth, \\and its inverse, $\phi^{-1}$ is smooth,\\ and
 $\tilde{\gamma}(t) = \gamma(\phi(t))$,\\ then $\phi$ is called a reparametrization of $\gamma$.
\end{definition}

Explictly:

$\gamma(t) = (f_1(t), f_2(t))$


\newpage

\begin{tikzpicture}
\draw[thin,->] (-5,0) -- (5,0) node[anchor=west] {$x$};
\foreach \x in {-4,-3,-2,-1,0,1,2,3,4}
\draw (\x cm,2pt) -- (\x cm,-2pt);
\foreach \x in {-4,-3,-2,-1,1,2,3,4}
\node[anchor=north] at (\x cm,0) {\footnotesize \x};
\draw[thin,->] (0,-5) -- (0,5) node[anchor=south] {$y$};
\foreach \y in {-4,-3,-2,-1,0,1,2,3,4}
\draw (2pt, \y cm) -- (-2pt, \y cm);
\foreach \y in {-4,-3,-2,-1,1,2,3,4}
\node[anchor=west] at (0, \y cm) {\footnotesize \y};
\node[anchor=north west] at (0,0) {\footnotesize O};

\draw[color=blue,thick] (2,0) arc
    [
        start angle=0,
        end angle=290,
        x radius=2cm,
        y radius =2cm
    ] ;
 \node at (2,0)[circle,fill,inner sep=2pt,color=blue]{};  \node at (0,2)[circle,fill,inner sep=2pt,color=green]{};  \node at (-2,0)[circle,fill,inner sep=2pt,color=yellow]{};  \node at (0,-2)[circle,fill,inner sep=2pt,color=red]{};  \node at (2,0)[circle,fill,inner sep=2pt,color=blue]{};  \node at (0,2)[circle,fill,inner sep=2pt,color=green]{};  \node at (-2,0)[circle,fill,inner sep=2pt,color=yellow]{};  \node at (0,-2)[circle,fill,inner sep=2pt,color=red]{}; 
\draw[thick,->, shorten >=2.5pt] (-5,-10) -- (-4,-7);
\draw[thick,->, shorten >=2.5pt] (-2,-10) -- (-2,-7);
\draw[thick,->, shorten >=2.5pt] (0,-10) -- (0,-7);
\draw[thick,->, shorten >=2.5pt] (4,-10) -- (2,-7);
\draw[thin,->] (-5,-7) -- (5,-7) node[anchor=west] {};
\node at (-4,-7)[circle,fill,inner sep=2pt,color=blue]{};
\node at (-2,-7)[circle,fill,inner sep=2pt,color=green]{}; 
\node at (0,-7)[circle,fill,inner sep=2pt,color=yellow]{};
\node at (2,-7)[circle,fill,inner sep=2pt,color=red]{};

\draw[thin,->] (-5,-10) -- (5,-10) node[anchor=west] {};
\node at (-5,-10)[circle,fill,inner sep=2pt,color=blue]{};
\node at (-2,-10)[circle,fill,inner sep=2pt,color=green]{}; 
\node at (0,-10)[circle,fill,inner sep=2pt,color=yellow]{};
\node at (4,-10)[circle,fill,inner sep=2pt,color=red]{};;
\end{tikzpicture}
 







\clearpage


\subsection{Reparametrization}
$\phi: (\alpha,\beta) \to (\alpha',\beta')$ is bijective, then it is invertible, its inverse is,
denoted: $\phi^{-1} : (\alpha',\beta') \to (\alpha,\beta)$.

If $\phi$ is smooth, is $\phi^{-1}$ smooth? Not necessarily!



\begin{definition}~\\
  $\gamma : (\alpha, \beta) \to \mathbb{R}^2$.\\
  $\tilde{\gamma} : (\tilde{\alpha}, \tilde{\beta}) \to \mathbb{R}^2$.\\
  If $\phi : (\tilde{\alpha}, \tilde{\beta})\to (\alpha, \beta)$ is bijective, smooth, \\and its inverse, $\phi^{-1}$ is smooth,\\ and
 $\tilde{\gamma}(t) = \gamma(\phi(t))$,\\ then $\phi$ is called a reparametrization of $\gamma$.
\end{definition}

Explictly:

$\gamma(t) = (f_1(t), f_2(t))$\\
$\gamma(\phi(t)) = (f_1(\phi(t)),$


\newpage

\begin{tikzpicture}
\draw[thin,->] (-5,0) -- (5,0) node[anchor=west] {$x$};
\foreach \x in {-4,-3,-2,-1,0,1,2,3,4}
\draw (\x cm,2pt) -- (\x cm,-2pt);
\foreach \x in {-4,-3,-2,-1,1,2,3,4}
\node[anchor=north] at (\x cm,0) {\footnotesize \x};
\draw[thin,->] (0,-5) -- (0,5) node[anchor=south] {$y$};
\foreach \y in {-4,-3,-2,-1,0,1,2,3,4}
\draw (2pt, \y cm) -- (-2pt, \y cm);
\foreach \y in {-4,-3,-2,-1,1,2,3,4}
\node[anchor=west] at (0, \y cm) {\footnotesize \y};
\node[anchor=north west] at (0,0) {\footnotesize O};

\draw[color=blue,thick] (2,0) arc
    [
        start angle=0,
        end angle=290,
        x radius=2cm,
        y radius =2cm
    ] ;
 \node at (2,0)[circle,fill,inner sep=2pt,color=blue]{};  \node at (0,2)[circle,fill,inner sep=2pt,color=green]{};  \node at (-2,0)[circle,fill,inner sep=2pt,color=yellow]{};  \node at (0,-2)[circle,fill,inner sep=2pt,color=red]{};  \node at (2,0)[circle,fill,inner sep=2pt,color=blue]{};  \node at (0,2)[circle,fill,inner sep=2pt,color=green]{};  \node at (-2,0)[circle,fill,inner sep=2pt,color=yellow]{};  \node at (0,-2)[circle,fill,inner sep=2pt,color=red]{}; 
\draw[thick,->, shorten >=2.5pt] (-5,-10) -- (-4,-7);
\draw[thick,->, shorten >=2.5pt] (-2,-10) -- (-2,-7);
\draw[thick,->, shorten >=2.5pt] (0,-10) -- (0,-7);
\draw[thick,->, shorten >=2.5pt] (4,-10) -- (2,-7);
\draw[thin,->] (-5,-7) -- (5,-7) node[anchor=west] {};
\node at (-4,-7)[circle,fill,inner sep=2pt,color=blue]{};
\node at (-2,-7)[circle,fill,inner sep=2pt,color=green]{}; 
\node at (0,-7)[circle,fill,inner sep=2pt,color=yellow]{};
\node at (2,-7)[circle,fill,inner sep=2pt,color=red]{};

\draw[thin,->] (-5,-10) -- (5,-10) node[anchor=west] {};
\node at (-5,-10)[circle,fill,inner sep=2pt,color=blue]{};
\node at (-2,-10)[circle,fill,inner sep=2pt,color=green]{}; 
\node at (0,-10)[circle,fill,inner sep=2pt,color=yellow]{};
\node at (4,-10)[circle,fill,inner sep=2pt,color=red]{};;
\end{tikzpicture}
 







\clearpage


\subsection{Reparametrization}
$\phi: (\alpha,\beta) \to (\alpha',\beta')$ is bijective, then it is invertible, its inverse is,
denoted: $\phi^{-1} : (\alpha',\beta') \to (\alpha,\beta)$.

If $\phi$ is smooth, is $\phi^{-1}$ smooth? Not necessarily!



\begin{definition}~\\
  $\gamma : (\alpha, \beta) \to \mathbb{R}^2$.\\
  $\tilde{\gamma} : (\tilde{\alpha}, \tilde{\beta}) \to \mathbb{R}^2$.\\
  If $\phi : (\tilde{\alpha}, \tilde{\beta})\to (\alpha, \beta)$ is bijective, smooth, \\and its inverse, $\phi^{-1}$ is smooth,\\ and
 $\tilde{\gamma}(t) = \gamma(\phi(t))$,\\ then $\phi$ is called a reparametrization of $\gamma$.
\end{definition}

Explictly:

$\gamma(t) = (f_1(t), f_2(t))$\\
$\gamma(\phi(t)) = (f_1(\phi(t)), f_2(\phi(t)))$



\newpage

\begin{tikzpicture}
\draw[thin,->] (-5,0) -- (5,0) node[anchor=west] {$x$};
\foreach \x in {-4,-3,-2,-1,0,1,2,3,4}
\draw (\x cm,2pt) -- (\x cm,-2pt);
\foreach \x in {-4,-3,-2,-1,1,2,3,4}
\node[anchor=north] at (\x cm,0) {\footnotesize \x};
\draw[thin,->] (0,-5) -- (0,5) node[anchor=south] {$y$};
\foreach \y in {-4,-3,-2,-1,0,1,2,3,4}
\draw (2pt, \y cm) -- (-2pt, \y cm);
\foreach \y in {-4,-3,-2,-1,1,2,3,4}
\node[anchor=west] at (0, \y cm) {\footnotesize \y};
\node[anchor=north west] at (0,0) {\footnotesize O};

\draw[color=blue,thick] (2,0) arc
    [
        start angle=0,
        end angle=290,
        x radius=2cm,
        y radius =2cm
    ] ;
 \node at (2,0)[circle,fill,inner sep=2pt,color=blue]{};  \node at (0,2)[circle,fill,inner sep=2pt,color=green]{};  \node at (-2,0)[circle,fill,inner sep=2pt,color=yellow]{};  \node at (0,-2)[circle,fill,inner sep=2pt,color=red]{};  \node at (2,0)[circle,fill,inner sep=2pt,color=blue]{};  \node at (0,2)[circle,fill,inner sep=2pt,color=green]{};  \node at (-2,0)[circle,fill,inner sep=2pt,color=yellow]{};  \node at (0,-2)[circle,fill,inner sep=2pt,color=red]{}; 
\draw[thick,->, shorten >=2.5pt] (-5,-10) -- (-4,-7);
\draw[thick,->, shorten >=2.5pt] (-2,-10) -- (-2,-7);
\draw[thick,->, shorten >=2.5pt] (0,-10) -- (0,-7);
\draw[thick,->, shorten >=2.5pt] (4,-10) -- (2,-7);
\draw[thin,->] (-5,-7) -- (5,-7) node[anchor=west] {};
\node at (-4,-7)[circle,fill,inner sep=2pt,color=blue]{};
\node at (-2,-7)[circle,fill,inner sep=2pt,color=green]{}; 
\node at (0,-7)[circle,fill,inner sep=2pt,color=yellow]{};
\node at (2,-7)[circle,fill,inner sep=2pt,color=red]{};

\draw[thin,->] (-5,-10) -- (5,-10) node[anchor=west] {};
\node at (-5,-10)[circle,fill,inner sep=2pt,color=blue]{};
\node at (-2,-10)[circle,fill,inner sep=2pt,color=green]{}; 
\node at (0,-10)[circle,fill,inner sep=2pt,color=yellow]{};
\node at (4,-10)[circle,fill,inner sep=2pt,color=red]{};;
\end{tikzpicture}
 








\clearpage



\subsection{Example}
  $\gamma : (-1, 1) \to \mathbb{R}^2$


\clearpage



\subsection{Example}
  $\gamma : (-1, 1) \to \mathbb{R}^2$\\
  $\gamma(t) = (t,t)$


\clearpage



\subsection{Example}
  $\gamma : (-1, 1) \to \mathbb{R}^2$\\
  $\gamma(t) = (t,t)$\\

  $\tilde{\gamma} : (-1/2, 1/2) \to \mathbb{R}^2$.


\clearpage



\subsection{Example}
  $\gamma : (-1, 1) \to \mathbb{R}^2$\\
  $\gamma(t) = (t,t)$\\

  $\tilde{\gamma} : (-1/2, 1/2) \to \mathbb{R}^2$.\\
  $\tilde{\gamma}(t) = (2t,2t)$


\clearpage



\subsection{Example}
  $\gamma : (-1, 1) \to \mathbb{R}^2$\\
  $\gamma(t) = (t,t)$\\

  $\tilde{\gamma} : (-1/2, 1/2) \to \mathbb{R}^2$.\\
  $\tilde{\gamma}(t) = (2t,2t)$\\

  $\phi: (-1/2,1/2) \to (-1,1)$


\clearpage



\subsection{Example}
  $\gamma : (-1, 1) \to \mathbb{R}^2$\\
  $\gamma(t) = (t,t)$\\

  $\tilde{\gamma} : (-1/2, 1/2) \to \mathbb{R}^2$.\\
  $\tilde{\gamma}(t) = (2t,2t)$\\

  $\phi: (-1/2,1/2) \to (-1,1)$\\
  $\phi(t) = 2t$\\


\clearpage



\subsection{Example}
  $\gamma : (-1, 1) \to \mathbb{R}^2$\\
  $\gamma(t) = (t,t)$\\

  $\tilde{\gamma} : (-1/2, 1/2) \to \mathbb{R}^2$.\\
  $\tilde{\gamma}(t) = (2t,2t)$\\

  $\phi: (-1/2,1/2) \to (-1,1)$\\
  $\phi(t) = 2t$\\
  So that $\tilde{\gamma}(t)$


\clearpage



\subsection{Example}
  $\gamma : (-1, 1) \to \mathbb{R}^2$\\
  $\gamma(t) = (t,t)$\\

  $\tilde{\gamma} : (-1/2, 1/2) \to \mathbb{R}^2$.\\
  $\tilde{\gamma}(t) = (2t,2t)$\\

  $\phi: (-1/2,1/2) \to (-1,1)$\\
  $\phi(t) = 2t$\\
  So that $\tilde{\gamma}(t) = \gamma(\phi(t))$


\clearpage



\subsection{Example}
  $\gamma : (-1, 1) \to \mathbb{R}^2$\\
  $\gamma(t) = (t,t)$\\

  $\tilde{\gamma} : (-1/2, 1/2) \to \mathbb{R}^2$.\\
  $\tilde{\gamma}(t) = (2t,2t)$\\

  $\phi: (-1/2,1/2) \to (-1,1)$\\
  $\phi(t) = 2t$\\
  So that $\tilde{\gamma}(t) = \gamma(\phi(t)) = \gamma(2t)$


\clearpage



\subsection{Example}
  $\gamma : (-1, 1) \to \mathbb{R}^2$\\
  $\gamma(t) = (t,t)$\\

  $\tilde{\gamma} : (-1/2, 1/2) \to \mathbb{R}^2$.\\
  $\tilde{\gamma}(t) = (2t,2t)$\\

  $\phi: (-1/2,1/2) \to (-1,1)$\\
  $\phi(t) = 2t$\\
  So that $\tilde{\gamma}(t) = \gamma(\phi(t)) = \gamma(2t) = (2t, 2t)$






\clearpage



\subsection{Vectors}
\begin{tikzpicture}
\draw[thin,->] (-5,0) -- (5,0) node[anchor=west] {$x$};
\foreach \x in {-4,-3,-2,-1,0,1,2,3,4}
\draw (\x cm,2pt) -- (\x cm,-2pt);
\foreach \x in {-4,-3,-2,-1,1,2,3,4}
\node[anchor=north] at (\x cm,0) {\footnotesize \x};
\draw[thin,->] (0,-5) -- (0,5) node[anchor=south] {$y$};
\foreach \y in {-4,-3,-2,-1,0,1,2,3,4}
\draw (2pt, \y cm) -- (-2pt, \y cm);
\foreach \y in {-4,-3,-2,-1,1,2,3,4}
\node[anchor=west] at (0, \y cm) {\footnotesize \y};
\node[anchor=north west] at (0,0) {\footnotesize O};
;
\end{tikzpicture}

\newpage







\clearpage



\subsection{Vectors}
\begin{tikzpicture}
\draw[thin,->] (-5,0) -- (5,0) node[anchor=west] {$x$};
\foreach \x in {-4,-3,-2,-1,0,1,2,3,4}
\draw (\x cm,2pt) -- (\x cm,-2pt);
\foreach \x in {-4,-3,-2,-1,1,2,3,4}
\node[anchor=north] at (\x cm,0) {\footnotesize \x};
\draw[thin,->] (0,-5) -- (0,5) node[anchor=south] {$y$};
\foreach \y in {-4,-3,-2,-1,0,1,2,3,4}
\draw (2pt, \y cm) -- (-2pt, \y cm);
\foreach \y in {-4,-3,-2,-1,1,2,3,4}
\node[anchor=west] at (0, \y cm) {\footnotesize \y};
\node[anchor=north west] at (0,0) {\footnotesize O};
\node at (2,3)[circle,fill,inner sep=1.5pt,color=blue]{} ;
\end{tikzpicture}

\newpage








\clearpage



\subsection{Vectors}
\begin{tikzpicture}
\draw[thin,->] (-5,0) -- (5,0) node[anchor=west] {$x$};
\foreach \x in {-4,-3,-2,-1,0,1,2,3,4}
\draw (\x cm,2pt) -- (\x cm,-2pt);
\foreach \x in {-4,-3,-2,-1,1,2,3,4}
\node[anchor=north] at (\x cm,0) {\footnotesize \x};
\draw[thin,->] (0,-5) -- (0,5) node[anchor=south] {$y$};
\foreach \y in {-4,-3,-2,-1,0,1,2,3,4}
\draw (2pt, \y cm) -- (-2pt, \y cm);
\foreach \y in {-4,-3,-2,-1,1,2,3,4}
\node[anchor=west] at (0, \y cm) {\footnotesize \y};
\node[anchor=north west] at (0,0) {\footnotesize O};
 \draw[thick,->,color=blue] (0,0) -- (2,3); ;
\end{tikzpicture}

\newpage










\clearpage



\subsection{Vectors}
\begin{tikzpicture}
\draw[thin,->] (-5,0) -- (5,0) node[anchor=west] {$x$};
\foreach \x in {-4,-3,-2,-1,0,1,2,3,4}
\draw (\x cm,2pt) -- (\x cm,-2pt);
\foreach \x in {-4,-3,-2,-1,1,2,3,4}
\node[anchor=north] at (\x cm,0) {\footnotesize \x};
\draw[thin,->] (0,-5) -- (0,5) node[anchor=south] {$y$};
\foreach \y in {-4,-3,-2,-1,0,1,2,3,4}
\draw (2pt, \y cm) -- (-2pt, \y cm);
\foreach \y in {-4,-3,-2,-1,1,2,3,4}
\node[anchor=west] at (0, \y cm) {\footnotesize \y};
\node[anchor=north west] at (0,0) {\footnotesize O};
 \draw[thick,->,color=blue] (0,0) -- (2,3); 
\draw[thick,->,color=blue] (0,0) -- (1,-1); ;
\end{tikzpicture}

\newpage

$v=(2,3)$









\clearpage



\subsection{Vectors}
\begin{tikzpicture}
\draw[thin,->] (-5,0) -- (5,0) node[anchor=west] {$x$};
\foreach \x in {-4,-3,-2,-1,0,1,2,3,4}
\draw (\x cm,2pt) -- (\x cm,-2pt);
\foreach \x in {-4,-3,-2,-1,1,2,3,4}
\node[anchor=north] at (\x cm,0) {\footnotesize \x};
\draw[thin,->] (0,-5) -- (0,5) node[anchor=south] {$y$};
\foreach \y in {-4,-3,-2,-1,0,1,2,3,4}
\draw (2pt, \y cm) -- (-2pt, \y cm);
\foreach \y in {-4,-3,-2,-1,1,2,3,4}
\node[anchor=west] at (0, \y cm) {\footnotesize \y};
\node[anchor=north west] at (0,0) {\footnotesize O};
 \draw[thick,->,color=blue] (0,0) -- (2,3); 
\draw[thick,->,color=blue] (0,0) -- (1,-1); ;
\end{tikzpicture}

\newpage

$v=(2,3)$\\
$w=(1,-1)$

\sep










\clearpage



\subsection{Vectors}
\begin{tikzpicture}
\draw[thin,->] (-5,0) -- (5,0) node[anchor=west] {$x$};
\foreach \x in {-4,-3,-2,-1,0,1,2,3,4}
\draw (\x cm,2pt) -- (\x cm,-2pt);
\foreach \x in {-4,-3,-2,-1,1,2,3,4}
\node[anchor=north] at (\x cm,0) {\footnotesize \x};
\draw[thin,->] (0,-5) -- (0,5) node[anchor=south] {$y$};
\foreach \y in {-4,-3,-2,-1,0,1,2,3,4}
\draw (2pt, \y cm) -- (-2pt, \y cm);
\foreach \y in {-4,-3,-2,-1,1,2,3,4}
\node[anchor=west] at (0, \y cm) {\footnotesize \y};
\node[anchor=north west] at (0,0) {\footnotesize O};
 \draw[thick,->,color=blue] (0,0) -- (2,3); 
\draw[thick,->,color=blue] (0,0) -- (1,-1); ;
\end{tikzpicture}

\newpage

$v=(2,3)$\\
$w=(1,-1)$

\sep
Vector addition :\\










\clearpage



\subsection{Vectors}
\begin{tikzpicture}
\draw[thin,->] (-5,0) -- (5,0) node[anchor=west] {$x$};
\foreach \x in {-4,-3,-2,-1,0,1,2,3,4}
\draw (\x cm,2pt) -- (\x cm,-2pt);
\foreach \x in {-4,-3,-2,-1,1,2,3,4}
\node[anchor=north] at (\x cm,0) {\footnotesize \x};
\draw[thin,->] (0,-5) -- (0,5) node[anchor=south] {$y$};
\foreach \y in {-4,-3,-2,-1,0,1,2,3,4}
\draw (2pt, \y cm) -- (-2pt, \y cm);
\foreach \y in {-4,-3,-2,-1,1,2,3,4}
\node[anchor=west] at (0, \y cm) {\footnotesize \y};
\node[anchor=north west] at (0,0) {\footnotesize O};
 \draw[thick,->,color=blue] (0,0) -- (2,3); 
\draw[thick,->,color=blue] (0,0) -- (1,-1); 
\draw[thick,->,color=blue] (0,0) -- (3,2); ;
\end{tikzpicture}

\newpage

$v=(2,3)$\\
$w=(1,-1)$

\sep
Vector addition :\\
$v+w = (3,2)$












\clearpage



\subsection{Vectors}
\begin{tikzpicture}
\draw[thin,->] (-5,0) -- (5,0) node[anchor=west] {$x$};
\foreach \x in {-4,-3,-2,-1,0,1,2,3,4}
\draw (\x cm,2pt) -- (\x cm,-2pt);
\foreach \x in {-4,-3,-2,-1,1,2,3,4}
\node[anchor=north] at (\x cm,0) {\footnotesize \x};
\draw[thin,->] (0,-5) -- (0,5) node[anchor=south] {$y$};
\foreach \y in {-4,-3,-2,-1,0,1,2,3,4}
\draw (2pt, \y cm) -- (-2pt, \y cm);
\foreach \y in {-4,-3,-2,-1,1,2,3,4}
\node[anchor=west] at (0, \y cm) {\footnotesize \y};
\node[anchor=north west] at (0,0) {\footnotesize O};
 \draw[thick,->,color=blue] (0,0) -- (2,3); 
 \draw[thick,->,color=blue] (2,3) -- (3,2); ; 
\draw[thick,->,color=blue] (0,0) -- (3,2); ;
\end{tikzpicture}

\newpage

$v=(2,3)$\\
$w=(1,-1)$

\sep
Vector addition :\\
$v+w = (3,2)$













\clearpage



\subsection{Vectors}
\begin{tikzpicture}
\draw[thin,->] (-5,0) -- (5,0) node[anchor=west] {$x$};
\foreach \x in {-4,-3,-2,-1,0,1,2,3,4}
\draw (\x cm,2pt) -- (\x cm,-2pt);
\foreach \x in {-4,-3,-2,-1,1,2,3,4}
\node[anchor=north] at (\x cm,0) {\footnotesize \x};
\draw[thin,->] (0,-5) -- (0,5) node[anchor=south] {$y$};
\foreach \y in {-4,-3,-2,-1,0,1,2,3,4}
\draw (2pt, \y cm) -- (-2pt, \y cm);
\foreach \y in {-4,-3,-2,-1,1,2,3,4}
\node[anchor=west] at (0, \y cm) {\footnotesize \y};
\node[anchor=north west] at (0,0) {\footnotesize O};
 \draw[thick,->,color=blue] (0,0) -- (2,3); 
 \draw[thick,->,color=blue] (2,3) -- (3,2); ; 
\draw[thick,->,color=blue] (0,0) -- (3,2); ;
\end{tikzpicture}

\newpage

$v=(2,3)$\\
$w=(1,-1)$

\sep
Vector addition :\\
$v+w = (3,2)$



In general:\\










\clearpage



\subsection{Vectors}
\begin{tikzpicture}
\draw[thin,->] (-5,0) -- (5,0) node[anchor=west] {$x$};
\foreach \x in {-4,-3,-2,-1,0,1,2,3,4}
\draw (\x cm,2pt) -- (\x cm,-2pt);
\foreach \x in {-4,-3,-2,-1,1,2,3,4}
\node[anchor=north] at (\x cm,0) {\footnotesize \x};
\draw[thin,->] (0,-5) -- (0,5) node[anchor=south] {$y$};
\foreach \y in {-4,-3,-2,-1,0,1,2,3,4}
\draw (2pt, \y cm) -- (-2pt, \y cm);
\foreach \y in {-4,-3,-2,-1,1,2,3,4}
\node[anchor=west] at (0, \y cm) {\footnotesize \y};
\node[anchor=north west] at (0,0) {\footnotesize O};
 \draw[thick,->,color=blue] (0,0) -- (2,3); 
 \draw[thick,->,color=blue] (2,3) -- (3,2); ; 
\draw[thick,->,color=blue] (0,0) -- (3,2); ;
\end{tikzpicture}

\newpage

$v=(2,3)$\\
$w=(1,-1)$

\sep
Vector addition :\\
$v+w = (3,2)$



In general:\\
$(x_1,y_2) + (x_2,y_2) $









\clearpage



\subsection{Vectors}
\begin{tikzpicture}
\draw[thin,->] (-5,0) -- (5,0) node[anchor=west] {$x$};
\foreach \x in {-4,-3,-2,-1,0,1,2,3,4}
\draw (\x cm,2pt) -- (\x cm,-2pt);
\foreach \x in {-4,-3,-2,-1,1,2,3,4}
\node[anchor=north] at (\x cm,0) {\footnotesize \x};
\draw[thin,->] (0,-5) -- (0,5) node[anchor=south] {$y$};
\foreach \y in {-4,-3,-2,-1,0,1,2,3,4}
\draw (2pt, \y cm) -- (-2pt, \y cm);
\foreach \y in {-4,-3,-2,-1,1,2,3,4}
\node[anchor=west] at (0, \y cm) {\footnotesize \y};
\node[anchor=north west] at (0,0) {\footnotesize O};
 \draw[thick,->,color=blue] (0,0) -- (2,3); 
 \draw[thick,->,color=blue] (2,3) -- (3,2); ; 
\draw[thick,->,color=blue] (0,0) -- (3,2); ;
\end{tikzpicture}

\newpage

$v=(2,3)$\\
$w=(1,-1)$

\sep
Vector addition :\\
$v+w = (3,2)$



In general:\\
$(x_1,y_2) + (x_2,y_2) := (x_1 + x_2$









\clearpage



\subsection{Vectors}
\begin{tikzpicture}
\draw[thin,->] (-5,0) -- (5,0) node[anchor=west] {$x$};
\foreach \x in {-4,-3,-2,-1,0,1,2,3,4}
\draw (\x cm,2pt) -- (\x cm,-2pt);
\foreach \x in {-4,-3,-2,-1,1,2,3,4}
\node[anchor=north] at (\x cm,0) {\footnotesize \x};
\draw[thin,->] (0,-5) -- (0,5) node[anchor=south] {$y$};
\foreach \y in {-4,-3,-2,-1,0,1,2,3,4}
\draw (2pt, \y cm) -- (-2pt, \y cm);
\foreach \y in {-4,-3,-2,-1,1,2,3,4}
\node[anchor=west] at (0, \y cm) {\footnotesize \y};
\node[anchor=north west] at (0,0) {\footnotesize O};
 \draw[thick,->,color=blue] (0,0) -- (2,3); 
 \draw[thick,->,color=blue] (2,3) -- (3,2); ; 
\draw[thick,->,color=blue] (0,0) -- (3,2); ;
\end{tikzpicture}

\newpage

$v=(2,3)$\\
$w=(1,-1)$

\sep
Vector addition :\\
$v+w = (3,2)$



In general:\\
$(x_1,y_2) + (x_2,y_2) := (x_1 + x_2, y_1 + y_2)$\\










\clearpage



\subsection{Vectors}
\begin{tikzpicture}
\draw[thin,->] (-5,0) -- (5,0) node[anchor=west] {$x$};
\foreach \x in {-4,-3,-2,-1,0,1,2,3,4}
\draw (\x cm,2pt) -- (\x cm,-2pt);
\foreach \x in {-4,-3,-2,-1,1,2,3,4}
\node[anchor=north] at (\x cm,0) {\footnotesize \x};
\draw[thin,->] (0,-5) -- (0,5) node[anchor=south] {$y$};
\foreach \y in {-4,-3,-2,-1,0,1,2,3,4}
\draw (2pt, \y cm) -- (-2pt, \y cm);
\foreach \y in {-4,-3,-2,-1,1,2,3,4}
\node[anchor=west] at (0, \y cm) {\footnotesize \y};
\node[anchor=north west] at (0,0) {\footnotesize O};
 \draw[thick,->,color=blue] (0,0) -- (2,3); 
 \draw[thick,->,color=blue] (2,3) -- (3,2); ; 
\draw[thick,->,color=blue] (0,0) -- (3,2); ;
\end{tikzpicture}

\newpage

$v=(2,3)$\\
$w=(1,-1)$

\sep
Vector addition :\\
$v+w = (3,2)$



In general:\\
$(x_1,y_2) + (x_2,y_2) := (x_1 + x_2, y_1 + y_2)$\\
$(x_1,y_2) - (x_2,y_2) $









\clearpage



\subsection{Vectors}
\begin{tikzpicture}
\draw[thin,->] (-5,0) -- (5,0) node[anchor=west] {$x$};
\foreach \x in {-4,-3,-2,-1,0,1,2,3,4}
\draw (\x cm,2pt) -- (\x cm,-2pt);
\foreach \x in {-4,-3,-2,-1,1,2,3,4}
\node[anchor=north] at (\x cm,0) {\footnotesize \x};
\draw[thin,->] (0,-5) -- (0,5) node[anchor=south] {$y$};
\foreach \y in {-4,-3,-2,-1,0,1,2,3,4}
\draw (2pt, \y cm) -- (-2pt, \y cm);
\foreach \y in {-4,-3,-2,-1,1,2,3,4}
\node[anchor=west] at (0, \y cm) {\footnotesize \y};
\node[anchor=north west] at (0,0) {\footnotesize O};
 \draw[thick,->,color=blue] (0,0) -- (2,3); 
 \draw[thick,->,color=blue] (2,3) -- (3,2); ; 
\draw[thick,->,color=blue] (0,0) -- (3,2); ;
\end{tikzpicture}

\newpage

$v=(2,3)$\\
$w=(1,-1)$

\sep
Vector addition :\\
$v+w = (3,2)$



In general:\\
$(x_1,y_2) + (x_2,y_2) := (x_1 + x_2, y_1 + y_2)$\\
$(x_1,y_2) - (x_2,y_2) := (x_1 - x_2$









\clearpage



\subsection{Vectors}
\begin{tikzpicture}
\draw[thin,->] (-5,0) -- (5,0) node[anchor=west] {$x$};
\foreach \x in {-4,-3,-2,-1,0,1,2,3,4}
\draw (\x cm,2pt) -- (\x cm,-2pt);
\foreach \x in {-4,-3,-2,-1,1,2,3,4}
\node[anchor=north] at (\x cm,0) {\footnotesize \x};
\draw[thin,->] (0,-5) -- (0,5) node[anchor=south] {$y$};
\foreach \y in {-4,-3,-2,-1,0,1,2,3,4}
\draw (2pt, \y cm) -- (-2pt, \y cm);
\foreach \y in {-4,-3,-2,-1,1,2,3,4}
\node[anchor=west] at (0, \y cm) {\footnotesize \y};
\node[anchor=north west] at (0,0) {\footnotesize O};
 \draw[thick,->,color=blue] (0,0) -- (2,3); 
 \draw[thick,->,color=blue] (2,3) -- (3,2); ; 
\draw[thick,->,color=blue] (0,0) -- (3,2); ;
\end{tikzpicture}

\newpage

$v=(2,3)$\\
$w=(1,-1)$

\sep
Vector addition :\\
$v+w = (3,2)$



In general:\\
$(x_1,y_2) + (x_2,y_2) := (x_1 + x_2, y_1 + y_2)$\\
$(x_1,y_2) - (x_2,y_2) := (x_1 - x_2, y_1 - y_2)$










\clearpage



\subsection{Vectors}
\begin{tikzpicture}
\draw[thin,->] (-5,0) -- (5,0) node[anchor=west] {$x$};
\foreach \x in {-4,-3,-2,-1,0,1,2,3,4}
\draw (\x cm,2pt) -- (\x cm,-2pt);
\foreach \x in {-4,-3,-2,-1,1,2,3,4}
\node[anchor=north] at (\x cm,0) {\footnotesize \x};
\draw[thin,->] (0,-5) -- (0,5) node[anchor=south] {$y$};
\foreach \y in {-4,-3,-2,-1,0,1,2,3,4}
\draw (2pt, \y cm) -- (-2pt, \y cm);
\foreach \y in {-4,-3,-2,-1,1,2,3,4}
\node[anchor=west] at (0, \y cm) {\footnotesize \y};
\node[anchor=north west] at (0,0) {\footnotesize O};
 \draw[thick,->,color=blue] (0,0) -- (2,3); 
 \draw[thick,->,color=blue] (2,3) -- (3,2); ; 
\draw[thick,->,color=blue] (0,0) -- (3,2); ;
\end{tikzpicture}

\newpage

$v=(2,3)$\\
$w=(1,-1)$

\sep
Vector addition  and subtraction :\\
$v+w = (3,2)$



In general:\\
$(x_1,y_2) + (x_2,y_2) := (x_1 + x_2, y_1 + y_2)$\\
$(x_1,y_2) - (x_2,y_2) := (x_1 - x_2, y_1 - y_2)$




\sep










\clearpage



\subsection{Vectors}
\begin{tikzpicture}
\draw[thin,->] (-5,0) -- (5,0) node[anchor=west] {$x$};
\foreach \x in {-4,-3,-2,-1,0,1,2,3,4}
\draw (\x cm,2pt) -- (\x cm,-2pt);
\foreach \x in {-4,-3,-2,-1,1,2,3,4}
\node[anchor=north] at (\x cm,0) {\footnotesize \x};
\draw[thin,->] (0,-5) -- (0,5) node[anchor=south] {$y$};
\foreach \y in {-4,-3,-2,-1,0,1,2,3,4}
\draw (2pt, \y cm) -- (-2pt, \y cm);
\foreach \y in {-4,-3,-2,-1,1,2,3,4}
\node[anchor=west] at (0, \y cm) {\footnotesize \y};
\node[anchor=north west] at (0,0) {\footnotesize O};
 \draw[thick,->,color=blue] (0,0) -- (2,3); 
 \draw[thick,->,color=blue] (2,3) -- (3,2); ; 
\draw[thick,->,color=blue] (0,0) -- (3,2); ;
\end{tikzpicture}

\newpage

$v=(2,3)$\\
$w=(1,-1)$

\sep
Vector addition  and subtraction :\\
$v+w = (3,2)$



In general:\\
$(x_1,y_2) + (x_2,y_2) := (x_1 + x_2, y_1 + y_2)$\\
$(x_1,y_2) - (x_2,y_2) := (x_1 - x_2, y_1 - y_2)$




\sep
Scalar multiplication:\\










\clearpage



\subsection{Vectors}
\begin{tikzpicture}
\draw[thin,->] (-5,0) -- (5,0) node[anchor=west] {$x$};
\foreach \x in {-4,-3,-2,-1,0,1,2,3,4}
\draw (\x cm,2pt) -- (\x cm,-2pt);
\foreach \x in {-4,-3,-2,-1,1,2,3,4}
\node[anchor=north] at (\x cm,0) {\footnotesize \x};
\draw[thin,->] (0,-5) -- (0,5) node[anchor=south] {$y$};
\foreach \y in {-4,-3,-2,-1,0,1,2,3,4}
\draw (2pt, \y cm) -- (-2pt, \y cm);
\foreach \y in {-4,-3,-2,-1,1,2,3,4}
\node[anchor=west] at (0, \y cm) {\footnotesize \y};
\node[anchor=north west] at (0,0) {\footnotesize O};
 \draw[thick,->,color=blue] (0,0) -- (1,2); ;
\end{tikzpicture}

\newpage

$v=(2,3)$\\
$w=(1,-1)$

\sep
Vector addition  and subtraction :\\
$v+w = (3,2)$



In general:\\
$(x_1,y_2) + (x_2,y_2) := (x_1 + x_2, y_1 + y_2)$\\
$(x_1,y_2) - (x_2,y_2) := (x_1 - x_2, y_1 - y_2)$




\sep
Scalar multiplication:\\
 $v:=(1,2)$\\










\clearpage



\subsection{Vectors}
\begin{tikzpicture}
\draw[thin,->] (-5,0) -- (5,0) node[anchor=west] {$x$};
\foreach \x in {-4,-3,-2,-1,0,1,2,3,4}
\draw (\x cm,2pt) -- (\x cm,-2pt);
\foreach \x in {-4,-3,-2,-1,1,2,3,4}
\node[anchor=north] at (\x cm,0) {\footnotesize \x};
\draw[thin,->] (0,-5) -- (0,5) node[anchor=south] {$y$};
\foreach \y in {-4,-3,-2,-1,0,1,2,3,4}
\draw (2pt, \y cm) -- (-2pt, \y cm);
\foreach \y in {-4,-3,-2,-1,1,2,3,4}
\node[anchor=west] at (0, \y cm) {\footnotesize \y};
\node[anchor=north west] at (0,0) {\footnotesize O};
 \draw[thick,->,color=blue] (0,0) -- (2,4); ;
\end{tikzpicture}

\newpage

$v=(2,3)$\\
$w=(1,-1)$

\sep
Vector addition  and subtraction :\\
$v+w = (3,2)$



In general:\\
$(x_1,y_2) + (x_2,y_2) := (x_1 + x_2, y_1 + y_2)$\\
$(x_1,y_2) - (x_2,y_2) := (x_1 - x_2, y_1 - y_2)$




\sep
Scalar multiplication:\\
 $v:=(1,2)$\\
 $2v$









\clearpage



\subsection{Vectors}
\begin{tikzpicture}
\draw[thin,->] (-5,0) -- (5,0) node[anchor=west] {$x$};
\foreach \x in {-4,-3,-2,-1,0,1,2,3,4}
\draw (\x cm,2pt) -- (\x cm,-2pt);
\foreach \x in {-4,-3,-2,-1,1,2,3,4}
\node[anchor=north] at (\x cm,0) {\footnotesize \x};
\draw[thin,->] (0,-5) -- (0,5) node[anchor=south] {$y$};
\foreach \y in {-4,-3,-2,-1,0,1,2,3,4}
\draw (2pt, \y cm) -- (-2pt, \y cm);
\foreach \y in {-4,-3,-2,-1,1,2,3,4}
\node[anchor=west] at (0, \y cm) {\footnotesize \y};
\node[anchor=north west] at (0,0) {\footnotesize O};
 \draw[thick,->,color=blue] (0,0) -- (2,4); ;
\end{tikzpicture}

\newpage

$v=(2,3)$\\
$w=(1,-1)$

\sep
Vector addition  and subtraction :\\
$v+w = (3,2)$



In general:\\
$(x_1,y_2) + (x_2,y_2) := (x_1 + x_2, y_1 + y_2)$\\
$(x_1,y_2) - (x_2,y_2) := (x_1 - x_2, y_1 - y_2)$




\sep
Scalar multiplication:\\
 $v:=(1,2)$\\
 $2v=2(1,2)$









\clearpage



\subsection{Vectors}
\begin{tikzpicture}
\draw[thin,->] (-5,0) -- (5,0) node[anchor=west] {$x$};
\foreach \x in {-4,-3,-2,-1,0,1,2,3,4}
\draw (\x cm,2pt) -- (\x cm,-2pt);
\foreach \x in {-4,-3,-2,-1,1,2,3,4}
\node[anchor=north] at (\x cm,0) {\footnotesize \x};
\draw[thin,->] (0,-5) -- (0,5) node[anchor=south] {$y$};
\foreach \y in {-4,-3,-2,-1,0,1,2,3,4}
\draw (2pt, \y cm) -- (-2pt, \y cm);
\foreach \y in {-4,-3,-2,-1,1,2,3,4}
\node[anchor=west] at (0, \y cm) {\footnotesize \y};
\node[anchor=north west] at (0,0) {\footnotesize O};
 \draw[thick,->,color=blue] (0,0) -- (2,4); ;
\end{tikzpicture}

\newpage

$v=(2,3)$\\
$w=(1,-1)$

\sep
Vector addition  and subtraction :\\
$v+w = (3,2)$



In general:\\
$(x_1,y_2) + (x_2,y_2) := (x_1 + x_2, y_1 + y_2)$\\
$(x_1,y_2) - (x_2,y_2) := (x_1 - x_2, y_1 - y_2)$




\sep
Scalar multiplication:\\
 $v:=(1,2)$\\
 $2v=2(1,2)=(2,4)$\\










\clearpage



\subsection{Vectors}
\begin{tikzpicture}
\draw[thin,->] (-5,0) -- (5,0) node[anchor=west] {$x$};
\foreach \x in {-4,-3,-2,-1,0,1,2,3,4}
\draw (\x cm,2pt) -- (\x cm,-2pt);
\foreach \x in {-4,-3,-2,-1,1,2,3,4}
\node[anchor=north] at (\x cm,0) {\footnotesize \x};
\draw[thin,->] (0,-5) -- (0,5) node[anchor=south] {$y$};
\foreach \y in {-4,-3,-2,-1,0,1,2,3,4}
\draw (2pt, \y cm) -- (-2pt, \y cm);
\foreach \y in {-4,-3,-2,-1,1,2,3,4}
\node[anchor=west] at (0, \y cm) {\footnotesize \y};
\node[anchor=north west] at (0,0) {\footnotesize O};
 \draw[thick,->,color=blue] (0,0) -- (2,4); ;
\end{tikzpicture}

\newpage

$v=(2,3)$\\
$w=(1,-1)$

\sep
Vector addition  and subtraction :\\
$v+w = (3,2)$



In general:\\
$(x_1,y_2) + (x_2,y_2) := (x_1 + x_2, y_1 + y_2)$\\
$(x_1,y_2) - (x_2,y_2) := (x_1 - x_2, y_1 - y_2)$




\sep
Scalar multiplication:\\
 $v:=(1,2)$\\
 $2v=2(1,2)=(2,4)$\\
In general:\\










\clearpage



\subsection{Vectors}
\begin{tikzpicture}
\draw[thin,->] (-5,0) -- (5,0) node[anchor=west] {$x$};
\foreach \x in {-4,-3,-2,-1,0,1,2,3,4}
\draw (\x cm,2pt) -- (\x cm,-2pt);
\foreach \x in {-4,-3,-2,-1,1,2,3,4}
\node[anchor=north] at (\x cm,0) {\footnotesize \x};
\draw[thin,->] (0,-5) -- (0,5) node[anchor=south] {$y$};
\foreach \y in {-4,-3,-2,-1,0,1,2,3,4}
\draw (2pt, \y cm) -- (-2pt, \y cm);
\foreach \y in {-4,-3,-2,-1,1,2,3,4}
\node[anchor=west] at (0, \y cm) {\footnotesize \y};
\node[anchor=north west] at (0,0) {\footnotesize O};
 \draw[thick,->,color=blue] (0,0) -- (2,4); ;
\end{tikzpicture}

\newpage

$v=(2,3)$\\
$w=(1,-1)$

\sep
Vector addition  and subtraction :\\
$v+w = (3,2)$



In general:\\
$(x_1,y_2) + (x_2,y_2) := (x_1 + x_2, y_1 + y_2)$\\
$(x_1,y_2) - (x_2,y_2) := (x_1 - x_2, y_1 - y_2)$




\sep
Scalar multiplication:\\
 $v:=(1,2)$\\
 $2v=2(1,2)=(2,4)$\\
In general:\\
$\lambda(x,y) $









\clearpage



\subsection{Vectors}
\begin{tikzpicture}
\draw[thin,->] (-5,0) -- (5,0) node[anchor=west] {$x$};
\foreach \x in {-4,-3,-2,-1,0,1,2,3,4}
\draw (\x cm,2pt) -- (\x cm,-2pt);
\foreach \x in {-4,-3,-2,-1,1,2,3,4}
\node[anchor=north] at (\x cm,0) {\footnotesize \x};
\draw[thin,->] (0,-5) -- (0,5) node[anchor=south] {$y$};
\foreach \y in {-4,-3,-2,-1,0,1,2,3,4}
\draw (2pt, \y cm) -- (-2pt, \y cm);
\foreach \y in {-4,-3,-2,-1,1,2,3,4}
\node[anchor=west] at (0, \y cm) {\footnotesize \y};
\node[anchor=north west] at (0,0) {\footnotesize O};
 \draw[thick,->,color=blue] (0,0) -- (2,4); ;
\end{tikzpicture}

\newpage

$v=(2,3)$\\
$w=(1,-1)$

\sep
Vector addition  and subtraction :\\
$v+w = (3,2)$



In general:\\
$(x_1,y_2) + (x_2,y_2) := (x_1 + x_2, y_1 + y_2)$\\
$(x_1,y_2) - (x_2,y_2) := (x_1 - x_2, y_1 - y_2)$




\sep
Scalar multiplication:\\
 $v:=(1,2)$\\
 $2v=2(1,2)=(2,4)$\\
In general:\\
$\lambda(x,y) := (\lambda x, \lambda y)$












\clearpage



$p := (2,3)$,


\clearpage



$p := (2,3)$,\\ $\mathbf{w} := (1,1)$,


\clearpage



$p := (2,3)$,\\ $\mathbf{w} := (1,1)$,\\ $q:=p + \mathbf{w} = (2,3) + (1,1) = (3,4)$ 


\clearpage



$p := (2,3)$,\\ $\mathbf{w} := (1,1)$,\\ $q:=p + \mathbf{w} = (2,3) + (1,1) = (3,4)$ \\ (displacement of $p$ by $\mathbf{w}$).




\clearpage



$p := (2,3)$,\\ $\mathbf{w} := (1,1)$,\\ $q:=p + \mathbf{w} = (2,3) + (1,1) = (3,4)$ \\ (displacement of $p$ by $\mathbf{w}$).

\sep
$p := (2,3)$ and $q=(3,4)$,


\clearpage



$p := (2,3)$,\\ $\mathbf{w} := (1,1)$,\\ $q:=p + \mathbf{w} = (2,3) + (1,1) = (3,4)$ \\ (displacement of $p$ by $\mathbf{w}$).

\sep
$p := (2,3)$ and $q=(3,4)$,\\
$\mathbf{v}= q - p$


\clearpage



$p := (2,3)$,\\ $\mathbf{w} := (1,1)$,\\ $q:=p + \mathbf{w} = (2,3) + (1,1) = (3,4)$ \\ (displacement of $p$ by $\mathbf{w}$).

\sep
$p := (2,3)$ and $q=(3,4)$,\\
$\mathbf{v}= q - p$ is the displacement


\clearpage



$p := (2,3)$,\\ $\mathbf{w} := (1,1)$,\\ $q:=p + \mathbf{w} = (2,3) + (1,1) = (3,4)$ \\ (displacement of $p$ by $\mathbf{w}$).

\sep
$p := (2,3)$ and $q=(3,4)$,\\
$\mathbf{v}= q - p$ is the displacement that takes $p$ to $q$ 



\clearpage



$p := (2,3)$,\\ $\mathbf{w} := (1,1)$,\\ $q:=p + \mathbf{w} = (2,3) + (1,1) = (3,4)$ \\ (displacement of $p$ by $\mathbf{w}$).

\sep
$p := (2,3)$ and $q=(3,4)$,\\
$\mathbf{v}= q - p$ is the displacement that takes $p$ to $q$ 
\sep

$\gamma : (\alpha,\beta) \to \mathbb{R}^2$


\clearpage



$p := (2,3)$,\\ $\mathbf{w} := (1,1)$,\\ $q:=p + \mathbf{w} = (2,3) + (1,1) = (3,4)$ \\ (displacement of $p$ by $\mathbf{w}$).

\sep
$p := (2,3)$ and $q=(3,4)$,\\
$\mathbf{v}= q - p$ is the displacement that takes $p$ to $q$ 
\sep

$\gamma : (\alpha,\beta) \to \mathbb{R}^2$ is a smooth parametrization.


\clearpage



$p := (2,3)$,\\ $\mathbf{w} := (1,1)$,\\ $q:=p + \mathbf{w} = (2,3) + (1,1) = (3,4)$ \\ (displacement of $p$ by $\mathbf{w}$).

\sep
$p := (2,3)$ and $q=(3,4)$,\\
$\mathbf{v}= q - p$ is the displacement that takes $p$ to $q$ 
\sep

$\gamma : (\alpha,\beta) \to \mathbb{R}^2$ is a smooth parametrization.\\
$\gamma(t)$ is the \emph{point} at $t$


\clearpage



$p := (2,3)$,\\ $\mathbf{w} := (1,1)$,\\ $q:=p + \mathbf{w} = (2,3) + (1,1) = (3,4)$ \\ (displacement of $p$ by $\mathbf{w}$).

\sep
$p := (2,3)$ and $q=(3,4)$,\\
$\mathbf{v}= q - p$ is the displacement that takes $p$ to $q$ 
\sep

$\gamma : (\alpha,\beta) \to \mathbb{R}^2$ is a smooth parametrization.\\
$\gamma(t)$ is the \emph{point} at $t$\\
$\gamma(t+h)$ is the \emph{point} at $t+h$


\clearpage



$p := (2,3)$,\\ $\mathbf{w} := (1,1)$,\\ $q:=p + \mathbf{w} = (2,3) + (1,1) = (3,4)$ \\ (displacement of $p$ by $\mathbf{w}$).

\sep
$p := (2,3)$ and $q=(3,4)$,\\
$\mathbf{v}= q - p$ is the displacement that takes $p$ to $q$ 
\sep

$\gamma : (\alpha,\beta) \to \mathbb{R}^2$ is a smooth parametrization.\\
$\gamma(t)$ is the \emph{point} at $t$\\
$\gamma(t+h)$ is the \emph{point} at $t+h$\\
$\gamma(t+h)-\gamma(t)$ is the displacement \emph{vector} at $t+h$


\clearpage



$p := (2,3)$,\\ $\mathbf{w} := (1,1)$,\\ $q:=p + \mathbf{w} = (2,3) + (1,1) = (3,4)$ \\ (displacement of $p$ by $\mathbf{w}$).

\sep
$p := (2,3)$ and $q=(3,4)$,\\
$\mathbf{v}= q - p$ is the displacement that takes $p$ to $q$ 
\sep

$\gamma : (\alpha,\beta) \to \mathbb{R}^2$ is a smooth parametrization.\\
$\gamma(t)$ is the \emph{point} at $t$\\
$\gamma(t+h)$ is the \emph{point} at $t+h$\\
$\gamma(t+h)-\gamma(t)$ is the displacement \emph{vector} at $t+h$\\



\begin{definition}
$\gamma : (\alpha,\beta) \to \mathbb{R}^2$\end{definition}


\clearpage



$p := (2,3)$,\\ $\mathbf{w} := (1,1)$,\\ $q:=p + \mathbf{w} = (2,3) + (1,1) = (3,4)$ \\ (displacement of $p$ by $\mathbf{w}$).

\sep
$p := (2,3)$ and $q=(3,4)$,\\
$\mathbf{v}= q - p$ is the displacement that takes $p$ to $q$ 
\sep

$\gamma : (\alpha,\beta) \to \mathbb{R}^2$ is a smooth parametrization.\\
$\gamma(t)$ is the \emph{point} at $t$\\
$\gamma(t+h)$ is the \emph{point} at $t+h$\\
$\gamma(t+h)-\gamma(t)$ is the displacement \emph{vector} at $t+h$\\



\begin{definition}
$\gamma : (\alpha,\beta) \to \mathbb{R}^2$ is a smooth parametrization.
\[\dot{\gamma}(t) \]\end{definition}


\clearpage



$p := (2,3)$,\\ $\mathbf{w} := (1,1)$,\\ $q:=p + \mathbf{w} = (2,3) + (1,1) = (3,4)$ \\ (displacement of $p$ by $\mathbf{w}$).

\sep
$p := (2,3)$ and $q=(3,4)$,\\
$\mathbf{v}= q - p$ is the displacement that takes $p$ to $q$ 
\sep

$\gamma : (\alpha,\beta) \to \mathbb{R}^2$ is a smooth parametrization.\\
$\gamma(t)$ is the \emph{point} at $t$\\
$\gamma(t+h)$ is the \emph{point} at $t+h$\\
$\gamma(t+h)-\gamma(t)$ is the displacement \emph{vector} at $t+h$\\



\begin{definition}
$\gamma : (\alpha,\beta) \to \mathbb{R}^2$ is a smooth parametrization.
\[\dot{\gamma}(t) = \lim_{h\to 0} \]\end{definition}


\clearpage



$p := (2,3)$,\\ $\mathbf{w} := (1,1)$,\\ $q:=p + \mathbf{w} = (2,3) + (1,1) = (3,4)$ \\ (displacement of $p$ by $\mathbf{w}$).

\sep
$p := (2,3)$ and $q=(3,4)$,\\
$\mathbf{v}= q - p$ is the displacement that takes $p$ to $q$ 
\sep

$\gamma : (\alpha,\beta) \to \mathbb{R}^2$ is a smooth parametrization.\\
$\gamma(t)$ is the \emph{point} at $t$\\
$\gamma(t+h)$ is the \emph{point} at $t+h$\\
$\gamma(t+h)-\gamma(t)$ is the displacement \emph{vector} at $t+h$\\



\begin{definition}
$\gamma : (\alpha,\beta) \to \mathbb{R}^2$ is a smooth parametrization.
\[\dot{\gamma}(t) = \lim_{h\to 0} (1/h)(\gamma(t+h) - \gamma(t))\]
\end{definition}


\clearpage



$p := (2,3)$,\\ $\mathbf{w} := (1,1)$,\\ $q:=p + \mathbf{w} = (2,3) + (1,1) = (3,4)$ \\ (displacement of $p$ by $\mathbf{w}$).

\sep
$p := (2,3)$ and $q=(3,4)$,\\
$\mathbf{v}= q - p$ is the displacement that takes $p$ to $q$ 
\sep

$\gamma : (\alpha,\beta) \to \mathbb{R}^2$ is a smooth parametrization.\\
$\gamma(t)$ is the \emph{point} at $t$\\
$\gamma(t+h)$ is the \emph{point} at $t+h$\\
$\gamma(t+h)-\gamma(t)$ is the displacement \emph{vector} at $t+h$\\



\begin{definition}
$\gamma : (\alpha,\beta) \to \mathbb{R}^2$ is a smooth parametrization.
\[\dot{\gamma}(t) = \lim_{h\to 0} (1/h)(\gamma(t+h) - \gamma(t))\]
is called the velocity vector at $t$ and $\dot{\gamma} : (\alpha, \beta) \to \mathbb{R}^2$ is called the velocity vector field of the parametrization $\gamma$.
\end{definition}



\clearpage



$p := (2,3)$,\\ $\mathbf{w} := (1,1)$,\\ $q:=p + \mathbf{w} = (2,3) + (1,1) = (3,4)$ \\ (displacement of $p$ by $\mathbf{w}$).

\sep
$p := (2,3)$ and $q=(3,4)$,\\
$\mathbf{v}= q - p$ is the displacement that takes $p$ to $q$ 
\sep

$\gamma : (\alpha,\beta) \to \mathbb{R}^2$ is a smooth parametrization.\\
$\gamma(t)$ is the \emph{point} at $t$\\
$\gamma(t+h)$ is the \emph{point} at $t+h$\\
$\gamma(t+h)-\gamma(t)$ is the displacement \emph{vector} at $t+h$\\



\begin{definition}
$\gamma : (\alpha,\beta) \to \mathbb{R}^2$ is a smooth parametrization.
\[\dot{\gamma}(t) = \lim_{h\to 0} (1/h)(\gamma(t+h) - \gamma(t))\]
is called the velocity vector at $t$ and $\dot{\gamma} : (\alpha, \beta) \to \mathbb{R}^2$ is called the velocity vector field of the parametrization $\gamma$.
\end{definition}
\newpage
Points on the straight line passing through $p$,


\clearpage



$p := (2,3)$,\\ $\mathbf{w} := (1,1)$,\\ $q:=p + \mathbf{w} = (2,3) + (1,1) = (3,4)$ \\ (displacement of $p$ by $\mathbf{w}$).

\sep
$p := (2,3)$ and $q=(3,4)$,\\
$\mathbf{v}= q - p$ is the displacement that takes $p$ to $q$ 
\sep

$\gamma : (\alpha,\beta) \to \mathbb{R}^2$ is a smooth parametrization.\\
$\gamma(t)$ is the \emph{point} at $t$\\
$\gamma(t+h)$ is the \emph{point} at $t+h$\\
$\gamma(t+h)-\gamma(t)$ is the displacement \emph{vector} at $t+h$\\



\begin{definition}
$\gamma : (\alpha,\beta) \to \mathbb{R}^2$ is a smooth parametrization.
\[\dot{\gamma}(t) = \lim_{h\to 0} (1/h)(\gamma(t+h) - \gamma(t))\]
is called the velocity vector at $t$ and $\dot{\gamma} : (\alpha, \beta) \to \mathbb{R}^2$ is called the velocity vector field of the parametrization $\gamma$.
\end{definition}
\newpage
Points on the straight line passing through $p$, parallel to $\mathbf{v}$


\clearpage



$p := (2,3)$,\\ $\mathbf{w} := (1,1)$,\\ $q:=p + \mathbf{w} = (2,3) + (1,1) = (3,4)$ \\ (displacement of $p$ by $\mathbf{w}$).

\sep
$p := (2,3)$ and $q=(3,4)$,\\
$\mathbf{v}= q - p$ is the displacement that takes $p$ to $q$ 
\sep

$\gamma : (\alpha,\beta) \to \mathbb{R}^2$ is a smooth parametrization.\\
$\gamma(t)$ is the \emph{point} at $t$\\
$\gamma(t+h)$ is the \emph{point} at $t+h$\\
$\gamma(t+h)-\gamma(t)$ is the displacement \emph{vector} at $t+h$\\



\begin{definition}
$\gamma : (\alpha,\beta) \to \mathbb{R}^2$ is a smooth parametrization.
\[\dot{\gamma}(t) = \lim_{h\to 0} (1/h)(\gamma(t+h) - \gamma(t))\]
is called the velocity vector at $t$ and $\dot{\gamma} : (\alpha, \beta) \to \mathbb{R}^2$ is called the velocity vector field of the parametrization $\gamma$.
\end{definition}
\newpage
Points on the straight line passing through $p$, parallel to $\mathbf{v}\neq 0$:


\clearpage



$p := (2,3)$,\\ $\mathbf{w} := (1,1)$,\\ $q:=p + \mathbf{w} = (2,3) + (1,1) = (3,4)$ \\ (displacement of $p$ by $\mathbf{w}$).

\sep
$p := (2,3)$ and $q=(3,4)$,\\
$\mathbf{v}= q - p$ is the displacement that takes $p$ to $q$ 
\sep

$\gamma : (\alpha,\beta) \to \mathbb{R}^2$ is a smooth parametrization.\\
$\gamma(t)$ is the \emph{point} at $t$\\
$\gamma(t+h)$ is the \emph{point} at $t+h$\\
$\gamma(t+h)-\gamma(t)$ is the displacement \emph{vector} at $t+h$\\



\begin{definition}
$\gamma : (\alpha,\beta) \to \mathbb{R}^2$ is a smooth parametrization.
\[\dot{\gamma}(t) = \lim_{h\to 0} (1/h)(\gamma(t+h) - \gamma(t))\]
is called the velocity vector at $t$ and $\dot{\gamma} : (\alpha, \beta) \to \mathbb{R}^2$ is called the velocity vector field of the parametrization $\gamma$.
\end{definition}
\newpage
Points on the straight line passing through $p$, parallel to $\mathbf{v}\neq 0$:
\[\{q \in \mathbb{R}^2 \ |\ \}\]


\clearpage



$p := (2,3)$,\\ $\mathbf{w} := (1,1)$,\\ $q:=p + \mathbf{w} = (2,3) + (1,1) = (3,4)$ \\ (displacement of $p$ by $\mathbf{w}$).

\sep
$p := (2,3)$ and $q=(3,4)$,\\
$\mathbf{v}= q - p$ is the displacement that takes $p$ to $q$ 
\sep

$\gamma : (\alpha,\beta) \to \mathbb{R}^2$ is a smooth parametrization.\\
$\gamma(t)$ is the \emph{point} at $t$\\
$\gamma(t+h)$ is the \emph{point} at $t+h$\\
$\gamma(t+h)-\gamma(t)$ is the displacement \emph{vector} at $t+h$\\



\begin{definition}
$\gamma : (\alpha,\beta) \to \mathbb{R}^2$ is a smooth parametrization.
\[\dot{\gamma}(t) = \lim_{h\to 0} (1/h)(\gamma(t+h) - \gamma(t))\]
is called the velocity vector at $t$ and $\dot{\gamma} : (\alpha, \beta) \to \mathbb{R}^2$ is called the velocity vector field of the parametrization $\gamma$.
\end{definition}
\newpage
Points on the straight line passing through $p$, parallel to $\mathbf{v}\neq 0$:
\[\{q \in \mathbb{R}^2 \ |\ q = p \}\]


\clearpage



$p := (2,3)$,\\ $\mathbf{w} := (1,1)$,\\ $q:=p + \mathbf{w} = (2,3) + (1,1) = (3,4)$ \\ (displacement of $p$ by $\mathbf{w}$).

\sep
$p := (2,3)$ and $q=(3,4)$,\\
$\mathbf{v}= q - p$ is the displacement that takes $p$ to $q$ 
\sep

$\gamma : (\alpha,\beta) \to \mathbb{R}^2$ is a smooth parametrization.\\
$\gamma(t)$ is the \emph{point} at $t$\\
$\gamma(t+h)$ is the \emph{point} at $t+h$\\
$\gamma(t+h)-\gamma(t)$ is the displacement \emph{vector} at $t+h$\\



\begin{definition}
$\gamma : (\alpha,\beta) \to \mathbb{R}^2$ is a smooth parametrization.
\[\dot{\gamma}(t) = \lim_{h\to 0} (1/h)(\gamma(t+h) - \gamma(t))\]
is called the velocity vector at $t$ and $\dot{\gamma} : (\alpha, \beta) \to \mathbb{R}^2$ is called the velocity vector field of the parametrization $\gamma$.
\end{definition}
\newpage
Points on the straight line passing through $p$, parallel to $\mathbf{v}\neq 0$:
\[\{q \in \mathbb{R}^2 \ |\ q = p  + k\mathbf{v} \}\]


\clearpage



$p := (2,3)$,\\ $\mathbf{w} := (1,1)$,\\ $q:=p + \mathbf{w} = (2,3) + (1,1) = (3,4)$ \\ (displacement of $p$ by $\mathbf{w}$).

\sep
$p := (2,3)$ and $q=(3,4)$,\\
$\mathbf{v}= q - p$ is the displacement that takes $p$ to $q$ 
\sep

$\gamma : (\alpha,\beta) \to \mathbb{R}^2$ is a smooth parametrization.\\
$\gamma(t)$ is the \emph{point} at $t$\\
$\gamma(t+h)$ is the \emph{point} at $t+h$\\
$\gamma(t+h)-\gamma(t)$ is the displacement \emph{vector} at $t+h$\\



\begin{definition}
$\gamma : (\alpha,\beta) \to \mathbb{R}^2$ is a smooth parametrization.
\[\dot{\gamma}(t) = \lim_{h\to 0} (1/h)(\gamma(t+h) - \gamma(t))\]
is called the velocity vector at $t$ and $\dot{\gamma} : (\alpha, \beta) \to \mathbb{R}^2$ is called the velocity vector field of the parametrization $\gamma$.
\end{definition}
\newpage
Points on the straight line passing through $p$, parallel to $\mathbf{v}\neq 0$:
\[\{q \in \mathbb{R}^2 \ |\ q = p  + k\mathbf{v} , k \in \mathbb{R}\}\]


\clearpage



$p := (2,3)$,\\ $\mathbf{w} := (1,1)$,\\ $q:=p + \mathbf{w} = (2,3) + (1,1) = (3,4)$ \\ (displacement of $p$ by $\mathbf{w}$).

\sep
$p := (2,3)$ and $q=(3,4)$,\\
$\mathbf{v}= q - p$ is the displacement that takes $p$ to $q$ 
\sep

$\gamma : (\alpha,\beta) \to \mathbb{R}^2$ is a smooth parametrization.\\
$\gamma(t)$ is the \emph{point} at $t$\\
$\gamma(t+h)$ is the \emph{point} at $t+h$\\
$\gamma(t+h)-\gamma(t)$ is the displacement \emph{vector} at $t+h$\\



\begin{definition}
$\gamma : (\alpha,\beta) \to \mathbb{R}^2$ is a smooth parametrization.
\[\dot{\gamma}(t) = \lim_{h\to 0} (1/h)(\gamma(t+h) - \gamma(t))\]
is called the velocity vector at $t$ and $\dot{\gamma} : (\alpha, \beta) \to \mathbb{R}^2$ is called the velocity vector field of the parametrization $\gamma$.
\end{definition}
\newpage
Points on the straight line passing through $p$, parallel to $\mathbf{v}\neq 0$:
\[\{q \in \mathbb{R}^2 \ |\ q = p  + k\mathbf{v} , k \in \mathbb{R}\}\]

\begin{definition}
  If $\dot{\gamma}(t)\neq 0$, the line tangent to $\gamma$\end{definition}


\clearpage



$p := (2,3)$,\\ $\mathbf{w} := (1,1)$,\\ $q:=p + \mathbf{w} = (2,3) + (1,1) = (3,4)$ \\ (displacement of $p$ by $\mathbf{w}$).

\sep
$p := (2,3)$ and $q=(3,4)$,\\
$\mathbf{v}= q - p$ is the displacement that takes $p$ to $q$ 
\sep

$\gamma : (\alpha,\beta) \to \mathbb{R}^2$ is a smooth parametrization.\\
$\gamma(t)$ is the \emph{point} at $t$\\
$\gamma(t+h)$ is the \emph{point} at $t+h$\\
$\gamma(t+h)-\gamma(t)$ is the displacement \emph{vector} at $t+h$\\



\begin{definition}
$\gamma : (\alpha,\beta) \to \mathbb{R}^2$ is a smooth parametrization.
\[\dot{\gamma}(t) = \lim_{h\to 0} (1/h)(\gamma(t+h) - \gamma(t))\]
is called the velocity vector at $t$ and $\dot{\gamma} : (\alpha, \beta) \to \mathbb{R}^2$ is called the velocity vector field of the parametrization $\gamma$.
\end{definition}
\newpage
Points on the straight line passing through $p$, parallel to $\mathbf{v}\neq 0$:
\[\{q \in \mathbb{R}^2 \ |\ q = p  + k\mathbf{v} , k \in \mathbb{R}\}\]

\begin{definition}
  If $\dot{\gamma}(t)\neq 0$, the line tangent to $\gamma$ at $t$ is,\end{definition}


\clearpage



$p := (2,3)$,\\ $\mathbf{w} := (1,1)$,\\ $q:=p + \mathbf{w} = (2,3) + (1,1) = (3,4)$ \\ (displacement of $p$ by $\mathbf{w}$).

\sep
$p := (2,3)$ and $q=(3,4)$,\\
$\mathbf{v}= q - p$ is the displacement that takes $p$ to $q$ 
\sep

$\gamma : (\alpha,\beta) \to \mathbb{R}^2$ is a smooth parametrization.\\
$\gamma(t)$ is the \emph{point} at $t$\\
$\gamma(t+h)$ is the \emph{point} at $t+h$\\
$\gamma(t+h)-\gamma(t)$ is the displacement \emph{vector} at $t+h$\\



\begin{definition}
$\gamma : (\alpha,\beta) \to \mathbb{R}^2$ is a smooth parametrization.
\[\dot{\gamma}(t) = \lim_{h\to 0} (1/h)(\gamma(t+h) - \gamma(t))\]
is called the velocity vector at $t$ and $\dot{\gamma} : (\alpha, \beta) \to \mathbb{R}^2$ is called the velocity vector field of the parametrization $\gamma$.
\end{definition}
\newpage
Points on the straight line passing through $p$, parallel to $\mathbf{v}\neq 0$:
\[\{q \in \mathbb{R}^2 \ |\ q = p  + k\mathbf{v} , k \in \mathbb{R}\}\]

\begin{definition}
  If $\dot{\gamma}(t)\neq 0$, the line tangent to $\gamma$ at $t$ is,
  \[T_\gamma(t):=\{q \in \mathbb{R}^2\}\]\end{definition}


\clearpage



$p := (2,3)$,\\ $\mathbf{w} := (1,1)$,\\ $q:=p + \mathbf{w} = (2,3) + (1,1) = (3,4)$ \\ (displacement of $p$ by $\mathbf{w}$).

\sep
$p := (2,3)$ and $q=(3,4)$,\\
$\mathbf{v}= q - p$ is the displacement that takes $p$ to $q$ 
\sep

$\gamma : (\alpha,\beta) \to \mathbb{R}^2$ is a smooth parametrization.\\
$\gamma(t)$ is the \emph{point} at $t$\\
$\gamma(t+h)$ is the \emph{point} at $t+h$\\
$\gamma(t+h)-\gamma(t)$ is the displacement \emph{vector} at $t+h$\\



\begin{definition}
$\gamma : (\alpha,\beta) \to \mathbb{R}^2$ is a smooth parametrization.
\[\dot{\gamma}(t) = \lim_{h\to 0} (1/h)(\gamma(t+h) - \gamma(t))\]
is called the velocity vector at $t$ and $\dot{\gamma} : (\alpha, \beta) \to \mathbb{R}^2$ is called the velocity vector field of the parametrization $\gamma$.
\end{definition}
\newpage
Points on the straight line passing through $p$, parallel to $\mathbf{v}\neq 0$:
\[\{q \in \mathbb{R}^2 \ |\ q = p  + k\mathbf{v} , k \in \mathbb{R}\}\]

\begin{definition}
  If $\dot{\gamma}(t)\neq 0$, the line tangent to $\gamma$ at $t$ is,
  \[T_\gamma(t):=\{q \in \mathbb{R}^2 \ |\ q = \gamma(t)\}\]\end{definition}


\clearpage



$p := (2,3)$,\\ $\mathbf{w} := (1,1)$,\\ $q:=p + \mathbf{w} = (2,3) + (1,1) = (3,4)$ \\ (displacement of $p$ by $\mathbf{w}$).

\sep
$p := (2,3)$ and $q=(3,4)$,\\
$\mathbf{v}= q - p$ is the displacement that takes $p$ to $q$ 
\sep

$\gamma : (\alpha,\beta) \to \mathbb{R}^2$ is a smooth parametrization.\\
$\gamma(t)$ is the \emph{point} at $t$\\
$\gamma(t+h)$ is the \emph{point} at $t+h$\\
$\gamma(t+h)-\gamma(t)$ is the displacement \emph{vector} at $t+h$\\



\begin{definition}
$\gamma : (\alpha,\beta) \to \mathbb{R}^2$ is a smooth parametrization.
\[\dot{\gamma}(t) = \lim_{h\to 0} (1/h)(\gamma(t+h) - \gamma(t))\]
is called the velocity vector at $t$ and $\dot{\gamma} : (\alpha, \beta) \to \mathbb{R}^2$ is called the velocity vector field of the parametrization $\gamma$.
\end{definition}
\newpage
Points on the straight line passing through $p$, parallel to $\mathbf{v}\neq 0$:
\[\{q \in \mathbb{R}^2 \ |\ q = p  + k\mathbf{v} , k \in \mathbb{R}\}\]

\begin{definition}
  If $\dot{\gamma}(t)\neq 0$, the line tangent to $\gamma$ at $t$ is,
  \[T_\gamma(t):=\{q \in \mathbb{R}^2 \ |\ q = \gamma(t) + k\dot{\gamma}(t)\}\]\end{definition}


\clearpage



$p := (2,3)$,\\ $\mathbf{w} := (1,1)$,\\ $q:=p + \mathbf{w} = (2,3) + (1,1) = (3,4)$ \\ (displacement of $p$ by $\mathbf{w}$).

\sep
$p := (2,3)$ and $q=(3,4)$,\\
$\mathbf{v}= q - p$ is the displacement that takes $p$ to $q$ 
\sep

$\gamma : (\alpha,\beta) \to \mathbb{R}^2$ is a smooth parametrization.\\
$\gamma(t)$ is the \emph{point} at $t$\\
$\gamma(t+h)$ is the \emph{point} at $t+h$\\
$\gamma(t+h)-\gamma(t)$ is the displacement \emph{vector} at $t+h$\\



\begin{definition}
$\gamma : (\alpha,\beta) \to \mathbb{R}^2$ is a smooth parametrization.
\[\dot{\gamma}(t) = \lim_{h\to 0} (1/h)(\gamma(t+h) - \gamma(t))\]
is called the velocity vector at $t$ and $\dot{\gamma} : (\alpha, \beta) \to \mathbb{R}^2$ is called the velocity vector field of the parametrization $\gamma$.
\end{definition}
\newpage
Points on the straight line passing through $p$, parallel to $\mathbf{v}\neq 0$:
\[\{q \in \mathbb{R}^2 \ |\ q = p  + k\mathbf{v} , k \in \mathbb{R}\}\]

\begin{definition}
  If $\dot{\gamma}(t)\neq 0$, the line tangent to $\gamma$ at $t$ is,
  \[T_\gamma(t):=\{q \in \mathbb{R}^2 \ |\ q = \gamma(t) + k\dot{\gamma}(t) \}\]\end{definition}


\clearpage



$p := (2,3)$,\\ $\mathbf{w} := (1,1)$,\\ $q:=p + \mathbf{w} = (2,3) + (1,1) = (3,4)$ \\ (displacement of $p$ by $\mathbf{w}$).

\sep
$p := (2,3)$ and $q=(3,4)$,\\
$\mathbf{v}= q - p$ is the displacement that takes $p$ to $q$ 
\sep

$\gamma : (\alpha,\beta) \to \mathbb{R}^2$ is a smooth parametrization.\\
$\gamma(t)$ is the \emph{point} at $t$\\
$\gamma(t+h)$ is the \emph{point} at $t+h$\\
$\gamma(t+h)-\gamma(t)$ is the displacement \emph{vector} at $t+h$\\



\begin{definition}
$\gamma : (\alpha,\beta) \to \mathbb{R}^2$ is a smooth parametrization.
\[\dot{\gamma}(t) = \lim_{h\to 0} (1/h)(\gamma(t+h) - \gamma(t))\]
is called the velocity vector at $t$ and $\dot{\gamma} : (\alpha, \beta) \to \mathbb{R}^2$ is called the velocity vector field of the parametrization $\gamma$.
\end{definition}
\newpage
Points on the straight line passing through $p$, parallel to $\mathbf{v}\neq 0$:
\[\{q \in \mathbb{R}^2 \ |\ q = p  + k\mathbf{v} , k \in \mathbb{R}\}\]

\begin{definition}
  If $\dot{\gamma}(t)\neq 0$, the line tangent to $\gamma$ at $t$ is,
  \[T_\gamma(t):=\{q \in \mathbb{R}^2 \ |\ q = \gamma(t) + k\dot{\gamma}(t) , k\in \mathbb{R}\}\]
\end{definition}

\begin{definition}
  A smooth parametrized curve, $\gamma : (\alpha,\beta) \to \mathbb{R}^2$,\end{definition}


\clearpage



$p := (2,3)$,\\ $\mathbf{w} := (1,1)$,\\ $q:=p + \mathbf{w} = (2,3) + (1,1) = (3,4)$ \\ (displacement of $p$ by $\mathbf{w}$).

\sep
$p := (2,3)$ and $q=(3,4)$,\\
$\mathbf{v}= q - p$ is the displacement that takes $p$ to $q$ 
\sep

$\gamma : (\alpha,\beta) \to \mathbb{R}^2$ is a smooth parametrization.\\
$\gamma(t)$ is the \emph{point} at $t$\\
$\gamma(t+h)$ is the \emph{point} at $t+h$\\
$\gamma(t+h)-\gamma(t)$ is the displacement \emph{vector} at $t+h$\\



\begin{definition}
$\gamma : (\alpha,\beta) \to \mathbb{R}^2$ is a smooth parametrization.
\[\dot{\gamma}(t) = \lim_{h\to 0} (1/h)(\gamma(t+h) - \gamma(t))\]
is called the velocity vector at $t$ and $\dot{\gamma} : (\alpha, \beta) \to \mathbb{R}^2$ is called the velocity vector field of the parametrization $\gamma$.
\end{definition}
\newpage
Points on the straight line passing through $p$, parallel to $\mathbf{v}\neq 0$:
\[\{q \in \mathbb{R}^2 \ |\ q = p  + k\mathbf{v} , k \in \mathbb{R}\}\]

\begin{definition}
  If $\dot{\gamma}(t)\neq 0$, the line tangent to $\gamma$ at $t$ is,
  \[T_\gamma(t):=\{q \in \mathbb{R}^2 \ |\ q = \gamma(t) + k\dot{\gamma}(t) , k\in \mathbb{R}\}\]
\end{definition}

\begin{definition}
  A smooth parametrized curve, $\gamma : (\alpha,\beta) \to \mathbb{R}^2$, is called a \textbf{regular parametrized curve}\end{definition}


\clearpage



$p := (2,3)$,\\ $\mathbf{w} := (1,1)$,\\ $q:=p + \mathbf{w} = (2,3) + (1,1) = (3,4)$ \\ (displacement of $p$ by $\mathbf{w}$).

\sep
$p := (2,3)$ and $q=(3,4)$,\\
$\mathbf{v}= q - p$ is the displacement that takes $p$ to $q$ 
\sep

$\gamma : (\alpha,\beta) \to \mathbb{R}^2$ is a smooth parametrization.\\
$\gamma(t)$ is the \emph{point} at $t$\\
$\gamma(t+h)$ is the \emph{point} at $t+h$\\
$\gamma(t+h)-\gamma(t)$ is the displacement \emph{vector} at $t+h$\\



\begin{definition}
$\gamma : (\alpha,\beta) \to \mathbb{R}^2$ is a smooth parametrization.
\[\dot{\gamma}(t) = \lim_{h\to 0} (1/h)(\gamma(t+h) - \gamma(t))\]
is called the velocity vector at $t$ and $\dot{\gamma} : (\alpha, \beta) \to \mathbb{R}^2$ is called the velocity vector field of the parametrization $\gamma$.
\end{definition}
\newpage
Points on the straight line passing through $p$, parallel to $\mathbf{v}\neq 0$:
\[\{q \in \mathbb{R}^2 \ |\ q = p  + k\mathbf{v} , k \in \mathbb{R}\}\]

\begin{definition}
  If $\dot{\gamma}(t)\neq 0$, the line tangent to $\gamma$ at $t$ is,
  \[T_\gamma(t):=\{q \in \mathbb{R}^2 \ |\ q = \gamma(t) + k\dot{\gamma}(t) , k\in \mathbb{R}\}\]
\end{definition}

\begin{definition}
  A smooth parametrized curve, $\gamma : (\alpha,\beta) \to \mathbb{R}^2$, is called a \textbf{regular parametrized curve} if $\dot{\gamma}(t) \neq 0$ for each  $t \in (\alpha,\beta)$.
\end{definition}

\textbf{From now on, we will assume all parametrized curves to be regular}



\clearpage



\begin{lemma}
If $\tilde{\gamma}(t) = \gamma(\phi(t))$ is a reparametrization,\end{lemma}


\clearpage



\begin{lemma}
If $\tilde{\gamma}(t) = \gamma(\phi(t))$ is a reparametrization, then $\dot{\tilde{\gamma}}(t) = \dot{\gamma}(\phi(t))\phi'(t)$\end{lemma}


\clearpage



\begin{lemma}
If $\tilde{\gamma}(t) = \gamma(\phi(t))$ is a reparametrization, then $\dot{\tilde{\gamma}}(t) = \dot{\gamma}(\phi(t))\phi'(t)$
\end{lemma}
\begin{proof}
\[\tilde{\gamma}(t) = \gamma(\phi(t))\]\end{proof}


\clearpage



\begin{lemma}
If $\tilde{\gamma}(t) = \gamma(\phi(t))$ is a reparametrization, then $\dot{\tilde{\gamma}}(t) = \dot{\gamma}(\phi(t))\phi'(t)$
\end{lemma}
\begin{proof}
\[\tilde{\gamma}(t) = \gamma(\phi(t))\]
\[\tilde{\gamma}(t) = (f_1(\phi(t)), f_2(\phi(t)))\]\end{proof}


\clearpage



\begin{lemma}
If $\tilde{\gamma}(t) = \gamma(\phi(t))$ is a reparametrization, then $\dot{\tilde{\gamma}}(t) = \dot{\gamma}(\phi(t))\phi'(t)$
\end{lemma}
\begin{proof}
\[\tilde{\gamma}(t) = \gamma(\phi(t))\]
\[\tilde{\gamma}(t) = (f_1(\phi(t)), f_2(\phi(t)))\]
\[\dot{\tilde{\gamma}}(t) = (f_1'(\phi(t))\phi'(t), f_2'(\phi(t))\phi'(t))\]\end{proof}


\clearpage



\begin{lemma}
If $\tilde{\gamma}(t) = \gamma(\phi(t))$ is a reparametrization, then $\dot{\tilde{\gamma}}(t) = \dot{\gamma}(\phi(t))\phi'(t)$
\end{lemma}
\begin{proof}
\[\tilde{\gamma}(t) = \gamma(\phi(t))\]
\[\tilde{\gamma}(t) = (f_1(\phi(t)), f_2(\phi(t)))\]
\[\dot{\tilde{\gamma}}(t) = (f_1'(\phi(t))\phi'(t), f_2'(\phi(t))\phi'(t))\]
\[\dot{\tilde{\gamma}}(t) = (f_1'(\phi(t)), f_2'(\phi(t)))\phi'(t)\]\end{proof}


\clearpage



\begin{lemma}
If $\tilde{\gamma}(t) = \gamma(\phi(t))$ is a reparametrization, then $\dot{\tilde{\gamma}}(t) = \dot{\gamma}(\phi(t))\phi'(t)$
\end{lemma}
\begin{proof}
\[\tilde{\gamma}(t) = \gamma(\phi(t))\]
\[\tilde{\gamma}(t) = (f_1(\phi(t)), f_2(\phi(t)))\]
\[\dot{\tilde{\gamma}}(t) = (f_1'(\phi(t))\phi'(t), f_2'(\phi(t))\phi'(t))\]
\[\dot{\tilde{\gamma}}(t) = (f_1'(\phi(t)), f_2'(\phi(t)))\phi'(t)\]
\[\dot{\tilde{\gamma}}(t) = \dot{\gamma}(\phi(t))\phi'(t)\]\end{proof}


\clearpage



\begin{lemma}
If $\tilde{\gamma}(t) = \gamma(\phi(t))$ is a reparametrization, then $\dot{\tilde{\gamma}}(t) = \dot{\gamma}(\phi(t))\phi'(t)$
\end{lemma}
\begin{proof}
\[\tilde{\gamma}(t) = \gamma(\phi(t))\]
\[\tilde{\gamma}(t) = (f_1(\phi(t)), f_2(\phi(t)))\]
\[\dot{\tilde{\gamma}}(t) = (f_1'(\phi(t))\phi'(t), f_2'(\phi(t))\phi'(t))\]
\[\dot{\tilde{\gamma}}(t) = (f_1'(\phi(t)), f_2'(\phi(t)))\phi'(t)\]
\[\dot{\tilde{\gamma}}(t) = \dot{\gamma}(\phi(t))\phi'(t)\]
\end{proof}

\begin{corollary}
The tangent line is invariant under a reparametrization, $\phi(t)$.
\end{corollary}
\begin{proof}
\begin{eqnarray*}
\{\gamma(t) + k\dot{\tilde{\gamma}}(t) \}\end{eqnarray*}\end{proof}


\clearpage



\begin{lemma}
If $\tilde{\gamma}(t) = \gamma(\phi(t))$ is a reparametrization, then $\dot{\tilde{\gamma}}(t) = \dot{\gamma}(\phi(t))\phi'(t)$
\end{lemma}
\begin{proof}
\[\tilde{\gamma}(t) = \gamma(\phi(t))\]
\[\tilde{\gamma}(t) = (f_1(\phi(t)), f_2(\phi(t)))\]
\[\dot{\tilde{\gamma}}(t) = (f_1'(\phi(t))\phi'(t), f_2'(\phi(t))\phi'(t))\]
\[\dot{\tilde{\gamma}}(t) = (f_1'(\phi(t)), f_2'(\phi(t)))\phi'(t)\]
\[\dot{\tilde{\gamma}}(t) = \dot{\gamma}(\phi(t))\phi'(t)\]
\end{proof}

\begin{corollary}
The tangent line is invariant under a reparametrization, $\phi(t)$.
\end{corollary}
\begin{proof}
\begin{eqnarray*}
\{\gamma(t) + k\dot{\tilde{\gamma}}(t) \ |\ k\in \mathbb{R}\} &=& \{\gamma(t) + k\dot{\gamma}(\phi(t))\phi'(t) \}\end{eqnarray*}\end{proof}


\clearpage



\begin{lemma}
If $\tilde{\gamma}(t) = \gamma(\phi(t))$ is a reparametrization, then $\dot{\tilde{\gamma}}(t) = \dot{\gamma}(\phi(t))\phi'(t)$
\end{lemma}
\begin{proof}
\[\tilde{\gamma}(t) = \gamma(\phi(t))\]
\[\tilde{\gamma}(t) = (f_1(\phi(t)), f_2(\phi(t)))\]
\[\dot{\tilde{\gamma}}(t) = (f_1'(\phi(t))\phi'(t), f_2'(\phi(t))\phi'(t))\]
\[\dot{\tilde{\gamma}}(t) = (f_1'(\phi(t)), f_2'(\phi(t)))\phi'(t)\]
\[\dot{\tilde{\gamma}}(t) = \dot{\gamma}(\phi(t))\phi'(t)\]
\end{proof}

\begin{corollary}
The tangent line is invariant under a reparametrization, $\phi(t)$.
\end{corollary}
\begin{proof}
\begin{eqnarray*}
\{\gamma(t) + k\dot{\tilde{\gamma}}(t) \ |\ k\in \mathbb{R}\} &=& \{\gamma(t) + k\dot{\gamma}(\phi(t))\phi'(t) \ |\ k\in \mathbb{R}\} \end{eqnarray*}\end{proof}


\clearpage



\begin{lemma}
If $\tilde{\gamma}(t) = \gamma(\phi(t))$ is a reparametrization, then $\dot{\tilde{\gamma}}(t) = \dot{\gamma}(\phi(t))\phi'(t)$
\end{lemma}
\begin{proof}
\[\tilde{\gamma}(t) = \gamma(\phi(t))\]
\[\tilde{\gamma}(t) = (f_1(\phi(t)), f_2(\phi(t)))\]
\[\dot{\tilde{\gamma}}(t) = (f_1'(\phi(t))\phi'(t), f_2'(\phi(t))\phi'(t))\]
\[\dot{\tilde{\gamma}}(t) = (f_1'(\phi(t)), f_2'(\phi(t)))\phi'(t)\]
\[\dot{\tilde{\gamma}}(t) = \dot{\gamma}(\phi(t))\phi'(t)\]
\end{proof}

\begin{corollary}
The tangent line is invariant under a reparametrization, $\phi(t)$.
\end{corollary}
\begin{proof}
\begin{eqnarray*}
\{\gamma(t) + k\dot{\tilde{\gamma}}(t) \ |\ k\in \mathbb{R}\} &=& \{\gamma(t) + k\dot{\gamma}(\phi(t))\phi'(t) \ |\ k\in \mathbb{R}\} \\ \end{eqnarray*}\end{proof}


\clearpage



\begin{lemma}
If $\tilde{\gamma}(t) = \gamma(\phi(t))$ is a reparametrization, then $\dot{\tilde{\gamma}}(t) = \dot{\gamma}(\phi(t))\phi'(t)$
\end{lemma}
\begin{proof}
\[\tilde{\gamma}(t) = \gamma(\phi(t))\]
\[\tilde{\gamma}(t) = (f_1(\phi(t)), f_2(\phi(t)))\]
\[\dot{\tilde{\gamma}}(t) = (f_1'(\phi(t))\phi'(t), f_2'(\phi(t))\phi'(t))\]
\[\dot{\tilde{\gamma}}(t) = (f_1'(\phi(t)), f_2'(\phi(t)))\phi'(t)\]
\[\dot{\tilde{\gamma}}(t) = \dot{\gamma}(\phi(t))\phi'(t)\]
\end{proof}

\begin{corollary}
The tangent line is invariant under a reparametrization, $\phi(t)$,  if $\phi'(t)\neq 0$ 
\end{corollary}
\begin{proof}
\begin{eqnarray*}
\{\gamma(t) + k\dot{\tilde{\gamma}}(t) \ |\ k\in \mathbb{R}\} &=& \{\gamma(t) + k\dot{\gamma}(\phi(t))\phi'(t) \ |\ k\in \mathbb{R}\} \\ 
                                                  &=& \{\gamma(t) + k\dot{\gamma}(\phi(t)) \}\end{eqnarray*}\end{proof}


\clearpage



\begin{lemma}
If $\tilde{\gamma}(t) = \gamma(\phi(t))$ is a reparametrization, then $\dot{\tilde{\gamma}}(t) = \dot{\gamma}(\phi(t))\phi'(t)$
\end{lemma}
\begin{proof}
\[\tilde{\gamma}(t) = \gamma(\phi(t))\]
\[\tilde{\gamma}(t) = (f_1(\phi(t)), f_2(\phi(t)))\]
\[\dot{\tilde{\gamma}}(t) = (f_1'(\phi(t))\phi'(t), f_2'(\phi(t))\phi'(t))\]
\[\dot{\tilde{\gamma}}(t) = (f_1'(\phi(t)), f_2'(\phi(t)))\phi'(t)\]
\[\dot{\tilde{\gamma}}(t) = \dot{\gamma}(\phi(t))\phi'(t)\]
\end{proof}

\begin{corollary}
The tangent line is invariant under a reparametrization, $\phi(t)$,  if $\phi'(t)\neq 0$ 
\end{corollary}
\begin{proof}
\begin{eqnarray*}
\{\gamma(t) + k\dot{\tilde{\gamma}}(t) \ |\ k\in \mathbb{R}\} &=& \{\gamma(t) + k\dot{\gamma}(\phi(t))\phi'(t) \ |\ k\in \mathbb{R}\} \\ 
                                                  &=& \{\gamma(t) + k\dot{\gamma}(\phi(t)) \ |\ k\in \mathbb{R}\}
\end{eqnarray*}
\end{proof}
\newpage



\clearpage



\begin{lemma}
If $\tilde{\gamma}(t) = \gamma(\phi(t))$ is a reparametrization, then $\dot{\tilde{\gamma}}(t) = \dot{\gamma}(\phi(t))\phi'(t)$
\end{lemma}
\begin{proof}
\[\tilde{\gamma}(t) = \gamma(\phi(t))\]
\[\tilde{\gamma}(t) = (f_1(\phi(t)), f_2(\phi(t)))\]
\[\dot{\tilde{\gamma}}(t) = (f_1'(\phi(t))\phi'(t), f_2'(\phi(t))\phi'(t))\]
\[\dot{\tilde{\gamma}}(t) = (f_1'(\phi(t)), f_2'(\phi(t)))\phi'(t)\]
\[\dot{\tilde{\gamma}}(t) = \dot{\gamma}(\phi(t))\phi'(t)\]
\end{proof}

\begin{corollary}
The tangent line is invariant under a reparametrization, $\phi(t)$,  if $\phi'(t)\neq 0$ 
\end{corollary}
\begin{proof}
\begin{eqnarray*}
\{\gamma(t) + k\dot{\tilde{\gamma}}(t) \ |\ k\in \mathbb{R}\} &=& \{\gamma(t) + k\dot{\gamma}(\phi(t))\phi'(t) \ |\ k\in \mathbb{R}\} \\ 
                                                  &=& \{\gamma(t) + k\dot{\gamma}(\phi(t)) \ |\ k\in \mathbb{R}\}
\end{eqnarray*}
\end{proof}
\newpage
Note: $\tilde{\gamma}(t)$ is the same point, $p$, as $\gamma(\phi(t))$


\clearpage



\begin{lemma}
If $\tilde{\gamma}(t) = \gamma(\phi(t))$ is a reparametrization, then $\dot{\tilde{\gamma}}(t) = \dot{\gamma}(\phi(t))\phi'(t)$
\end{lemma}
\begin{proof}
\[\tilde{\gamma}(t) = \gamma(\phi(t))\]
\[\tilde{\gamma}(t) = (f_1(\phi(t)), f_2(\phi(t)))\]
\[\dot{\tilde{\gamma}}(t) = (f_1'(\phi(t))\phi'(t), f_2'(\phi(t))\phi'(t))\]
\[\dot{\tilde{\gamma}}(t) = (f_1'(\phi(t)), f_2'(\phi(t)))\phi'(t)\]
\[\dot{\tilde{\gamma}}(t) = \dot{\gamma}(\phi(t))\phi'(t)\]
\end{proof}

\begin{corollary}
The tangent line is invariant under a reparametrization, $\phi(t)$,  if $\phi'(t)\neq 0$ 
\end{corollary}
\begin{proof}
\begin{eqnarray*}
\{\gamma(t) + k\dot{\tilde{\gamma}}(t) \ |\ k\in \mathbb{R}\} &=& \{\gamma(t) + k\dot{\gamma}(\phi(t))\phi'(t) \ |\ k\in \mathbb{R}\} \\ 
                                                  &=& \{\gamma(t) + k\dot{\gamma}(\phi(t)) \ |\ k\in \mathbb{R}\}
\end{eqnarray*}
\end{proof}
\newpage
Note: $\tilde{\gamma}(t)$ is the same point, $p$, as $\gamma(\phi(t))$\\
When using $\tilde{\gamma}$, the point $p$ ``appears at time $t$''


\clearpage



\begin{lemma}
If $\tilde{\gamma}(t) = \gamma(\phi(t))$ is a reparametrization, then $\dot{\tilde{\gamma}}(t) = \dot{\gamma}(\phi(t))\phi'(t)$
\end{lemma}
\begin{proof}
\[\tilde{\gamma}(t) = \gamma(\phi(t))\]
\[\tilde{\gamma}(t) = (f_1(\phi(t)), f_2(\phi(t)))\]
\[\dot{\tilde{\gamma}}(t) = (f_1'(\phi(t))\phi'(t), f_2'(\phi(t))\phi'(t))\]
\[\dot{\tilde{\gamma}}(t) = (f_1'(\phi(t)), f_2'(\phi(t)))\phi'(t)\]
\[\dot{\tilde{\gamma}}(t) = \dot{\gamma}(\phi(t))\phi'(t)\]
\end{proof}

\begin{corollary}
The tangent line is invariant under a reparametrization, $\phi(t)$,  if $\phi'(t)\neq 0$ 
\end{corollary}
\begin{proof}
\begin{eqnarray*}
\{\gamma(t) + k\dot{\tilde{\gamma}}(t) \ |\ k\in \mathbb{R}\} &=& \{\gamma(t) + k\dot{\gamma}(\phi(t))\phi'(t) \ |\ k\in \mathbb{R}\} \\ 
                                                  &=& \{\gamma(t) + k\dot{\gamma}(\phi(t)) \ |\ k\in \mathbb{R}\}
\end{eqnarray*}
\end{proof}
\newpage
Note: $\tilde{\gamma}(t)$ is the same point, $p$, as $\gamma(\phi(t))$\\
When using $\tilde{\gamma}$, the point $p$ ``appears at time $t$''\\
When using $\gamma$, the point $p$ ``appears at time $\phi(t)$''


\clearpage



\begin{lemma}
If $\tilde{\gamma}(t) = \gamma(\phi(t))$ is a reparametrization, then $\dot{\tilde{\gamma}}(t) = \dot{\gamma}(\phi(t))\phi'(t)$
\end{lemma}
\begin{proof}
\[\tilde{\gamma}(t) = \gamma(\phi(t))\]
\[\tilde{\gamma}(t) = (f_1(\phi(t)), f_2(\phi(t)))\]
\[\dot{\tilde{\gamma}}(t) = (f_1'(\phi(t))\phi'(t), f_2'(\phi(t))\phi'(t))\]
\[\dot{\tilde{\gamma}}(t) = (f_1'(\phi(t)), f_2'(\phi(t)))\phi'(t)\]
\[\dot{\tilde{\gamma}}(t) = \dot{\gamma}(\phi(t))\phi'(t)\]
\end{proof}

\begin{corollary}
The tangent line is invariant under a reparametrization, $\phi(t)$,  if $\phi'(t)\neq 0$ 
\end{corollary}
\begin{proof}
\begin{eqnarray*}
\{\gamma(t) + k\dot{\tilde{\gamma}}(t) \ |\ k\in \mathbb{R}\} &=& \{\gamma(t) + k\dot{\gamma}(\phi(t))\phi'(t) \ |\ k\in \mathbb{R}\} \\ 
                                                  &=& \{\gamma(t) + k\dot{\gamma}(\phi(t)) \ |\ k\in \mathbb{R}\}
\end{eqnarray*}
\end{proof}
\newpage
Note: $\tilde{\gamma}(t)$ is the same point, $p$, as $\gamma(\phi(t))$\\
When using $\tilde{\gamma}$, the point $p$ ``appears at time $t$''\\
When using $\gamma$, the point $p$ ``appears at time $\phi(t)$''\\
So, $\dot{\tilde{\gamma}}(t)$ and $\dot{\gamma}(\phi(t))$ are velocity vectors at the same point $p$





\end{document}