\documentclass[a4paper]{article}
\usepackage{amssymb}
\usepackage{amsthm}
\usepackage{amsmath}
\setcounter{secnumdepth}{0}
\newtheorem{theorem}{Theorem}
\title{Exercise sheet 3\\ {\small Theory of Computation, IDC204} }
\date{}
\begin{document}
\maketitle
\begin{enumerate}
\item If $L_1$ and $L_2$ are both regular languages over an alphabet $\Sigma$. If $L_1$ can be recognized by a finite state automaton $A_1$ and $L_2$ can be recognized by a finite state automaton $A_2$, then show that one can use $A_1$ and $A_2$ to design a *non-deterministic* finite state automaton that recognizes,
  \begin{enumerate}
   \item $L_1 \cup L_2$
   \item $L_1 \circ L_2 = \{s_1s_2 \ |\ s_1 \in L_1, s_2 \in L_2\}$, i.e., it is obtained by concatenating each string in $L_1$ with a string in $L_2$.
   \item $L_1^* = \{s_1s_2\ldots s_n \ |\ s_i \in L_1\}$.
  \end{enumerate}

  
\item Design a *non*-deterministic finite state automaton over the alphabet $\Sigma={0,1}$ that will accept a string if and only if it
  \begin{enumerate}
    \item is the empty string
    \item is not the empty string
    \item is precisely the string 111
\item begins with the substring 111
\item is the empty string or begins with the substring 111
\item ends with the substring 111
\item begins with 0 and ends with 1
\item contains the substring 111 
  \end{enumerate}
\end{enumerate}
\end{document}