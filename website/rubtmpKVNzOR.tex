\documentclass[twocolumn,20pt,fleqn]{extarticle}
\usepackage{color}
\usepackage{geometry}
\usepackage[normalem]{ulem}
\usepackage{cancel}
\usepackage{dingbat}
\usepackage{marginnote}
\usepackage{amsmath}
\usepackage{amsthm}
\usepackage{amssymb}
\usepackage{amsfonts}
\usepackage{mathtools}
\usepackage{tcolorbox}
\usepackage{multicol}
\usepackage{tikz}
\usepackage{tikz-cd}
\usepackage{wrapfig}
\usepackage{caption}
%\usepackage{draftwatermark}

\usetikzlibrary{calc}

\setlength{\parindent}{0pt}
\setlength{\parskip}{0.5em}
\makeatletter
\renewcommand{\@seccntformat}[1]{}
\makeatother

\geometry{paperwidth=34cm, paperheight=19cm, margin=0.5cm}

\newcommand{\clr}[2]{{\color{#1} #2}}
\newcommand{\alert}[1]{{\color{red} #1}}
\newcommand{\ovrbrc}[2]{\overbrace{#2}^{#1}}
\newcommand{\undrbrc}[2]{\underbrace{#2}_{#1}}
\newcommand{\lnspc}{\vspace{1cm}}
\newcommand{\sep}{\vspace{0.5cm}}
\newcommand{\e}{\textrm{e}}
%\newcommand{\dydx}[2]{\frac{\mathrm{d}}}

\theoremstyle{plain}
\newtheorem*{theorem}{Theorem}
\newtheorem*{proposition}{Proposition}
\newtheorem*{lemma}{Lemma}
\newtheorem*{corollary}{Corollary}

\theoremstyle{definition}
\newtheorem*{exercise}{Exercise}
\newtheorem*{definition}{Definition}
\newtheorem*{example}{Example}
\newtheorem*{exmpls}{Examples}


\theoremstyle{remark}
\newtheorem*{remark}{Remark}
\newtheorem*{note}{Note}

\newenvironment*{examples}{\begin{exmpls} ~ \begin{enumerate}}{\end{enumerate}\end{exmpls}}

\begin{document}



\clearpage




\begin{definition}
  A  ``parametrized plane curve'' \end{definition}


\clearpage




\begin{definition}
  A  ``parametrized plane curve''  is a  function,\\ $\gamma $\end{definition}


\clearpage




\begin{definition}
  A  ``parametrized plane curve''  is a  function,\\ $\gamma  : (\alpha, \beta) $\end{definition}


\clearpage




\begin{definition}
  A  ``parametrized plane curve''  is a  function,\\ $\gamma  : (\alpha, \beta) \to \mathbb{R}^2$.
\end{definition}




\clearpage




\begin{definition}
  A  ``parametrized plane curve''  is a  function,\\ $\gamma  : (\alpha, \beta) \to \mathbb{R}^2$.
\end{definition}

Explicitly,\\
$\gamma(t) = (f_1(t), f_2(t))$, for planes


\clearpage




\begin{definition}
  A  ``parametrized plane curve''  is a  function,\\ $\gamma  : (\alpha, \beta) \to \mathbb{R}^2$.
\end{definition}

Explicitly,\\
$\gamma(t) = (f_1(t), f_2(t))$, for planes\\

Set of points on the curve:\\ 


\clearpage




\begin{definition}
  A  ``parametrized plane curve''  is a  function,\\ $\gamma  : (\alpha, \beta) \to \mathbb{R}^2$.
\end{definition}

Explicitly,\\
$\gamma(t) = (f_1(t), f_2(t))$, for planes\\

Set of points on the curve:\\  $\textrm{ Image } \gamma $


\clearpage




\begin{definition}
  A  ``parametrized plane curve''  is a  function,\\ $\gamma  : (\alpha, \beta) \to \mathbb{R}^2$.
\end{definition}

Explicitly,\\
$\gamma(t) = (f_1(t), f_2(t))$, for planes\\

Set of points on the curve:\\  $\textrm{ Image } \gamma = \{(x,y) \in \mathbb{R}^2\}$


\clearpage




\begin{definition}
  A  ``parametrized plane curve''  is a  function,\\ $\gamma  : (\alpha, \beta) \to \mathbb{R}^2$.
\end{definition}

Explicitly,\\
$\gamma(t) = (f_1(t), f_2(t))$, for planes\\

Set of points on the curve:\\  $\textrm{ Image } \gamma = \{(x,y) \in \mathbb{R}^2 \ |\ (x,y) = \gamma(t),\}$


\clearpage




\begin{definition}
  A  ``parametrized plane curve''  is a  function,\\ $\gamma  : (\alpha, \beta) \to \mathbb{R}^2$.
\end{definition}

Explicitly,\\
$\gamma(t) = (f_1(t), f_2(t))$, for planes\\

Set of points on the curve:\\  $\textrm{ Image } \gamma = \{(x,y) \in \mathbb{R}^2 \ |\ (x,y) = \gamma(t), t \in \mathbb{R}\}$







\clearpage




\begin{definition}
  A  ``parametrized plane curve''  is a  function,\\ $\gamma  : (\alpha, \beta) \to \mathbb{R}^2$.
\end{definition}

Explicitly,\\
$\gamma(t) = (f_1(t), f_2(t))$, for planes\\

Set of points on the curve:\\  $\textrm{ Image } \gamma = \{(x,y) \in \mathbb{R}^2 \ |\ (x,y) = \gamma(t), t \in \mathbb{R}\}$




\begin{examples}
  \item $L:=\{(x,y) \in \mathbb{R}^2\ |\ 4y = 7x + 3\}$\\
  \end{examples}


\clearpage




\begin{definition}
  A  ``parametrized plane curve''  is a  function,\\ $\gamma  : (\alpha, \beta) \to \mathbb{R}^2$.
\end{definition}

Explicitly,\\
$\gamma(t) = (f_1(t), f_2(t))$, for planes\\

Set of points on the curve:\\  $\textrm{ Image } \gamma = \{(x,y) \in \mathbb{R}^2 \ |\ (x,y) = \gamma(t), t \in \mathbb{R}\}$




\begin{examples}
  \item $L:=\{(x,y) \in \mathbb{R}^2\ |\ 4y = 7x + 3\}$\\
  $\gamma $\end{examples}


\clearpage




\begin{definition}
  A  ``parametrized plane curve''  is a  function,\\ $\gamma  : (\alpha, \beta) \to \mathbb{R}^2$.
\end{definition}

Explicitly,\\
$\gamma(t) = (f_1(t), f_2(t))$, for planes\\

Set of points on the curve:\\  $\textrm{ Image } \gamma = \{(x,y) \in \mathbb{R}^2 \ |\ (x,y) = \gamma(t), t \in \mathbb{R}\}$




\begin{examples}
  \item $L:=\{(x,y) \in \mathbb{R}^2\ |\ 4y = 7x + 3\}$\\
  $\gamma  : (-\infty,\infty) $\end{examples}


\clearpage




\begin{definition}
  A  ``parametrized plane curve''  is a  function,\\ $\gamma  : (\alpha, \beta) \to \mathbb{R}^2$.
\end{definition}

Explicitly,\\
$\gamma(t) = (f_1(t), f_2(t))$, for planes\\

Set of points on the curve:\\  $\textrm{ Image } \gamma = \{(x,y) \in \mathbb{R}^2 \ |\ (x,y) = \gamma(t), t \in \mathbb{R}\}$




\begin{examples}
  \item $L:=\{(x,y) \in \mathbb{R}^2\ |\ 4y = 7x + 3\}$\\
  $\gamma  : (-\infty,\infty) \to \mathbb{R}^2$ \end{examples}


\clearpage




\begin{definition}
  A  ``parametrized plane curve''  is a  function,\\ $\gamma  : (\alpha, \beta) \to \mathbb{R}^2$.
\end{definition}

Explicitly,\\
$\gamma(t) = (f_1(t), f_2(t))$, for planes\\

Set of points on the curve:\\  $\textrm{ Image } \gamma = \{(x,y) \in \mathbb{R}^2 \ |\ (x,y) = \gamma(t), t \in \mathbb{R}\}$




\begin{examples}
  \item $L:=\{(x,y) \in \mathbb{R}^2\ |\ 4y = 7x + 3\}$\\
  $\gamma  : (-\infty,\infty) \to \mathbb{R}^2$ \\
  $\gamma(t) = (t, \frac{7t+3}{4}) $\end{examples}


\clearpage




\begin{definition}
  A  ``parametrized plane curve''  is a  function,\\ $\gamma  : (\alpha, \beta) \to \mathbb{R}^2$.
\end{definition}

Explicitly,\\
$\gamma(t) = (f_1(t), f_2(t))$, for planes\\

Set of points on the curve:\\  $\textrm{ Image } \gamma = \{(x,y) \in \mathbb{R}^2 \ |\ (x,y) = \gamma(t), t \in \mathbb{R}\}$




\begin{examples}
  \item $L:=\{(x,y) \in \mathbb{R}^2\ |\ 4y = 7x + 3\}$\\
  $\gamma  : (-\infty,\infty) \to \mathbb{R}^2$ \\
  $\gamma(t) = (t, \frac{7t+3}{4})  \in L$\\\end{examples}


\clearpage




\begin{definition}
  A  ``parametrized plane curve''  is a  function,\\ $\gamma  : (\alpha, \beta) \to \mathbb{R}^2$.
\end{definition}

Explicitly,\\
$\gamma(t) = (f_1(t), f_2(t))$, for planes\\

Set of points on the curve:\\  $\textrm{ Image } \gamma = \{(x,y) \in \mathbb{R}^2 \ |\ (x,y) = \gamma(t), t \in \mathbb{R}\}$




\begin{examples}
  \item $L:=\{(x,y) \in \mathbb{R}^2\ |\ 4y = 7x + 3\}$\\
  $\gamma  : (-\infty,\infty) \to \mathbb{R}^2$ \\
  $\gamma(t) = (t, \frac{7t+3}{4})  \in L$\\
  \item $P:=\{(x,y) \in \mathbb{R}^2\ |\ y^2 = x\}$\\
  \end{examples}


\clearpage




\begin{definition}
  A  ``parametrized plane curve''  is a  function,\\ $\gamma  : (\alpha, \beta) \to \mathbb{R}^2$.
\end{definition}

Explicitly,\\
$\gamma(t) = (f_1(t), f_2(t))$, for planes\\

Set of points on the curve:\\  $\textrm{ Image } \gamma = \{(x,y) \in \mathbb{R}^2 \ |\ (x,y) = \gamma(t), t \in \mathbb{R}\}$




\begin{examples}
  \item $L:=\{(x,y) \in \mathbb{R}^2\ |\ 4y = 7x + 3\}$\\
  $\gamma  : (-\infty,\infty) \to \mathbb{R}^2$ \\
  $\gamma(t) = (t, \frac{7t+3}{4})  \in L$\\
  \item $P:=\{(x,y) \in \mathbb{R}^2\ |\ y^2 = x\}$\\
  $\gamma $\end{examples}


\clearpage




\begin{definition}
  A  ``parametrized plane curve''  is a  function,\\ $\gamma  : (\alpha, \beta) \to \mathbb{R}^2$.
\end{definition}

Explicitly,\\
$\gamma(t) = (f_1(t), f_2(t))$, for planes\\

Set of points on the curve:\\  $\textrm{ Image } \gamma = \{(x,y) \in \mathbb{R}^2 \ |\ (x,y) = \gamma(t), t \in \mathbb{R}\}$




\begin{examples}
  \item $L:=\{(x,y) \in \mathbb{R}^2\ |\ 4y = 7x + 3\}$\\
  $\gamma  : (-\infty,\infty) \to \mathbb{R}^2$ \\
  $\gamma(t) = (t, \frac{7t+3}{4})  \in L$\\
  \item $P:=\{(x,y) \in \mathbb{R}^2\ |\ y^2 = x\}$\\
  $\gamma  : (-\infty,\infty) $\end{examples}


\clearpage




\begin{definition}
  A  ``parametrized plane curve''  is a  function,\\ $\gamma  : (\alpha, \beta) \to \mathbb{R}^2$.
\end{definition}

Explicitly,\\
$\gamma(t) = (f_1(t), f_2(t))$, for planes\\

Set of points on the curve:\\  $\textrm{ Image } \gamma = \{(x,y) \in \mathbb{R}^2 \ |\ (x,y) = \gamma(t), t \in \mathbb{R}\}$




\begin{examples}
  \item $L:=\{(x,y) \in \mathbb{R}^2\ |\ 4y = 7x + 3\}$\\
  $\gamma  : (-\infty,\infty) \to \mathbb{R}^2$ \\
  $\gamma(t) = (t, \frac{7t+3}{4})  \in L$\\
  \item $P:=\{(x,y) \in \mathbb{R}^2\ |\ y^2 = x\}$\\
  $\gamma  : (-\infty,\infty) \to \mathbb{R}^2$ \end{examples}


\clearpage




\begin{definition}
  A  ``parametrized plane curve''  is a  function,\\ $\gamma  : (\alpha, \beta) \to \mathbb{R}^2$.
\end{definition}

Explicitly,\\
$\gamma(t) = (f_1(t), f_2(t))$, for planes\\

Set of points on the curve:\\  $\textrm{ Image } \gamma = \{(x,y) \in \mathbb{R}^2 \ |\ (x,y) = \gamma(t), t \in \mathbb{R}\}$




\begin{examples}
  \item $L:=\{(x,y) \in \mathbb{R}^2\ |\ 4y = 7x + 3\}$\\
  $\gamma  : (-\infty,\infty) \to \mathbb{R}^2$ \\
  $\gamma(t) = (t, \frac{7t+3}{4})  \in L$\\
  \item $P:=\{(x,y) \in \mathbb{R}^2\ |\ y^2 = x\}$\\
  $\gamma  : (-\infty,\infty) \to \mathbb{R}^2$ \\
  $\gamma(t) = (t^2, t) $\end{examples}


\clearpage




\begin{definition}
  A  ``parametrized plane curve''  is a  function,\\ $\gamma  : (\alpha, \beta) \to \mathbb{R}^2$.
\end{definition}

Explicitly,\\
$\gamma(t) = (f_1(t), f_2(t))$, for planes\\

Set of points on the curve:\\  $\textrm{ Image } \gamma = \{(x,y) \in \mathbb{R}^2 \ |\ (x,y) = \gamma(t), t \in \mathbb{R}\}$




\begin{examples}
  \item $L:=\{(x,y) \in \mathbb{R}^2\ |\ 4y = 7x + 3\}$\\
  $\gamma  : (-\infty,\infty) \to \mathbb{R}^2$ \\
  $\gamma(t) = (t, \frac{7t+3}{4})  \in L$\\
  \item $P:=\{(x,y) \in \mathbb{R}^2\ |\ y^2 = x\}$\\
  $\gamma  : (-\infty,\infty) \to \mathbb{R}^2$ \\
  $\gamma(t) = (t^2, t)  \in P$\\\end{examples}


\clearpage




\begin{definition}
  A  ``parametrized plane curve''  is a  function,\\ $\gamma  : (\alpha, \beta) \to \mathbb{R}^2$.
\end{definition}

Explicitly,\\
$\gamma(t) = (f_1(t), f_2(t))$, for planes\\

Set of points on the curve:\\  $\textrm{ Image } \gamma = \{(x,y) \in \mathbb{R}^2 \ |\ (x,y) = \gamma(t), t \in \mathbb{R}\}$




\begin{examples}
  \item $L:=\{(x,y) \in \mathbb{R}^2\ |\ 4y = 7x + 3\}$\\
  $\gamma  : (-\infty,\infty) \to \mathbb{R}^2$ \\
  $\gamma(t) = (t, \frac{7t+3}{4})  \in L$\\
  \item $P:=\{(x,y) \in \mathbb{R}^2\ |\ y^2 = x\}$\\
  $\gamma  : (-\infty,\infty) \to \mathbb{R}^2$ \\
  $\gamma(t) = (t^2, t)  \in P$\\
\end{examples}




\clearpage



\subsection{Parametrizing a circle}


\begin{tikzpicture}
\draw[thin,->] (-5,0) -- (5,0) node[anchor=west] {$x$};
\foreach \x in {-4,-3,-2,-1,0,1,2,3,4}
\draw (\x cm,2pt) -- (\x cm,-2pt);
\end{tikzpicture}


\clearpage



\subsection{Parametrizing a circle}


\begin{tikzpicture}
\draw[thin,->] (-5,0) -- (5,0) node[anchor=west] {$x$};
\foreach \x in {-4,-3,-2,-1,0,1,2,3,4}
\draw (\x cm,2pt) -- (\x cm,-2pt);
\foreach \x in {-4,-3,-2,-1,1,2,3,4}
\node[anchor=north] at (\x cm,0) {\footnotesize \x};
\draw[thin,->] (0,-5) -- (0,5) node[anchor=south] {$y$};
\foreach \y in {-4,-3,-2,-1,0,1,2,3,4}
\draw (2pt, \y cm) -- (-2pt, \y cm);
\end{tikzpicture}


\clearpage



\subsection{Parametrizing a circle}


\begin{tikzpicture}
\draw[thin,->] (-5,0) -- (5,0) node[anchor=west] {$x$};
\foreach \x in {-4,-3,-2,-1,0,1,2,3,4}
\draw (\x cm,2pt) -- (\x cm,-2pt);
\foreach \x in {-4,-3,-2,-1,1,2,3,4}
\node[anchor=north] at (\x cm,0) {\footnotesize \x};
\draw[thin,->] (0,-5) -- (0,5) node[anchor=south] {$y$};
\foreach \y in {-4,-3,-2,-1,0,1,2,3,4}
\draw (2pt, \y cm) -- (-2pt, \y cm);
\foreach \y in {-4,-3,-2,-1,1,2,3,4}
\node[anchor=west] at (0, \y cm) {\footnotesize \y};
\end{tikzpicture}


\clearpage



\subsection{Parametrizing a circle}


\begin{tikzpicture}
\draw[thin,->] (-5,0) -- (5,0) node[anchor=west] {$x$};
\foreach \x in {-4,-3,-2,-1,0,1,2,3,4}
\draw (\x cm,2pt) -- (\x cm,-2pt);
\foreach \x in {-4,-3,-2,-1,1,2,3,4}
\node[anchor=north] at (\x cm,0) {\footnotesize \x};
\draw[thin,->] (0,-5) -- (0,5) node[anchor=south] {$y$};
\foreach \y in {-4,-3,-2,-1,0,1,2,3,4}
\draw (2pt, \y cm) -- (-2pt, \y cm);
\foreach \y in {-4,-3,-2,-1,1,2,3,4}
\node[anchor=west] at (0, \y cm) {\footnotesize \y};
\node[anchor=north west] at (0,0) {\footnotesize O};
;
\end{tikzpicture}
 





\clearpage



\subsection{Parametrizing a circle}


\begin{tikzpicture}
\draw[thin,->] (-5,0) -- (5,0) node[anchor=west] {$x$};
\foreach \x in {-4,-3,-2,-1,0,1,2,3,4}
\draw (\x cm,2pt) -- (\x cm,-2pt);
\foreach \x in {-4,-3,-2,-1,1,2,3,4}
\node[anchor=north] at (\x cm,0) {\footnotesize \x};
\draw[thin,->] (0,-5) -- (0,5) node[anchor=south] {$y$};
\foreach \y in {-4,-3,-2,-1,0,1,2,3,4}
\draw (2pt, \y cm) -- (-2pt, \y cm);
\foreach \y in {-4,-3,-2,-1,1,2,3,4}
\node[anchor=west] at (0, \y cm) {\footnotesize \y};
\node[anchor=north west] at (0,0) {\footnotesize O};

\draw[color=blue,thick] (2,0) arc
    [
        start angle=0,
        end angle=10,
        x radius=2cm,
        y radius =2cm
    ] ;
;
\end{tikzpicture}
 



\newpage
 
$\gamma : (0, \pi/18) \to \mathbb{R}^2$\\




\clearpage



\subsection{Parametrizing a circle}


\begin{tikzpicture}
\draw[thin,->] (-5,0) -- (5,0) node[anchor=west] {$x$};
\foreach \x in {-4,-3,-2,-1,0,1,2,3,4}
\draw (\x cm,2pt) -- (\x cm,-2pt);
\foreach \x in {-4,-3,-2,-1,1,2,3,4}
\node[anchor=north] at (\x cm,0) {\footnotesize \x};
\draw[thin,->] (0,-5) -- (0,5) node[anchor=south] {$y$};
\foreach \y in {-4,-3,-2,-1,0,1,2,3,4}
\draw (2pt, \y cm) -- (-2pt, \y cm);
\foreach \y in {-4,-3,-2,-1,1,2,3,4}
\node[anchor=west] at (0, \y cm) {\footnotesize \y};
\node[anchor=north west] at (0,0) {\footnotesize O};

\draw[color=blue,thick] (2,0) arc
    [
        start angle=0,
        end angle=10,
        x radius=2cm,
        y radius =2cm
    ] ;
;
\end{tikzpicture}
 



\newpage
 
$\gamma : (0, \pi/18) \to \mathbb{R}^2$\\
$\gamma(t) := (2\cos(t), 2\sin(t))$




\clearpage



\subsection{Parametrizing a circle}


\begin{tikzpicture}
\draw[thin,->] (-5,0) -- (5,0) node[anchor=west] {$x$};
\foreach \x in {-4,-3,-2,-1,0,1,2,3,4}
\draw (\x cm,2pt) -- (\x cm,-2pt);
\foreach \x in {-4,-3,-2,-1,1,2,3,4}
\node[anchor=north] at (\x cm,0) {\footnotesize \x};
\draw[thin,->] (0,-5) -- (0,5) node[anchor=south] {$y$};
\foreach \y in {-4,-3,-2,-1,0,1,2,3,4}
\draw (2pt, \y cm) -- (-2pt, \y cm);
\foreach \y in {-4,-3,-2,-1,1,2,3,4}
\node[anchor=west] at (0, \y cm) {\footnotesize \y};
\node[anchor=north west] at (0,0) {\footnotesize O};

\draw[color=blue,thick] (2,0) arc
    [
        start angle=0,
        end angle= 20 ,
        x radius=2cm,
        y radius =2cm
    ] ;
;
\end{tikzpicture}
 



\newpage
 
$\gamma : (0,  2\pi/18 ) \to \mathbb{R}^2$\\
$\gamma(t) := (2\cos(t), 2\sin(t))$
  




\clearpage



\subsection{Parametrizing a circle}


\begin{tikzpicture}
\draw[thin,->] (-5,0) -- (5,0) node[anchor=west] {$x$};
\foreach \x in {-4,-3,-2,-1,0,1,2,3,4}
\draw (\x cm,2pt) -- (\x cm,-2pt);
\foreach \x in {-4,-3,-2,-1,1,2,3,4}
\node[anchor=north] at (\x cm,0) {\footnotesize \x};
\draw[thin,->] (0,-5) -- (0,5) node[anchor=south] {$y$};
\foreach \y in {-4,-3,-2,-1,0,1,2,3,4}
\draw (2pt, \y cm) -- (-2pt, \y cm);
\foreach \y in {-4,-3,-2,-1,1,2,3,4}
\node[anchor=west] at (0, \y cm) {\footnotesize \y};
\node[anchor=north west] at (0,0) {\footnotesize O};

\draw[color=blue,thick] (2,0) arc
    [
        start angle=0,
        end angle= 30 ,
        x radius=2cm,
        y radius =2cm
    ] ;
;
\end{tikzpicture}
 



\newpage
 
$\gamma : (0,  3\pi/18 ) \to \mathbb{R}^2$\\
$\gamma(t) := (2\cos(t), 2\sin(t))$
  
  




\clearpage



\subsection{Parametrizing a circle}


\begin{tikzpicture}
\draw[thin,->] (-5,0) -- (5,0) node[anchor=west] {$x$};
\foreach \x in {-4,-3,-2,-1,0,1,2,3,4}
\draw (\x cm,2pt) -- (\x cm,-2pt);
\foreach \x in {-4,-3,-2,-1,1,2,3,4}
\node[anchor=north] at (\x cm,0) {\footnotesize \x};
\draw[thin,->] (0,-5) -- (0,5) node[anchor=south] {$y$};
\foreach \y in {-4,-3,-2,-1,0,1,2,3,4}
\draw (2pt, \y cm) -- (-2pt, \y cm);
\foreach \y in {-4,-3,-2,-1,1,2,3,4}
\node[anchor=west] at (0, \y cm) {\footnotesize \y};
\node[anchor=north west] at (0,0) {\footnotesize O};

\draw[color=blue,thick] (2,0) arc
    [
        start angle=0,
        end angle= 60 ,
        x radius=2cm,
        y radius =2cm
    ] ;
;
\end{tikzpicture}
 



\newpage
 
$\gamma : (0,  6\pi/18 ) \to \mathbb{R}^2$\\
$\gamma(t) := (2\cos(t), 2\sin(t))$
  
  
  




\clearpage



\subsection{Parametrizing a circle}


\begin{tikzpicture}
\draw[thin,->] (-5,0) -- (5,0) node[anchor=west] {$x$};
\foreach \x in {-4,-3,-2,-1,0,1,2,3,4}
\draw (\x cm,2pt) -- (\x cm,-2pt);
\foreach \x in {-4,-3,-2,-1,1,2,3,4}
\node[anchor=north] at (\x cm,0) {\footnotesize \x};
\draw[thin,->] (0,-5) -- (0,5) node[anchor=south] {$y$};
\foreach \y in {-4,-3,-2,-1,0,1,2,3,4}
\draw (2pt, \y cm) -- (-2pt, \y cm);
\foreach \y in {-4,-3,-2,-1,1,2,3,4}
\node[anchor=west] at (0, \y cm) {\footnotesize \y};
\node[anchor=north west] at (0,0) {\footnotesize O};

\draw[color=blue,thick] (2,0) arc
    [
        start angle=0,
        end angle= 90 ,
        x radius=2cm,
        y radius =2cm
    ] ;
;
\end{tikzpicture}
 



\newpage
 
$\gamma : (0,  9\pi/18 ) \to \mathbb{R}^2$\\
$\gamma(t) := (2\cos(t), 2\sin(t))$
  
  
  
  




\clearpage



\subsection{Parametrizing a circle}


\begin{tikzpicture}
\draw[thin,->] (-5,0) -- (5,0) node[anchor=west] {$x$};
\foreach \x in {-4,-3,-2,-1,0,1,2,3,4}
\draw (\x cm,2pt) -- (\x cm,-2pt);
\foreach \x in {-4,-3,-2,-1,1,2,3,4}
\node[anchor=north] at (\x cm,0) {\footnotesize \x};
\draw[thin,->] (0,-5) -- (0,5) node[anchor=south] {$y$};
\foreach \y in {-4,-3,-2,-1,0,1,2,3,4}
\draw (2pt, \y cm) -- (-2pt, \y cm);
\foreach \y in {-4,-3,-2,-1,1,2,3,4}
\node[anchor=west] at (0, \y cm) {\footnotesize \y};
\node[anchor=north west] at (0,0) {\footnotesize O};

\draw[color=blue,thick] (2,0) arc
    [
        start angle=0,
        end angle= 120 ,
        x radius=2cm,
        y radius =2cm
    ] ;
;
\end{tikzpicture}
 



\newpage
 
$\gamma : (0,  12\pi/18 ) \to \mathbb{R}^2$\\
$\gamma(t) := (2\cos(t), 2\sin(t))$
  
  
  
  
  




\clearpage



\subsection{Parametrizing a circle}


\begin{tikzpicture}
\draw[thin,->] (-5,0) -- (5,0) node[anchor=west] {$x$};
\foreach \x in {-4,-3,-2,-1,0,1,2,3,4}
\draw (\x cm,2pt) -- (\x cm,-2pt);
\foreach \x in {-4,-3,-2,-1,1,2,3,4}
\node[anchor=north] at (\x cm,0) {\footnotesize \x};
\draw[thin,->] (0,-5) -- (0,5) node[anchor=south] {$y$};
\foreach \y in {-4,-3,-2,-1,0,1,2,3,4}
\draw (2pt, \y cm) -- (-2pt, \y cm);
\foreach \y in {-4,-3,-2,-1,1,2,3,4}
\node[anchor=west] at (0, \y cm) {\footnotesize \y};
\node[anchor=north west] at (0,0) {\footnotesize O};

\draw[color=blue,thick] (2,0) arc
    [
        start angle=0,
        end angle= 180 ,
        x radius=2cm,
        y radius =2cm
    ] ;
;
\end{tikzpicture}
 



\newpage
 
$\gamma : (0,  18\pi/18 ) \to \mathbb{R}^2$\\
$\gamma(t) := (2\cos(t), 2\sin(t))$
  
  
  
  
  
  




\clearpage



\subsection{Parametrizing a circle}


\begin{tikzpicture}
\draw[thin,->] (-5,0) -- (5,0) node[anchor=west] {$x$};
\foreach \x in {-4,-3,-2,-1,0,1,2,3,4}
\draw (\x cm,2pt) -- (\x cm,-2pt);
\foreach \x in {-4,-3,-2,-1,1,2,3,4}
\node[anchor=north] at (\x cm,0) {\footnotesize \x};
\draw[thin,->] (0,-5) -- (0,5) node[anchor=south] {$y$};
\foreach \y in {-4,-3,-2,-1,0,1,2,3,4}
\draw (2pt, \y cm) -- (-2pt, \y cm);
\foreach \y in {-4,-3,-2,-1,1,2,3,4}
\node[anchor=west] at (0, \y cm) {\footnotesize \y};
\node[anchor=north west] at (0,0) {\footnotesize O};

\draw[color=blue,thick] (2,0) arc
    [
        start angle=0,
        end angle= 240 ,
        x radius=2cm,
        y radius =2cm
    ] ;
;
\end{tikzpicture}
 



\newpage
 
$\gamma : (0,  24\pi/18 ) \to \mathbb{R}^2$\\
$\gamma(t) := (2\cos(t), 2\sin(t))$
  
  
  
  
  
  
  




\clearpage



\subsection{Parametrizing a circle}


\begin{tikzpicture}
\draw[thin,->] (-5,0) -- (5,0) node[anchor=west] {$x$};
\foreach \x in {-4,-3,-2,-1,0,1,2,3,4}
\draw (\x cm,2pt) -- (\x cm,-2pt);
\foreach \x in {-4,-3,-2,-1,1,2,3,4}
\node[anchor=north] at (\x cm,0) {\footnotesize \x};
\draw[thin,->] (0,-5) -- (0,5) node[anchor=south] {$y$};
\foreach \y in {-4,-3,-2,-1,0,1,2,3,4}
\draw (2pt, \y cm) -- (-2pt, \y cm);
\foreach \y in {-4,-3,-2,-1,1,2,3,4}
\node[anchor=west] at (0, \y cm) {\footnotesize \y};
\node[anchor=north west] at (0,0) {\footnotesize O};

\draw[color=blue,thick] (2,0) arc
    [
        start angle=0,
        end angle= 270 ,
        x radius=2cm,
        y radius =2cm
    ] ;
;
\end{tikzpicture}
 



\newpage
 
$\gamma : (0,  27\pi/18 ) \to \mathbb{R}^2$\\
$\gamma(t) := (2\cos(t), 2\sin(t))$
  
  
  
  
  
  
  
  






\clearpage



\subsection{Parametrizing a  line }


\begin{tikzpicture}
\draw[thin,->] (-5,0) -- (5,0) node[anchor=west] {$x$};
\foreach \x in {-4,-3,-2,-1,0,1,2,3,4}
\draw (\x cm,2pt) -- (\x cm,-2pt);
\foreach \x in {-4,-3,-2,-1,1,2,3,4}
\node[anchor=north] at (\x cm,0) {\footnotesize \x};
\draw[thin,->] (0,-5) -- (0,5) node[anchor=south] {$y$};
\foreach \y in {-4,-3,-2,-1,0,1,2,3,4}
\draw (2pt, \y cm) -- (-2pt, \y cm);
\foreach \y in {-4,-3,-2,-1,1,2,3,4}
\node[anchor=west] at (0, \y cm) {\footnotesize \y};
\node[anchor=north west] at (0,0) {\footnotesize O};
  ;
\end{tikzpicture}
 



\newpage
 
$\gamma : (0,  27\pi/18 ) \to \mathbb{R}^2$\\
$\gamma(t) := (2\cos(t), 2\sin(t))$
  
  
  
  
  
  
  
  










\clearpage



\subsection{Parametrizing a  line }


\begin{tikzpicture}
\draw[thin,->] (-5,0) -- (5,0) node[anchor=west] {$x$};
\foreach \x in {-4,-3,-2,-1,0,1,2,3,4}
\draw (\x cm,2pt) -- (\x cm,-2pt);
\foreach \x in {-4,-3,-2,-1,1,2,3,4}
\node[anchor=north] at (\x cm,0) {\footnotesize \x};
\draw[thin,->] (0,-5) -- (0,5) node[anchor=south] {$y$};
\foreach \y in {-4,-3,-2,-1,0,1,2,3,4}
\draw (2pt, \y cm) -- (-2pt, \y cm);
\foreach \y in {-4,-3,-2,-1,1,2,3,4}
\node[anchor=west] at (0, \y cm) {\footnotesize \y};
\node[anchor=north west] at (0,0) {\footnotesize O};

\draw[color=blue,thick] (-5,-5) -- (-4,-4);
;
\end{tikzpicture}
 



\newpage
 
$\gamma : (-5, -4) \to \mathbb{R}^2$\\











\clearpage



\subsection{Parametrizing a  line }


\begin{tikzpicture}
\draw[thin,->] (-5,0) -- (5,0) node[anchor=west] {$x$};
\foreach \x in {-4,-3,-2,-1,0,1,2,3,4}
\draw (\x cm,2pt) -- (\x cm,-2pt);
\foreach \x in {-4,-3,-2,-1,1,2,3,4}
\node[anchor=north] at (\x cm,0) {\footnotesize \x};
\draw[thin,->] (0,-5) -- (0,5) node[anchor=south] {$y$};
\foreach \y in {-4,-3,-2,-1,0,1,2,3,4}
\draw (2pt, \y cm) -- (-2pt, \y cm);
\foreach \y in {-4,-3,-2,-1,1,2,3,4}
\node[anchor=west] at (0, \y cm) {\footnotesize \y};
\node[anchor=north west] at (0,0) {\footnotesize O};

\draw[color=blue,thick] (-5,-5) -- (-4,-4);
;
\end{tikzpicture}
 



\newpage
 
$\gamma : (-5, -4) \to \mathbb{R}^2$\\
$\gamma(t) := (t,t)$











\clearpage



\subsection{Parametrizing a  line }


\begin{tikzpicture}
\draw[thin,->] (-5,0) -- (5,0) node[anchor=west] {$x$};
\foreach \x in {-4,-3,-2,-1,0,1,2,3,4}
\draw (\x cm,2pt) -- (\x cm,-2pt);
\foreach \x in {-4,-3,-2,-1,1,2,3,4}
\node[anchor=north] at (\x cm,0) {\footnotesize \x};
\draw[thin,->] (0,-5) -- (0,5) node[anchor=south] {$y$};
\foreach \y in {-4,-3,-2,-1,0,1,2,3,4}
\draw (2pt, \y cm) -- (-2pt, \y cm);
\foreach \y in {-4,-3,-2,-1,1,2,3,4}
\node[anchor=west] at (0, \y cm) {\footnotesize \y};
\node[anchor=north west] at (0,0) {\footnotesize O};

\draw[color=blue,thick] (-5,-5) -- ( -3 , -3 );
;
\end{tikzpicture}
 



\newpage
 
$\gamma : (-5,  -3 ) \to \mathbb{R}^2$\\
$\gamma(t) := (t,t)$
 











\clearpage



\subsection{Parametrizing a  line }


\begin{tikzpicture}
\draw[thin,->] (-5,0) -- (5,0) node[anchor=west] {$x$};
\foreach \x in {-4,-3,-2,-1,0,1,2,3,4}
\draw (\x cm,2pt) -- (\x cm,-2pt);
\foreach \x in {-4,-3,-2,-1,1,2,3,4}
\node[anchor=north] at (\x cm,0) {\footnotesize \x};
\draw[thin,->] (0,-5) -- (0,5) node[anchor=south] {$y$};
\foreach \y in {-4,-3,-2,-1,0,1,2,3,4}
\draw (2pt, \y cm) -- (-2pt, \y cm);
\foreach \y in {-4,-3,-2,-1,1,2,3,4}
\node[anchor=west] at (0, \y cm) {\footnotesize \y};
\node[anchor=north west] at (0,0) {\footnotesize O};

\draw[color=blue,thick] (-5,-5) -- ( -2 , -2 );
;
\end{tikzpicture}
 



\newpage
 
$\gamma : (-5,  -2 ) \to \mathbb{R}^2$\\
$\gamma(t) := (t,t)$
 
 











\clearpage



\subsection{Parametrizing a  line }


\begin{tikzpicture}
\draw[thin,->] (-5,0) -- (5,0) node[anchor=west] {$x$};
\foreach \x in {-4,-3,-2,-1,0,1,2,3,4}
\draw (\x cm,2pt) -- (\x cm,-2pt);
\foreach \x in {-4,-3,-2,-1,1,2,3,4}
\node[anchor=north] at (\x cm,0) {\footnotesize \x};
\draw[thin,->] (0,-5) -- (0,5) node[anchor=south] {$y$};
\foreach \y in {-4,-3,-2,-1,0,1,2,3,4}
\draw (2pt, \y cm) -- (-2pt, \y cm);
\foreach \y in {-4,-3,-2,-1,1,2,3,4}
\node[anchor=west] at (0, \y cm) {\footnotesize \y};
\node[anchor=north west] at (0,0) {\footnotesize O};

\draw[color=blue,thick] (-5,-5) -- ( -1 , -1 );
;
\end{tikzpicture}
 



\newpage
 
$\gamma : (-5,  -1 ) \to \mathbb{R}^2$\\
$\gamma(t) := (t,t)$
 
 
 











\clearpage



\subsection{Parametrizing a  line }


\begin{tikzpicture}
\draw[thin,->] (-5,0) -- (5,0) node[anchor=west] {$x$};
\foreach \x in {-4,-3,-2,-1,0,1,2,3,4}
\draw (\x cm,2pt) -- (\x cm,-2pt);
\foreach \x in {-4,-3,-2,-1,1,2,3,4}
\node[anchor=north] at (\x cm,0) {\footnotesize \x};
\draw[thin,->] (0,-5) -- (0,5) node[anchor=south] {$y$};
\foreach \y in {-4,-3,-2,-1,0,1,2,3,4}
\draw (2pt, \y cm) -- (-2pt, \y cm);
\foreach \y in {-4,-3,-2,-1,1,2,3,4}
\node[anchor=west] at (0, \y cm) {\footnotesize \y};
\node[anchor=north west] at (0,0) {\footnotesize O};

\draw[color=blue,thick] (-5,-5) -- ( 0 , 0 );
;
\end{tikzpicture}
 



\newpage
 
$\gamma : (-5,  0 ) \to \mathbb{R}^2$\\
$\gamma(t) := (t,t)$
 
 
 
 











\clearpage



\subsection{Parametrizing a  line }


\begin{tikzpicture}
\draw[thin,->] (-5,0) -- (5,0) node[anchor=west] {$x$};
\foreach \x in {-4,-3,-2,-1,0,1,2,3,4}
\draw (\x cm,2pt) -- (\x cm,-2pt);
\foreach \x in {-4,-3,-2,-1,1,2,3,4}
\node[anchor=north] at (\x cm,0) {\footnotesize \x};
\draw[thin,->] (0,-5) -- (0,5) node[anchor=south] {$y$};
\foreach \y in {-4,-3,-2,-1,0,1,2,3,4}
\draw (2pt, \y cm) -- (-2pt, \y cm);
\foreach \y in {-4,-3,-2,-1,1,2,3,4}
\node[anchor=west] at (0, \y cm) {\footnotesize \y};
\node[anchor=north west] at (0,0) {\footnotesize O};

\draw[color=blue,thick] (-5,-5) -- ( 1 , 1 );
;
\end{tikzpicture}
 



\newpage
 
$\gamma : (-5,  1 ) \to \mathbb{R}^2$\\
$\gamma(t) := (t,t)$
 
 
 
 
 











\clearpage



\subsection{Parametrizing a  line }


\begin{tikzpicture}
\draw[thin,->] (-5,0) -- (5,0) node[anchor=west] {$x$};
\foreach \x in {-4,-3,-2,-1,0,1,2,3,4}
\draw (\x cm,2pt) -- (\x cm,-2pt);
\foreach \x in {-4,-3,-2,-1,1,2,3,4}
\node[anchor=north] at (\x cm,0) {\footnotesize \x};
\draw[thin,->] (0,-5) -- (0,5) node[anchor=south] {$y$};
\foreach \y in {-4,-3,-2,-1,0,1,2,3,4}
\draw (2pt, \y cm) -- (-2pt, \y cm);
\foreach \y in {-4,-3,-2,-1,1,2,3,4}
\node[anchor=west] at (0, \y cm) {\footnotesize \y};
\node[anchor=north west] at (0,0) {\footnotesize O};

\draw[color=blue,thick] (-5,-5) -- ( 2 , 2 );
;
\end{tikzpicture}
 



\newpage
 
$\gamma : (-5,  2 ) \to \mathbb{R}^2$\\
$\gamma(t) := (t,t)$
 
 
 
 
 
 











\clearpage



\subsection{Parametrizing a  line }


\begin{tikzpicture}
\draw[thin,->] (-5,0) -- (5,0) node[anchor=west] {$x$};
\foreach \x in {-4,-3,-2,-1,0,1,2,3,4}
\draw (\x cm,2pt) -- (\x cm,-2pt);
\foreach \x in {-4,-3,-2,-1,1,2,3,4}
\node[anchor=north] at (\x cm,0) {\footnotesize \x};
\draw[thin,->] (0,-5) -- (0,5) node[anchor=south] {$y$};
\foreach \y in {-4,-3,-2,-1,0,1,2,3,4}
\draw (2pt, \y cm) -- (-2pt, \y cm);
\foreach \y in {-4,-3,-2,-1,1,2,3,4}
\node[anchor=west] at (0, \y cm) {\footnotesize \y};
\node[anchor=north west] at (0,0) {\footnotesize O};

\draw[color=blue,thick] (-5,-5) -- ( 3 , 3 );
;
\end{tikzpicture}
 



\newpage
 
$\gamma : (-5,  3 ) \to \mathbb{R}^2$\\
$\gamma(t) := (t,t)$
 
 
 
 
 
 
 











\clearpage



\subsection{Parametrizing a  line }


\begin{tikzpicture}
\draw[thin,->] (-5,0) -- (5,0) node[anchor=west] {$x$};
\foreach \x in {-4,-3,-2,-1,0,1,2,3,4}
\draw (\x cm,2pt) -- (\x cm,-2pt);
\foreach \x in {-4,-3,-2,-1,1,2,3,4}
\node[anchor=north] at (\x cm,0) {\footnotesize \x};
\draw[thin,->] (0,-5) -- (0,5) node[anchor=south] {$y$};
\foreach \y in {-4,-3,-2,-1,0,1,2,3,4}
\draw (2pt, \y cm) -- (-2pt, \y cm);
\foreach \y in {-4,-3,-2,-1,1,2,3,4}
\node[anchor=west] at (0, \y cm) {\footnotesize \y};
\node[anchor=north west] at (0,0) {\footnotesize O};

\draw[color=blue,thick] (-5,-5) -- ( 4 , 4 );
;
\end{tikzpicture}
 



\newpage
 
$\gamma : (-5,  4 ) \to \mathbb{R}^2$\\
$\gamma(t) := (t,t)$
 
 
 
 
 
 
 
 











\clearpage



\subsection{Parametrizing a  line }


\begin{tikzpicture}
\draw[thin,->] (-5,0) -- (5,0) node[anchor=west] {$x$};
\foreach \x in {-4,-3,-2,-1,0,1,2,3,4}
\draw (\x cm,2pt) -- (\x cm,-2pt);
\foreach \x in {-4,-3,-2,-1,1,2,3,4}
\node[anchor=north] at (\x cm,0) {\footnotesize \x};
\draw[thin,->] (0,-5) -- (0,5) node[anchor=south] {$y$};
\foreach \y in {-4,-3,-2,-1,0,1,2,3,4}
\draw (2pt, \y cm) -- (-2pt, \y cm);
\foreach \y in {-4,-3,-2,-1,1,2,3,4}
\node[anchor=west] at (0, \y cm) {\footnotesize \y};
\node[anchor=north west] at (0,0) {\footnotesize O};

\draw[color=blue,thick] (-5,-5) -- ( 5 , 5 );
;
\end{tikzpicture}
 



\newpage
 
$\gamma : (-5,  5 ) \to \mathbb{R}^2$\\
$\gamma(t) := (t,t)$
 
 
 
 
 
 
 
 
 
















\clearpage



\subsection{Quick review: Derivative}



\clearpage



\subsection{Quick review: Derivative}
\begin{example}
  $f : \mathbb{R}\to \mathbb{R}$\end{example}


\clearpage



\subsection{Quick review: Derivative}
\begin{example}
  $f : \mathbb{R}\to \mathbb{R}$
  \[f(x) =
      \begin{cases}
       \end{cases}\]\end{example}


\clearpage



\subsection{Quick review: Derivative}
\begin{example}
  $f : \mathbb{R}\to \mathbb{R}$
  \[f(x) =
      \begin{cases}
        x^2 & x < 5\\
       \end{cases}\]\end{example}


\clearpage



\subsection{Quick review: Derivative}
\begin{example}
  $f : \mathbb{R}\to \mathbb{R}$
  \[f(x) =
      \begin{cases}
        x^2 & x < 5\\
        0 & x = 5\\
       \end{cases}\]\end{example}


\clearpage



\subsection{Quick review: Derivative}
\begin{example}
  $f : \mathbb{R}\to \mathbb{R}$
  \[f(x) =
      \begin{cases}
        x^2 & x < 5\\
        0 & x = 5\\
        x^3 & x > 5
      \end{cases}
      \]
      \end{example}


\clearpage



\subsection{Quick review: Derivative}
\begin{example}
  $f : \mathbb{R}\to \mathbb{R}$
  \[f(x) =
      \begin{cases}
        x^2 & x < 5\\
        0 & x = 5\\
        x^3 & x > 5
      \end{cases}
      \]
      $f(5)=0$\\
\end{example}


\clearpage



\subsection{Quick review: Derivative}
\begin{example}
  $f : \mathbb{R}\to \mathbb{R}$
  \[f(x) =
      \begin{cases}
        x^2 & x < 5\\
        0 & x = 5\\
        x^3 & x > 5
      \end{cases}
      \]
      $f(5)=0$\\
$\displaystyle\lim_{x \to 5^-} f(x) $\end{example}


\clearpage



\subsection{Quick review: Derivative}
\begin{example}
  $f : \mathbb{R}\to \mathbb{R}$
  \[f(x) =
      \begin{cases}
        x^2 & x < 5\\
        0 & x = 5\\
        x^3 & x > 5
      \end{cases}
      \]
      $f(5)=0$\\
$\displaystyle\lim_{x \to 5^-} f(x)  = 5^2$\\
\end{example}


\clearpage



\subsection{Quick review: Derivative}
\begin{example}
  $f : \mathbb{R}\to \mathbb{R}$
  \[f(x) =
      \begin{cases}
        x^2 & x < 5\\
        0 & x = 5\\
        x^3 & x > 5
      \end{cases}
      \]
      $f(5)=0$\\
$\displaystyle\lim_{x \to 5^-} f(x)  = 5^2$\\
$\displaystyle\lim_{x \to 5^+} f(x) $\end{example}


\clearpage



\subsection{Quick review: Derivative}
\begin{example}
  $f : \mathbb{R}\to \mathbb{R}$
  \[f(x) =
      \begin{cases}
        x^2 & x < 5\\
        0 & x = 5\\
        x^3 & x > 5
      \end{cases}
      \]
      $f(5)=0$\\
$\displaystyle\lim_{x \to 5^-} f(x)  = 5^2$\\
$\displaystyle\lim_{x \to 5^+} f(x)  = 5^3$
\end{example}





\clearpage



\subsection{Quick review: Derivative}
\begin{example}
  $f : \mathbb{R}\to \mathbb{R}$
  \[f(x) =
      \begin{cases}
        x^2 & x < 5\\
        0 & x = 5\\
        x^3 & x > 5
      \end{cases}
      \]
      $f(5)=0$\\
$\displaystyle\lim_{x \to 5^-} f(x)  = 5^2$\\
$\displaystyle\lim_{x \to 5^+} f(x)  = 5^3$
\end{example}


\begin{example}
  \[f(x) =
      \begin{cases}
        x^2 & x < 5\\
        0 & x = 5\\
       x^{\alert{2}} & x > 5
      \end{cases}
      \]
\end{example}


\clearpage



\subsection{Quick review: Derivative}
\begin{example}
  $f : \mathbb{R}\to \mathbb{R}$
  \[f(x) =
      \begin{cases}
        x^2 & x < 5\\
        0 & x = 5\\
        x^3 & x > 5
      \end{cases}
      \]
      $f(5)=0$\\
$\displaystyle\lim_{x \to 5^-} f(x)  = 5^2$\\
$\displaystyle\lim_{x \to 5^+} f(x)  = 5^3$
\end{example}


\begin{example}
  \[f(x) =
      \begin{cases}
        x^2 & x  \neq  5\\
        0 & x = 5\\
        
      \end{cases}
      \]
      
      


\end{example}


\clearpage



\subsection{Quick review: Derivative}
\begin{example}
  $f : \mathbb{R}\to \mathbb{R}$
  \[f(x) =
      \begin{cases}
        x^2 & x < 5\\
        0 & x = 5\\
        x^3 & x > 5
      \end{cases}
      \]
      $f(5)=0$\\
$\displaystyle\lim_{x \to 5^-} f(x)  = 5^2$\\
$\displaystyle\lim_{x \to 5^+} f(x)  = 5^3$
\end{example}


\begin{example}
  \[f(x) =
      \begin{cases}
        x^2 & x  \neq  5\\
        0 & x = 5\\
        
      \end{cases}
      \]
      
      


$\displaystyle\lim_{x \to 5^-} f(x) = 5^2 $\end{example}


\clearpage



\subsection{Quick review: Derivative}
\begin{example}
  $f : \mathbb{R}\to \mathbb{R}$
  \[f(x) =
      \begin{cases}
        x^2 & x < 5\\
        0 & x = 5\\
        x^3 & x > 5
      \end{cases}
      \]
      $f(5)=0$\\
$\displaystyle\lim_{x \to 5^-} f(x)  = 5^2$\\
$\displaystyle\lim_{x \to 5^+} f(x)  = 5^3$
\end{example}


\begin{example}
  \[f(x) =
      \begin{cases}
        x^2 & x  \neq  5\\
        0 & x = 5\\
        
      \end{cases}
      \]
      
      


$\displaystyle\lim_{x \to 5^-} f(x) = 5^2  = \displaystyle\lim_{x \to 5^+} f(x)$\\
\end{example}


\clearpage



\subsection{Quick review: Derivative}
\begin{example}
  $f : \mathbb{R}\to \mathbb{R}$
  \[f(x) =
      \begin{cases}
        x^2 & x < 5\\
        0 & x = 5\\
        x^3 & x > 5
      \end{cases}
      \]
      $f(5)=0$\\
$\displaystyle\lim_{x \to 5^-} f(x)  = 5^2$\\
$\displaystyle\lim_{x \to 5^+} f(x)  = 5^3$
\end{example}


\begin{example}
  \[f(x) =
      \begin{cases}
        x^2 & x  \neq  5\\
        0 & x = 5\\
        
      \end{cases}
      \]
      
      


$\displaystyle\lim_{x \to 5^-} f(x) = 5^2  = \displaystyle\lim_{x \to 5^+} f(x)$\\
Can say, $\displaystyle\lim_{x \to 5} f(x) = 5^2$
\end{example}

\newpage



\clearpage



\subsection{Quick review: Derivative}
\begin{example}
  $f : \mathbb{R}\to \mathbb{R}$
  \[f(x) =
      \begin{cases}
        x^2 & x < 5\\
        0 & x = 5\\
        x^3 & x > 5
      \end{cases}
      \]
      $f(5)=0$\\
$\displaystyle\lim_{x \to 5^-} f(x)  = 5^2$\\
$\displaystyle\lim_{x \to 5^+} f(x)  = 5^3$
\end{example}


\begin{example}
  \[f(x) =
      \begin{cases}
        x^2 & x  \neq  5\\
        0 & x = 5\\
        
      \end{cases}
      \]
      
      


$\displaystyle\lim_{x \to 5^-} f(x) = 5^2  = \displaystyle\lim_{x \to 5^+} f(x)$\\
Can say, $\displaystyle\lim_{x \to 5} f(x) = 5^2$
\end{example}

\newpage
\begin{example}
$f(x) = x^2$\\
\end{example}


\clearpage



\subsection{Quick review: Derivative}
\begin{example}
  $f : \mathbb{R}\to \mathbb{R}$
  \[f(x) =
      \begin{cases}
        x^2 & x < 5\\
        0 & x = 5\\
        x^3 & x > 5
      \end{cases}
      \]
      $f(5)=0$\\
$\displaystyle\lim_{x \to 5^-} f(x)  = 5^2$\\
$\displaystyle\lim_{x \to 5^+} f(x)  = 5^3$
\end{example}


\begin{example}
  \[f(x) =
      \begin{cases}
        x^2 & x  \neq  5\\
        0 & x = 5\\
        
      \end{cases}
      \]
      
      


$\displaystyle\lim_{x \to 5^-} f(x) = 5^2  = \displaystyle\lim_{x \to 5^+} f(x)$\\
Can say, $\displaystyle\lim_{x \to 5} f(x) = 5^2$
\end{example}

\newpage
\begin{example}
$f(x) = x^2$\\
$\displaystyle\lim_{x \to 5} f(x) = 5^2 $\end{example}


\clearpage



\subsection{Quick review: Derivative}
\begin{example}
  $f : \mathbb{R}\to \mathbb{R}$
  \[f(x) =
      \begin{cases}
        x^2 & x < 5\\
        0 & x = 5\\
        x^3 & x > 5
      \end{cases}
      \]
      $f(5)=0$\\
$\displaystyle\lim_{x \to 5^-} f(x)  = 5^2$\\
$\displaystyle\lim_{x \to 5^+} f(x)  = 5^3$
\end{example}


\begin{example}
  \[f(x) =
      \begin{cases}
        x^2 & x  \neq  5\\
        0 & x = 5\\
        
      \end{cases}
      \]
      
      


$\displaystyle\lim_{x \to 5^-} f(x) = 5^2  = \displaystyle\lim_{x \to 5^+} f(x)$\\
Can say, $\displaystyle\lim_{x \to 5} f(x) = 5^2$
\end{example}

\newpage
\begin{example}
$f(x) = x^2$\\
$\displaystyle\lim_{x \to 5} f(x) = 5^2 = f(5)$\end{example}


\clearpage



\subsection{Quick review: Derivative}
\begin{example}
  $f : \mathbb{R}\to \mathbb{R}$
  \[f(x) =
      \begin{cases}
        x^2 & x < 5\\
        0 & x = 5\\
        x^3 & x > 5
      \end{cases}
      \]
      $f(5)=0$\\
$\displaystyle\lim_{x \to 5^-} f(x)  = 5^2$\\
$\displaystyle\lim_{x \to 5^+} f(x)  = 5^3$
\end{example}


\begin{example}
  \[f(x) =
      \begin{cases}
        x^2 & x  \neq  5\\
        0 & x = 5\\
        
      \end{cases}
      \]
      
      


$\displaystyle\lim_{x \to 5^-} f(x) = 5^2  = \displaystyle\lim_{x \to 5^+} f(x)$\\
Can say, $\displaystyle\lim_{x \to 5} f(x) = 5^2$
\end{example}

\newpage
\begin{example}
$f(x) = x^2$\\
$\displaystyle\lim_{x \to 5} f(x) = 5^2 = f(5)$\\
$f$ is ``continous''.
\end{example}




\clearpage



\subsection{Quick review: Derivative}
\begin{example}
  $f : \mathbb{R}\to \mathbb{R}$
  \[f(x) =
      \begin{cases}
        x^2 & x < 5\\
        0 & x = 5\\
        x^3 & x > 5
      \end{cases}
      \]
      $f(5)=0$\\
$\displaystyle\lim_{x \to 5^-} f(x)  = 5^2$\\
$\displaystyle\lim_{x \to 5^+} f(x)  = 5^3$
\end{example}


\begin{example}
  \[f(x) =
      \begin{cases}
        x^2 & x  \neq  5\\
        0 & x = 5\\
        
      \end{cases}
      \]
      
      


$\displaystyle\lim_{x \to 5^-} f(x) = 5^2  = \displaystyle\lim_{x \to 5^+} f(x)$\\
Can say, $\displaystyle\lim_{x \to 5} f(x) = 5^2$
\end{example}

\newpage
\begin{example}
$f(x) = x^2$\\
$\displaystyle\lim_{x \to 5} f(x) = 5^2 = f(5)$\\
$f$ is ``continous''.
\end{example}

\begin{definition}[Continuous function]
$f : \mathbb{R} \to \mathbb{R}$ is continuous if $\displaystyle\lim_{x \to a} f(x) = f(a)$
\end{definition}




\clearpage



\subsection{Quick review: Derivative}
\begin{example}
  $f : \mathbb{R}\to \mathbb{R}$
  \[f(x) =
      \begin{cases}
        x^2 & x < 5\\
        0 & x = 5\\
        x^3 & x > 5
      \end{cases}
      \]
      $f(5)=0$\\
$\displaystyle\lim_{x \to 5^-} f(x)  = 5^2$\\
$\displaystyle\lim_{x \to 5^+} f(x)  = 5^3$
\end{example}


\begin{example}
  \[f(x) =
      \begin{cases}
        x^2 & x  \neq  5\\
        0 & x = 5\\
        
      \end{cases}
      \]
      
      


$\displaystyle\lim_{x \to 5^-} f(x) = 5^2  = \displaystyle\lim_{x \to 5^+} f(x)$\\
Can say, $\displaystyle\lim_{x \to 5} f(x) = 5^2$
\end{example}

\newpage
\begin{example}
$f(x) = x^2$\\
$\displaystyle\lim_{x \to 5} f(x) = 5^2 = f(5)$\\
$f$ is ``continous''.
\end{example}

\begin{definition}[Continuous function]
$f : \mathbb{R} \to \mathbb{R}$ is continuous if $\displaystyle\lim_{x \to a} f(x) = f(a)$
\end{definition}

\begin{definition}[Derivative]
  If $f : \mathbb{R}\to \mathbb{R}$ is such that\end{definition}


\clearpage



\subsection{Quick review: Derivative}
\begin{example}
  $f : \mathbb{R}\to \mathbb{R}$
  \[f(x) =
      \begin{cases}
        x^2 & x < 5\\
        0 & x = 5\\
        x^3 & x > 5
      \end{cases}
      \]
      $f(5)=0$\\
$\displaystyle\lim_{x \to 5^-} f(x)  = 5^2$\\
$\displaystyle\lim_{x \to 5^+} f(x)  = 5^3$
\end{example}


\begin{example}
  \[f(x) =
      \begin{cases}
        x^2 & x  \neq  5\\
        0 & x = 5\\
        
      \end{cases}
      \]
      
      


$\displaystyle\lim_{x \to 5^-} f(x) = 5^2  = \displaystyle\lim_{x \to 5^+} f(x)$\\
Can say, $\displaystyle\lim_{x \to 5} f(x) = 5^2$
\end{example}

\newpage
\begin{example}
$f(x) = x^2$\\
$\displaystyle\lim_{x \to 5} f(x) = 5^2 = f(5)$\\
$f$ is ``continous''.
\end{example}

\begin{definition}[Continuous function]
$f : \mathbb{R} \to \mathbb{R}$ is continuous if $\displaystyle\lim_{x \to a} f(x) = f(a)$
\end{definition}

\begin{definition}[Derivative]
  If $f : \mathbb{R}\to \mathbb{R}$ is such that
  \[\displaystyle\lim_{h\to 0}\frac{f(x+h) - f(x)}{h}\] exists,\end{definition}


\clearpage



\subsection{Quick review: Derivative}
\begin{example}
  $f : \mathbb{R}\to \mathbb{R}$
  \[f(x) =
      \begin{cases}
        x^2 & x < 5\\
        0 & x = 5\\
        x^3 & x > 5
      \end{cases}
      \]
      $f(5)=0$\\
$\displaystyle\lim_{x \to 5^-} f(x)  = 5^2$\\
$\displaystyle\lim_{x \to 5^+} f(x)  = 5^3$
\end{example}


\begin{example}
  \[f(x) =
      \begin{cases}
        x^2 & x  \neq  5\\
        0 & x = 5\\
        
      \end{cases}
      \]
      
      


$\displaystyle\lim_{x \to 5^-} f(x) = 5^2  = \displaystyle\lim_{x \to 5^+} f(x)$\\
Can say, $\displaystyle\lim_{x \to 5} f(x) = 5^2$
\end{example}

\newpage
\begin{example}
$f(x) = x^2$\\
$\displaystyle\lim_{x \to 5} f(x) = 5^2 = f(5)$\\
$f$ is ``continous''.
\end{example}

\begin{definition}[Continuous function]
$f : \mathbb{R} \to \mathbb{R}$ is continuous if $\displaystyle\lim_{x \to a} f(x) = f(a)$
\end{definition}

\begin{definition}[Derivative]
  If $f : \mathbb{R}\to \mathbb{R}$ is such that
  \[\displaystyle\lim_{h\to 0}\frac{f(x+h) - f(x)}{h}\] exists, then $f$ is ``differentiable''\end{definition}


\clearpage



\subsection{Quick review: Derivative}
\begin{example}
  $f : \mathbb{R}\to \mathbb{R}$
  \[f(x) =
      \begin{cases}
        x^2 & x < 5\\
        0 & x = 5\\
        x^3 & x > 5
      \end{cases}
      \]
      $f(5)=0$\\
$\displaystyle\lim_{x \to 5^-} f(x)  = 5^2$\\
$\displaystyle\lim_{x \to 5^+} f(x)  = 5^3$
\end{example}


\begin{example}
  \[f(x) =
      \begin{cases}
        x^2 & x  \neq  5\\
        0 & x = 5\\
        
      \end{cases}
      \]
      
      


$\displaystyle\lim_{x \to 5^-} f(x) = 5^2  = \displaystyle\lim_{x \to 5^+} f(x)$\\
Can say, $\displaystyle\lim_{x \to 5} f(x) = 5^2$
\end{example}

\newpage
\begin{example}
$f(x) = x^2$\\
$\displaystyle\lim_{x \to 5} f(x) = 5^2 = f(5)$\\
$f$ is ``continous''.
\end{example}

\begin{definition}[Continuous function]
$f : \mathbb{R} \to \mathbb{R}$ is continuous if $\displaystyle\lim_{x \to a} f(x) = f(a)$
\end{definition}

\begin{definition}[Derivative]
  If $f : \mathbb{R}\to \mathbb{R}$ is such that
  \[\displaystyle\lim_{h\to 0}\frac{f(x+h) - f(x)}{h}\] exists, then $f$ is ``differentiable'' and the limit is the derivative\end{definition}


\clearpage



\subsection{Quick review: Derivative}
\begin{example}
  $f : \mathbb{R}\to \mathbb{R}$
  \[f(x) =
      \begin{cases}
        x^2 & x < 5\\
        0 & x = 5\\
        x^3 & x > 5
      \end{cases}
      \]
      $f(5)=0$\\
$\displaystyle\lim_{x \to 5^-} f(x)  = 5^2$\\
$\displaystyle\lim_{x \to 5^+} f(x)  = 5^3$
\end{example}


\begin{example}
  \[f(x) =
      \begin{cases}
        x^2 & x  \neq  5\\
        0 & x = 5\\
        
      \end{cases}
      \]
      
      


$\displaystyle\lim_{x \to 5^-} f(x) = 5^2  = \displaystyle\lim_{x \to 5^+} f(x)$\\
Can say, $\displaystyle\lim_{x \to 5} f(x) = 5^2$
\end{example}

\newpage
\begin{example}
$f(x) = x^2$\\
$\displaystyle\lim_{x \to 5} f(x) = 5^2 = f(5)$\\
$f$ is ``continous''.
\end{example}

\begin{definition}[Continuous function]
$f : \mathbb{R} \to \mathbb{R}$ is continuous if $\displaystyle\lim_{x \to a} f(x) = f(a)$
\end{definition}

\begin{definition}[Derivative]
  If $f : \mathbb{R}\to \mathbb{R}$ is such that
  \[\displaystyle\lim_{h\to 0}\frac{f(x+h) - f(x)}{h}\] exists, then $f$ is ``differentiable'' and the limit is the derivative of $f$\end{definition}


\clearpage



\subsection{Quick review: Derivative}
\begin{example}
  $f : \mathbb{R}\to \mathbb{R}$
  \[f(x) =
      \begin{cases}
        x^2 & x < 5\\
        0 & x = 5\\
        x^3 & x > 5
      \end{cases}
      \]
      $f(5)=0$\\
$\displaystyle\lim_{x \to 5^-} f(x)  = 5^2$\\
$\displaystyle\lim_{x \to 5^+} f(x)  = 5^3$
\end{example}


\begin{example}
  \[f(x) =
      \begin{cases}
        x^2 & x  \neq  5\\
        0 & x = 5\\
        
      \end{cases}
      \]
      
      


$\displaystyle\lim_{x \to 5^-} f(x) = 5^2  = \displaystyle\lim_{x \to 5^+} f(x)$\\
Can say, $\displaystyle\lim_{x \to 5} f(x) = 5^2$
\end{example}

\newpage
\begin{example}
$f(x) = x^2$\\
$\displaystyle\lim_{x \to 5} f(x) = 5^2 = f(5)$\\
$f$ is ``continous''.
\end{example}

\begin{definition}[Continuous function]
$f : \mathbb{R} \to \mathbb{R}$ is continuous if $\displaystyle\lim_{x \to a} f(x) = f(a)$
\end{definition}

\begin{definition}[Derivative]
  If $f : \mathbb{R}\to \mathbb{R}$ is such that
  \[\displaystyle\lim_{h\to 0}\frac{f(x+h) - f(x)}{h}\] exists, then $f$ is ``differentiable'' and the limit is the derivative of $f$ at $x$, \end{definition}


\clearpage



\subsection{Quick review: Derivative}
\begin{example}
  $f : \mathbb{R}\to \mathbb{R}$
  \[f(x) =
      \begin{cases}
        x^2 & x < 5\\
        0 & x = 5\\
        x^3 & x > 5
      \end{cases}
      \]
      $f(5)=0$\\
$\displaystyle\lim_{x \to 5^-} f(x)  = 5^2$\\
$\displaystyle\lim_{x \to 5^+} f(x)  = 5^3$
\end{example}


\begin{example}
  \[f(x) =
      \begin{cases}
        x^2 & x  \neq  5\\
        0 & x = 5\\
        
      \end{cases}
      \]
      
      


$\displaystyle\lim_{x \to 5^-} f(x) = 5^2  = \displaystyle\lim_{x \to 5^+} f(x)$\\
Can say, $\displaystyle\lim_{x \to 5} f(x) = 5^2$
\end{example}

\newpage
\begin{example}
$f(x) = x^2$\\
$\displaystyle\lim_{x \to 5} f(x) = 5^2 = f(5)$\\
$f$ is ``continous''.
\end{example}

\begin{definition}[Continuous function]
$f : \mathbb{R} \to \mathbb{R}$ is continuous if $\displaystyle\lim_{x \to a} f(x) = f(a)$
\end{definition}

\begin{definition}[Derivative]
  If $f : \mathbb{R}\to \mathbb{R}$ is such that
  \[\displaystyle\lim_{h\to 0}\frac{f(x+h) - f(x)}{h}\] exists, then $f$ is ``differentiable'' and the limit is the derivative of $f$ at $x$, denoted $f'(x)$ or $\frac{df}{dx}$. \end{definition}


\clearpage



\subsection{Quick review: Derivative}
\begin{example}
  $f : \mathbb{R}\to \mathbb{R}$
  \[f(x) =
      \begin{cases}
        x^2 & x < 5\\
        0 & x = 5\\
        x^3 & x > 5
      \end{cases}
      \]
      $f(5)=0$\\
$\displaystyle\lim_{x \to 5^-} f(x)  = 5^2$\\
$\displaystyle\lim_{x \to 5^+} f(x)  = 5^3$
\end{example}


\begin{example}
  \[f(x) =
      \begin{cases}
        x^2 & x  \neq  5\\
        0 & x = 5\\
        
      \end{cases}
      \]
      
      


$\displaystyle\lim_{x \to 5^-} f(x) = 5^2  = \displaystyle\lim_{x \to 5^+} f(x)$\\
Can say, $\displaystyle\lim_{x \to 5} f(x) = 5^2$
\end{example}

\newpage
\begin{example}
$f(x) = x^2$\\
$\displaystyle\lim_{x \to 5} f(x) = 5^2 = f(5)$\\
$f$ is ``continous''.
\end{example}

\begin{definition}[Continuous function]
$f : \mathbb{R} \to \mathbb{R}$ is continuous if $\displaystyle\lim_{x \to a} f(x) = f(a)$
\end{definition}

\begin{definition}[Derivative]
  If $f : \mathbb{R}\to \mathbb{R}$ is such that
  \[ f'(x) := \displaystyle\lim_{h\to 0}\frac{f(x+h) - f(x)}{h}\] exists, then $f$ is ``differentiable'' and the limit is the derivative of $f$ at $x$, denoted $f'(x)$ or $\frac{df}{dx}$.  
\end{definition}


\end{document}