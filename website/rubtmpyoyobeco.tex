\documentclass[a4paper]{article}
\usepackage{amssymb}
\usepackage{amsthm}
\usepackage{amsmath}
\setcounter{secnumdepth}{0}
\newtheorem{theorem}{Theorem}
\title{Exercise sheet 5\\ {\small Theory of Computation, IDC204} }
\date{}
\begin{document}
\maketitle
\begin{enumerate}
\item
  Design a Turing machine over the alphabet {\(\{0, 1\}\)}, that accepts
  a string if it has an even number of 1s and rejects a string if it has
  an odd number of 1s.
\item
  Design a Turing machine over the alphabet {\(\{0, 1\}\)}, that shifts
  any string that is provided as input on the tape, one character to the
  right.
\item
  Given a finite state automaton that recognizes a language, how will
  you design a Turing machine that accepts each string that belongs to
  the language and rejects each string that does not belong to the
  language. As usual, the string is entered as input by writing it on
  the tape.
\item
  Given a Turing machine that accepts all strings in the languge {\(L\)}
  and rejects all strings that are not in it, how will you design a
  Turing machine that accepts the complement of the language?
\item
  Given a Turing machine {\(T_1\)} that accepts all strings in a
  language {\(L_1\)} and rejects all strings that are not in it and a
  Turing machine {\(T_2\)} that accepts all strings in a language
  {\(L_2\)} and rejects all strings that are not in it, how will you
  design a Turing machine that accepts strings in {\(L_1 \cup L_2\)} and
  rejects all the strings that are not it.
\item
  Design a Turing machine that takes a natural number represented in
  binary as input and replaces that input on the tape with the binary
  representation of the number obtained by adding 1 to it.
\end{enumerate}

\end{document}